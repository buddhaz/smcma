\documentclass[a4paper, 11pt]{ctexbook}

\usepackage[top=2cm, bottom = 2cm, left = 2.5cm, right = 2.5cm]{geometry}

\usepackage[bookmarksnumbered=true, colorlinks, linkcolor=black]{hyperref}

% \usepackage{ctex}
\usepackage{amsmath, amsfonts, amssymb}
\usepackage{enumerate}

\newcommand{\arccot}{\mathrm{arccot}\,}
\newcommand{\dif}{\mathrm{d}\,}

\title{数学分析教程习题解答}
\author{张书宁}
\date{\today}

\begin{document}
    \frontmatter
    \maketitle

    \tableofcontents
    \mainmatter

    \chapter{实数和数列极限}
        \section{实数}
        \section{数列和收敛数列}
            % \documentclass[a4paper, 12pt]{article}

% \usepackage[top=2cm, bottom = 2cm, left = 2.5cm, right = 2.5cm]{geometry}

% \usepackage{ctex}
% \usepackage{amsmath, amsfonts, amssymb}
% \usepackage{enumerate}

% \begin{document}

\begin{center}
    {\heiti 练习题 1.2}
\end{center}

\begin{enumerate}
    \item 1
    \item % 2
        {\heiti 证明}\quad 由题意有 $\forall \varepsilon > 0$, $\exists N \in \mathrm{N}^*$, 当 $n > N$ 时,有
        \begin{equation*}
            |a_n - a| < \varepsilon.
        \end{equation*}
        另一方面,有 $\vert a_n \vert - \vert a \vert = \vert a_n \vert - \vert -a \vert \leqslant \vert a_n - a \vert < \varepsilon$, 即 $\lim\limits_{n\to\infty}\vert a_n \vert = \vert a \vert$.
        
        当 $a_n = (-1)^{n-1}$ 时, 该命题的逆命题为假.
    \item 3
    \item 4
    \item 5
\end{enumerate}

% \end{document}

        \section{收敛数列的性质}
            \input{chapter-1/section-1.3/section-1.3.tex}
        \section{数列极限概念的推广}
            \input{chapter-1/section-1.4/section-1.4.tex}
        \section{单调数列}
            % %@author 张书宁

% \documentclass[12pt]{article}

% \usepackage[top=2cm, bottom=2cm, left=2.5cm, right=2.5cm]{geometry}

% \usepackage{ctex}
% \usepackage{amsmath, amssymb}
% \usepackage{enumerate}

% \begin{document}

% \pagestyle{empty}

\begin{center}
    {\heiti 练习题 1.5}
\end{center}

\begin{enumerate}
    \item 
        \begin{enumerate}[(1)]
            \item {\heiti 证明}\quad 显然数列 $\{x_n\}$ 有下界 $0$. 现证明 $\{x_n\}$ 是递减的. 考察 $x_n / x_{n+1}$, 即
                \[
                    \frac{x_n}{x_{n+1}} = \frac{\frac{10}{1}\cdot\frac{11}{3}\cdots\frac{n+9}{2n-1}}{\frac{10}{1}\cdot\frac{11}{3}\cdots\frac{n+9}{2n-1}\cdot\frac{n+10}{2n+1}} = \frac{2n+1}{n+10}.    
                \]
                当 $n \geqslant 9$ 时, 有 $(2n+1)/(n+10) \geqslant 1$, 即 $x_n \geqslant x_{n+1}$. 由定理 1.5.1 可知 $\{x_n\}$ 的极限存在.
            \item {\heiti 证明}\quad 先对通项 $x_n$ 进行化简,
                \begin{align*}
                    x_n &= \left(1 - \frac12\right)\left(1 - \frac13\right)\cdots\left(1 - \frac{1}{n+1}\right) \\
                        &= \frac12 \times \frac23 \times \cdots \times \frac{n}{n+1} \\
                        &= \frac{n!}{(n+1)!} \\
                        &= \frac{1}{n+1}.    
                \end{align*}
                显然 $x_n > 0$, 考察 $x_n/x_{n+1}$, 则有
                \[
                    \left. \frac{x_n}{x_{n+1}} = \frac{1}{n+1} \right/ \frac{1}{n+2} = \frac{n+2}{n+1} > 1.    
                \]
                即 $x_n > x_{n+1}$. 由定理 1.5.1 可知 $\{x_n\}$ 的极限存在.
        \end{enumerate}
    \item {\heiti 证明}\quad 由题意显然知道 $\{x_n\}$ 是一个递增数列. 现在来证明 $\{x_n\}$ 是有上界的.
        已知 $x_1 = \sqrt{2} < 2$. 现假设 $x_{n - 1} < 2$, 那么
        \[
            x_n = \sqrt{2 + \sqrt{x_{n-1}}} < \sqrt{2 + 2} = 2.    
        \]
        由数学归纳法可知 $x_n < 2$ 成立, 即 2 是数列 $\{x_n\}$ 的一个上界. 由定理 1.5.1 可知 $\{x_n\}$ 的极限存在.
    \item {\heiti 证明}\quad 设递增数列 $\{a_n\}$ 有一子列 $\{a_{m_n}\}$ 收敛.
        我们有 $\forall \varepsilon > 0$, $\exists N \in \mathrm{N}^*$, 当 $m_n \geqslant n > N$ 时, 有
        \[
            |a_{m_n} - a| < \varepsilon.    
        \]
        又因数列 $\{a_n\}$ 是递增的, 则有 $a_n \leqslant a_{m_n}$, 因此
        \[
            |a_n - a| \leqslant |a_{m_n} - a| < \varepsilon.    
        \]
        这样就证明了数列 $\{a_n\}$ 也是收敛的.

        若数列 $\{a_n\}$ 是递减的, 那么 $\{-a_n\}$ 就是递增的, 同理可证 $\{-a_n\}$ 存在极限 $a$,
        即 $-a$ 就是 $\{a_n\}$ 的极限.
    \item {\heiti 证明}\quad 考察 $a_n(1 - a_n)$, 则有
        \begin{align*}
            a_n(1 - a_n) &= a_n - a_n^2 + \frac14 - \frac14 \\
                         &= \frac14 - \left(a_n^2 - a_n + \frac14\right) \\
                         &= \frac14 - \left(a_n - \frac12\right)^2 \leqslant \frac14. 
        \end{align*}
        再结合已知条件, 则有 $a_n(1 - a_n) \leqslant 1/4 < a_{n+1}(1 - a_n)$, 即 $a_n < a_{n+1}$.
        又因 $0 < a_n < 1$, 所以数列 $\{a_n\}$ 的极限存在.

        现设 $\{a_n\}$ 的极限为 $a$. 已知不等式
        \[
            a_n(1 - a_n) \leqslant \frac14 < a_{n+1}(1 - a_n)    
        \]
        成立, 由夹逼原理可以得到 $a(1 - a) = 1/4$, 解这个一元二次方程, 即可得到 $a = 1/2$.

    \item {\heiti 证明}\quad 根据均值不等式
        \[
            \frac{n}{1 + \frac12 + \cdots + \frac1n} \leqslant \sqrt[n]{n!} = a_n,    
        \]
        则有
        \begin{align*}
            a_{n+1} - a_n &\geqslant \frac{n+1}{1 + \frac12 + \cdots + \frac{1}{n+1}} - \frac{n}{1 + \frac12 + \cdots + \frac1n} \\
                          &= \frac{(n+1)\left(1 + \frac12 + \cdots \frac1n\right) - n\left(1 + \frac12 + \cdots + \frac{1}{n+1}\right)}{\left(1 + \cdots + \frac1n\right)\left(1 + \cdots + \frac{1}{n+1}\right)} \\
                          &= \frac{\left(n + \frac n2 + \cdots + 1\right) + \left(1 + \frac12 + \cdots + \frac1n\right) - \left(n + \frac n2 + \cdots + 1\right) - \frac{n}{n+1}}{\left(1 + \cdots + \frac1n\right)\left(1 + \cdots + \frac{1}{n+1}\right)} \\
                          &= \frac{\left(1 + \frac12 + \cdots + \frac1n\right) - \frac{n}{n+1}}{\left(1 + \cdots + \frac1n\right)\left(1 + \cdots + \frac{1}{n+1}\right)} \\
                          &= \frac{\left(1 - \frac{1}{n+1}\right) + \left(\frac12 - \frac{1}{n+1}\right) + \cdots + \left(\frac1n - \frac{1}{n+1}\right)}{\left(1 + \cdots + \frac1n\right)\left(1 + \cdots + \frac{1}{n+1}\right)} > 0. \\
        \end{align*}
        因此 $\{(n!)^{1/n}\}$ 是递增数列.
    \item 6
\end{enumerate}
% \end{document}

        \section{自然对数的底 $\mathrm{e}$}
            \input{chapter-1/section-1.6/section-1.6.tex}
        \section{基本列和 Cauchy 收敛原理}
            \input{chapter-1/section-1.7/section-1.7.tex}
        \section{上确界和下确界}
            \input{chapter-1/section-1.8/section-1.8.tex}
        \section{有限覆盖定理}
        \section{上极限和下极限}
        \section{Stolz 定理}
            \input{chapter-1/section-1.11/section-1.11.tex}
    \chapter{函数的连续性}
        \section{集合的映射}
        \section{集合的势}
        \section{函数}
            %@author Shuning Zhang
%@date 2019-12-03

\documentclass[12pt]{article}

\usepackage[top=2cm, bottom=2cm, left=2.5cm, right=2.5cm]{geometry}

\usepackage{ctex}

\usepackage{enumerate}
\usepackage{amsmath, amssymb, amssymb, amsthm}

\begin{document}

\pagestyle{empty}

\begin{center}
    {\heiti 练习题 2.3}
\end{center}

\begin{enumerate}
    \item % 1
        略.
    \item % 2
        {\heiti 证明}\quad 先证明存在性, 用反证法. 假设 $f$ 不存在不动点. 因为 $f \circ f$ 有唯一的不动点 $x$,
        所以 $f(f(x)) = x$. 又因 $f$ 不存在不动点, 所以 $f(x) = y \ne x$. 那么
        \[
            f(y) = f(f(x)) = x,   
        \]
        即 $f(y) = x$. 更进一步, 有 $f(f(y)) = f(x) = y$, 即 $y$ 也是 $f \circ f$ 的一个不动点. 这与题意 $f \circ f$ 有唯一的不动点相矛盾.
        因此 $f$ 必然存在不动点.
        
        再证明唯一性, 依然用反证法. 假设 $x$ 和 $y$ 都是 $f$ 的不动点, 并且 $x \neq y$, 则有 $f(x) = x$, $f(y) = y$.
        那么
        \begin{gather*}
            f(f(x)) = f(x) = x, \\
            f(f(y)) = f(y) = y.
        \end{gather*}
        这与题意 $f \circ f$ 有唯一的不动点相矛盾. 因此 $f$ 的不动点是唯一的.
    \item 3
    \item 4
    \item % 5
        \begin{enumerate}[(1)]
            \item % 5.1
                $f^n(x) = \dfrac{x}{\sqrt{1 + nx^2}}$;
            \item % 5.2
                $f^n(x) = \dfrac{x}{1 + nbx}$.
        \end{enumerate}
    \item 6
    \item 7
    \item 8
    \item % 9
        {\heiti 证明} 设 $f(x)$ 是定义在 $(-a, a)$ 上的函数, 则有
        \[
            f(x) = \frac{f(x) + f(-x) + f(x) - f(-x)}{2} = \frac{f(x) + f(-x)}{2} + \frac{f(x) - f(-x)}{2}.    
        \]
        令 $g(x) = \dfrac{f(x) + f(-x)}{2}$, $h(x) = \dfrac{f(x) - f(-x)}{2}$, 显然有
        \begin{gather*}
            g(-x) = g(x), \\
            h(-x) = -h(x).
        \end{gather*}
    \item % 10
        \begin{enumerate}[(1)]
            \item % 10.1
                {\heiti 证明}\quad 对双曲正弦函数, 有
                \[
                    \sinh(-x) = \frac{\mathrm{e}^{-x} - \mathrm{e}^x}{2} = -\frac{\mathrm{e}^x - \mathrm{e}^{-x}}{2} = -\sinh x.    
                \]
                对双曲余弦函数, 有
                \[
                    \cosh(-x) = \frac{\mathrm{e}^{-x} + \mathrm{e}^x}{2} = \frac{\mathrm{e}^x + \mathrm{e}^{-x}}{2} = \cosh x.     
                \]
            \item % 10.2
                {\heiti 证明}\quad 根据题意, 有
                \begin{align*}
                    \cosh^2x - \sinh^2x &= \frac{(\mathrm{e}^x + \mathrm{e}^{-x})^2}{4} - \frac{(\mathrm{e}^x - \mathrm{e}^{-x})^2}{4} \\
                                        &= \frac{\mathrm{e}^{2x} + \mathrm{e}^{-2x} + 2\mathrm{e}^x\mathrm{e}^{-x} - (\mathrm{e}^{2x} + \mathrm{e}^{-2x} - 2\mathrm{e}^x\mathrm{e}^{-x})}{4} \\
                                        &= \frac{2 \cdot \mathrm{e}^0 + 2 \cdot \mathrm{e}^0}{4} \\
                                        &= 1.
                \end{align*}
        \end{enumerate}
    \item % 11
        令 $u = \mathrm{e}^x$, 则有
        \[
            u^2 - 2yu - 1 = 0,    
        \]
        解得 $u = y \pm \sqrt{y^2 + 1}$. 因为 $u = \mathrm{e}^x > 0$, 所以 $u = y + \sqrt{y^2 + 1}$, 即
        \[
            x = \ln(y + \sqrt{y^2 + 1}).    
        \]
\end{enumerate}

\end{document}

        \section{函数的极限}
            % %@author Shuning Zhang
% %@date 2019-12-04

% \documentclass[12pt]{ctexart}

% \usepackage[top=2cm, bottom=2cm, left=2.5cm, right=2.5cm]{geometry}

% \usepackage{enumerate}
% \usepackage{amsmath, amsfonts, amssymb, amsthm}

% \begin{document}

% \pagestyle{empty}

\begin{center}
    {\heiti 练习题 2.4}
\end{center}

\begin{enumerate}
    \item % 1
        对 $\forall \varepsilon > 0$, $\exists \delta > 0$, 当 $0 < x_0 - x < \delta$ 时, 有 $|f(x) - 1| < \varepsilon$.
    \item % 2
    \item % 3
        \begin{enumerate}[(1)]
            \item % 3.1
                {\heiti 证明}\quad 对 $\forall \varepsilon > 0$, $\exists \delta > 0$, 当 $0 < |x - x_0| < \delta$ 时, 有
                \[
                    ||f(x)| - |A|| \leqslant |f(x) - A| < \varepsilon.    
                \]
                这便证明了 $\lim\limits_{x\to x_0}|f(x)| = |A|$.
            \item % 3.2
                {\heiti 证明}\quad 因为 $f(x)$ 的极限存在, 所以 $f(x)$ 是有界的, 即 $|f(x)| \leqslant M$.
                对 $\forall \varepsilon > 0$, 取 $\delta = \varepsilon/(M + |A|)$, 当 $0 < |x - x_0| < \delta$ 时, 有
                \[
                    |f^2(x) - A^2| = |f(x) + A||f(x) - A| \leqslant (|f(x)| + |A|)|f(x) - A| < \varepsilon.
                \]
            \item % 3.3
                略.
            \item % 3.4
                略.
        \end{enumerate}
    \item % 4
        \begin{enumerate}[(1)]
            \item % 4.1
                {\heiti 证明}\quad 对 $\forall \varepsilon > 0$, 取 $\delta = \min(1, \varepsilon/19)$, 当 $0 < |x - 2| < \delta$ 时, 有
                \[
                    |x^3 - 8| = |x - 2||x^2 + 2x + 2^2| < 19|x - 2| < \varepsilon.   
                \]
            \item % 4.2
                {\heiti 证明}\quad 对 $\forall \varepsilon > 0$, 取 $\delta = \min(1, 30\varepsilon)$, 当 $0 < |x - 3| < \delta$ 时, 有
                \[
                    \left|\frac{x-3}{x^2-9} - \frac16\right| = \left|\frac{1}{x+3} - \frac16\right| = \frac{|x-3|}{6|x+3|} < \frac{|x-3|}{30} < \varepsilon.
                \]
            \item % 4.3
                {\heiti 证明}\quad 对 $\forall \varepsilon > 0$, 取 $\delta = \min(1, \varepsilon/11)$, 当 $0 < |x - 1| < \delta$ 时, 有
                \begin{align*}
                    \left| \frac{x^4-1}{x-1} - 4 \right| &= |(x^2+1)(x+1) - 4| = |x^3 + x^2 + x - 3| \\
                                                         &= |x^3 - 1 + x^2 - 1 + x - 1| = |x - 1||x^2 + 2x + 3| < 11|x - 1| < \varepsilon.
                \end{align*}
            \item % 4.4
                略.
            \item % 4.5
                略.
        \end{enumerate}
    \item % 5
        \begin{enumerate}[(1)]
            \item % 5.1
                $f(2+) = 4$, $f(2-) = -2a$;
            \item % 5.2
                $a = -2$.
        \end{enumerate}
    \item % 6
        {\heiti 证明}\quad 令 $\lim\limits_{x\to x_0}f(x) = l > a$. 对 $\varepsilon_0 = l - a > 0$, $\exists \delta > 0$, 当 $0 < |x - x_0| < \delta$ 时, 有
        \[
            |f(x) - l| < \varepsilon_0,    
        \]
        即 $a = l - \varepsilon_0 < f(x)$.
    \item % 7
        {\heiti 证明}\quad 对 $\varepsilon_0 = \dfrac{f(x_0+) - f(x_0-)}{2} > 0$, $\exists \delta_1 > 0$, 当 $0 < x_0 - x < \delta_1$ 时, 有
        \[
            |f(x) - f(x_0-)| < \varepsilon_0,    
        \]
        即 $f(x) < f(x_0-) + \varepsilon_0$. 同时, $\exists \delta_2 > 0$, 当 $0 < y - x_0 < \delta_2$ 时, 有
        \[
            |f(y) - f(x_0+)| < \varepsilon_0,    
        \]
        即 $f(x_0+) - \varepsilon_0 < f(y)$. 现取 $\delta = \min(\delta_1, \delta_2)$, 当 $0 < x_0 - x < \delta$, $0 < y - x_0 < \delta$ 时, 有
        \[
            f(x) < f(x_0-) + \varepsilon_0 = f(x_0+) - \varepsilon_0 < f(y).
        \]
    \item % 8
    \item % 9
        存在一个 $\varepsilon_0 > 0$, 对任意的 $\delta > 0$, 即使 $0 < |x - x_0| < \delta$, 依然有 $|f(x) - l| \geqslant \varepsilon_0$.
    \item % 10
    \item % 11
        \begin{enumerate}[(1)]
            \item % 11.1
                $-1$;
            \item % 11.2
                $0$;
            \item % 11.3
                $\lim\limits_{x\to1}\dfrac{x^m-1}{x-1} = \lim\limits_{x\to1}\dfrac{1-x^m}{1-x} = \lim\limits_{x\to1}(1 + x + \cdots + x^{m-1}) = m$;
            \item % 11.4
                $\lim\limits_{x\to1}\dfrac{x^m-1}{x^n-1} = \lim\limits_{x\to1}\dfrac{(1-x^m)/(1-x)}{(1-x^n)/(1-x)} = \dfrac mn$;
            \item % 11.5
                $1/2$;
            \item % 11.6
                $1$;
            \item % 11.7
                $1/m$;
            \item % 11.8
                $\dfrac{m(m+1)}{2}$.
        \end{enumerate}
    \item % 12
        \begin{enumerate}[(1)]
            \item % 12.1
                $a/b$;
            \item % 12.2
                $2$;
            \item % 12.3
                $1$;
            \item % 12.4
                $1$;
            \item % 12.5
                $\sin x$;
            \item % 12.6
                $\cos x$;
            \item % 12.7
                $\lim\limits_{x\to0}\dfrac{1-\cos x\cos 2x\cdots\cos nx}{x^2}$ \\
                $= \lim\limits_{x\to0} \dfrac{1 - (\cos 2x\cos 3x\cdots + \cos nx) + (\cos 2x\cos 3x\cdots + \cos nx) -\cos x\cos 2x\cdots\cos nx}{x^2}$ \\
                $= \lim\limits_{x\to0} \dfrac{1 - \cos 2x\cos 3x\cdots + \cos nx}{x^2} + \lim\limits_{x\to0} (\cos 2x\cos 3x\cdots + \cos nx) \dfrac{1 - \cos x}{x^2}$ \\
                $= \lim\limits_{x\to0} \dfrac{1 - \cos 3x\cos 4x\cdots + \cos nx}{x^2} + \lim\limits_{x\to0} (\cos 3x\cos 4x\cdots + \cos nx) \dfrac{1 - \cos2x}{x^2} + \dfrac12$ \\
                $= \lim\limits_{x\to0} \dfrac{1 - \cos nx}{x^2} + \dfrac{(n-1)^2}{2} + \dfrac{(n-2)^2}{2} + \cdots + \dfrac12$ \\
                $= \dfrac{1 + 2^2 + \cdots + n^2}{2}$ \\
                $= \dfrac{n(n+1)(2n+1)}{12}$;
            \item % 12.8
                $\lim\limits_{n\to\infty}\cos\dfrac{x}{2}\cos\dfrac{x}{2^2}\cdots\cos\dfrac{x}{2^n}$ \\
                $= \lim\limits_{n\to\infty}\cos\dfrac{x}{2}\cos\dfrac{x}{2^2}\cdots\cos\dfrac{x}{2^n} \cdot \lim\limits_{n\to\infty} \dfrac{\sin(x/2^n)}{x/2^n}$ \\
                $= \lim\limits_{n\to\infty} \dfrac{2^{n-1}}{x} \cos\dfrac{x}{2} \cos\dfrac{x}{2^2} \cdots \cos\dfrac{x}{2^{n-1}}\left(2\cos\dfrac{x}{2^n}\sin\dfrac{x}{2^n}\right)$ \\
                $= \lim\limits_{n\to\infty} \dfrac{2^{n-2}}{x} \cos\dfrac{x}{2} \cos\dfrac{x}{2^2} \cdots \left(2\cos\dfrac{x}{2^{n-1}}\sin\dfrac{x}{2^{n-1}}\right)$ \\
                $= \lim\limits_{n\to\infty} \dfrac1x\left(2\cos\dfrac{x}{2}\sin\dfrac{x}{2}\right)$ \\
                $= \lim\limits_{n\to\infty} \dfrac{\sin x}{x}$ \\
                $= \dfrac{\sin x}{x}$.
        \end{enumerate}
    \item % 13
        \begin{enumerate}[(1)]
            \item % 13.1
                $1$;
            \item % 13.2
                $0$;
            \item % 13.3
                $5/8$;
            \item % 13.4
                $3/2$.
        \end{enumerate}
    \item % 14
        {\heiti 证明}\quad 考察 $n!\mathrm{e}$, 有
        \[
            n!\mathrm{e} = n! \sum_{k=0}^\infty\frac{1}{k!} = n!\left(\sum_{k=0}^n\frac{1}{k!} + \sum_{k=n+1}^\infty\frac{1}{k!}\right) = n! \sum_{k=0}^n\frac{1}{k!} + n!\sum_{k=n+1}^\infty\frac{1}{k!} = A + B.
        \]
        显然 $A \in \mathrm{N}^*$, 对于 $B$, 则有
        \[
            B = n!\sum_{k=n+1}^\infty\frac{1}{k!} = \frac{1}{n+1} + \frac{1}{(n+1)(n+2)} + \cdots.    
        \]
        
        现在来证明 $\dfrac{1}{n+1} < B < \dfrac{1}{n-1}$. 不等式 $\dfrac{1}{n+1} < B$ 是显然成立的, 对于 $B < \dfrac{1}{n-1}$, 有
        \[
            B = \frac{1}{n+1} + \frac{1}{(n+1)(n+2)} + \cdots < \frac1n + \frac{1}{n^2} + \cdots = \sum_{i=1}^\infty\left(\frac1n\right)^i,
        \]
        其中
        \[
            \sum_{i=1}^\infty\left(\frac1n\right)^i = \lim_{m\to\infty}\sum_{i=1}^m\left(\frac1n\right)^i = \lim_{m\to\infty}\frac{\frac1n\left(1 - \left(\frac1n\right)^m\right)}{1 - \frac1n} = \frac{1}{n-1}.   
        \]
        
        现在, 则有
        \begin{align*}
            n\sin(2\pi n!\mathrm{e}) &= n\sin\left(2\pi (A+B)\right) \\
                                     &= n\sin(2A\pi)\cos(2B\pi) + n\sin(2B\pi)\cos(2A\pi) \\
                                     &= n\sin(2B\pi). 
        \end{align*}
        因为 $\dfrac{1}{n+1} < B < \dfrac{1}{n-1}$, 所以 $\lim\limits_{n\to\infty}B = 0$, 那么
        \[
            \lim\limits_{n\to\infty} n\sin(2B\pi) = \lim\limits_{n\to\infty} 2\pi nB \frac{\sin(2B\pi)}{2B\pi} = 2\pi\lim\limits_{n\to\infty}nB. 
        \]
        因为 $\dfrac{n}{n+1} < nB < \dfrac{n}{n-1}$, 所以 $\lim\limits_{n\to\infty}nB = 1$, 即
        \[
            \lim\limits_{n\to\infty} n\sin(2\pi n!\mathrm{e}) = 2\pi.    
        \]
    \item % 15
        对 $t_0 = 0$ 的任何一个 $B_\eta(t_0)$, 都能找到一个无理数 $p/q \in B_\eta(t_0)$, 使得 $g(p/q) = 1/q = x_0$. 这违背了复合函数求极限的条件.
    \item % 16
\end{enumerate}

% \end{document}
        \section{极限过程的其他形式}
            % %@author Shuning Zhang
% %@date 2019-12-05

% \documentclass[12pt]{ctexart}

% \usepackage[top=2cm, bottom=2cm, left=2.5cm, right=2.5cm]{geometry}

% \usepackage{enumerate}
% \usepackage{amsmath, amsfonts, amssymb, amsthm}

% \begin{document}

% \pagestyle{empty}

\begin{center}
    {\heiti 练习题 2.5}
\end{center}

\begin{enumerate}
    \item % 1
        略.
    \item % 2
        略.
    \item % 3
        {\heiti 证明}\quad 令 $y = \sin\sqrt{x + 1} - \sin\sqrt{x-1}$. 对 $y$ 进行化简, 则有
        \begin{align*}
            y &= \sin\sqrt{x + 1} - \sin\sqrt{x-1} \\
              &= 2\sin\frac{\sqrt{x+1} - \sqrt{x-1}}{2}\cos\frac{\sqrt{x+1} + \sqrt{x-1}}{2} \\
              &= 2\sin\frac{1}{\sqrt{x+1} + \sqrt{x-1}}\cos\frac{\sqrt{x+1} + \sqrt{x-1}}{2}.
        \end{align*}
        因为 $\lim\limits_{n\to+\infty}\sin\dfrac{1}{\sqrt{x+1} + \sqrt{x-1}} = 0$, 而 $-1 \leqslant \cos\dfrac{\sqrt{x+1} + \sqrt{x-1}}{2} \leqslant 1$, 所以
        \[
            \lim_{n\to+\infty} y = 0.    
        \]
    \item % 4
        {\heiti 证明}\quad 因为 $a_1 + a_2 + \cdots + a_n = 0$, 所以 $a_1 = -a_2 - a_3 - \cdots - a_n$, 那么
        \begin{align*}
            \lim_{n\to+\infty}\sum_{i=1}^n a_i\sin\sqrt{x+i} &= \lim_{n\to+\infty}\left(a_1\sin\sqrt{x+1} + \sum_{i=2}^n a_i\sin\sqrt{x+i}\right) \\
                                                             &= \lim_{n\to+\infty}\left(\sin\sqrt{x+1}\sum_{i=2}^n(-a_i) + \sum_{i=2}^n a_i\sin\sqrt{x+i}\right) \\
                                                             &= \lim_{n\to+\infty}\sum_{i=2}^n a_i(\sin\sqrt{x+i} - \sin\sqrt{x+1}) \\
                                                             &= \lim_{n\to+\infty}\sum_{i=2}^n 2a_i\sin\frac{\sqrt{x+i} - \sqrt{x+1}}{2}\cos\frac{\sqrt{x+i} + \sqrt{x+1}}{2} \\
                                                             &= \lim_{n\to+\infty}\sum_{i=2}^n 2a_i\sin\frac{i-1}{2(\sqrt{x+i} + \sqrt{x+1})}\cos\frac{\sqrt{x+i} + \sqrt{x+1}}{2}.
        \end{align*}
        利用第 3 题的结论, 可得
        \begin{align*}
                & \lim_{n\to+\infty}\sum_{i=2}^n 2a_i\sin\frac{i-1}{2(\sqrt{x+i} + \sqrt{x+1})}\cos\frac{\sqrt{x+i} + \sqrt{x+1}}{2} \\
            ={} & \underbrace{0 + 0 + \cdots + 0}_{\text{$n-1$ 个}} \\
            ={} & 0.
        \end{align*}
    \item % 5
        令 $f(n) = \sin(\pi\sqrt{n^2+1})$. 对 $f(n)$ 进行化简, 则有
        \begin{align*}
            f(n) &= \sin(\pi\sqrt{n^2+1}) \\
                 &= \sin(\pi\sqrt{n^2+1} - \pi + \pi) \\
                 &= \sin\left(\pi(\sqrt{n^2+1} - 1) + \pi\right) \\
                 &= \sin\left(\pi(\sqrt{n^2+1} - 1)\right)\cos\pi + \cos\left(\pi(\sqrt{n^2+1} - 1)\right)\sin\pi \\
                 &= -\sin\left(\pi(\sqrt{n^2+1} - 1)\right) \\
                 &= -\sin\frac{n^2\pi}{\sqrt{n^2+1} + 1} \\
                 &= -\sin\frac{\pi}{\sqrt{1+\frac{1}{n^2}} + \frac{1}{n^2}}.
        \end{align*}
        因此 $\lim\limits_{n\to\infty}f(n) = 0$.
    \item % 6
        \begin{enumerate}[(1)]
            \item % 6.1
                $1/e^2$;
            \item % 6.2
                $e^{2a}$;
            \item % 6.3
                $1/e^2$;
            \item % 6.4
                $e^{x+2}$.
        \end{enumerate}
    \item % 7
        根据题意, 有
        \begin{align*}
            f(x) &= \lim_{n\to\infty}n^x\left(\left(1+\frac1n\right)^{n+1} - \left(1+\frac1n\right)^n\right) \\
                 &= \lim_{n\to\infty} n^{x-1} \left(1+\frac1n\right)^n \\
                 &= \mathrm{e} \lim_{n\to\infty} n^{x-1}.
        \end{align*}
        因此
        \[
            f(x) =
                \begin{cases}
                    0, & x < 1; \\
                    \mathrm{e}, & x = 1. \\
                \end{cases}
        \]
        当 $x > 1$ 时, $f(x)$ 不存在.
    \item % 8
        {\heiti 证明}\quad 由题目已知条件 $\left|\sum\limits_{i=1}^n a_i\sin ix\right| \leqslant |\sin x|$, 有
        \[
            \frac{\left|\sum\limits_{i=1}^n a_i\sin ix\right|}{|\sin x|} \leqslant 1,   
        \]
        那么
        \[
            \left| \frac{a_1\sin x}{x}\cdot\frac{x}{\sin x} + \frac{a_2\sin2x}{2x}\cdot\frac{2x}{\sin x} + \cdots + \frac{a_n\sin x}{nx}\cdot\frac{nx}{\sin x} \right| \leqslant 1.
        \]
        对上面的不等式取极限, 令 $x \to 0$, 即可得到
        \[
            |a_1 + 2a_2 + \cdots + na_n| \leqslant 1.
        \]
    \item % 9
        {\heiti 证明}\quad 先证明必要性. 设 $\lim\limits_{x\to+\infty}f(x) = l$. 对 $\forall \varepsilon/2 > 0$, $\exists A_1 > 0$, 当 $x_1 > A_1$ 时, 有
        \[
            |f(x_1) - l| < \frac\varepsilon2.    
        \]
        同时, $\exists N_2 > 0$, 当 $x_2 > A_2$ 时, 有
        \[
            |f(x_2) - l| < \frac\varepsilon2.    
        \]
        现取 $A = \max(A_1, A_2)$, 当 $x_1 > A$, $x_2 > A$ 时, 有
        \begin{align*}
            |f(x_1) - f(x_2)| &= |f(x_1) - l + l - f(x_2)| \\
                              &\leqslant |f(x_1) - l| + |f(x_2) - l| \\
                              &< \frac\varepsilon2 + \frac\varepsilon2 = \varepsilon.
        \end{align*}
        再证明充分性. 考察任一发散到 $+\infty$ 的数列 $\{x_n\}$. 因为 $x_n\rightarrow+\infty\ (n\to\infty)$, 所以对 $\forall A > 0$, $\exists N \in \mathrm{N}^*$, 当 $n > N$ 时, 有
        \[
            x_n > A.    
        \]
        进一步, 对 $\forall \varepsilon > 0$, $\exists A > 0$, 对这个 $A$, $\exists N \in \mathrm{N}^*$, 当 $m, n > N$ 时, 有
        \[
            x_m > A, x_n > A,    
        \]
        那么 $|f(x_m) - f(x_n)| < \varepsilon$. 这表明数列 $\{f(x_n)\}$ 是一个基本列. 因此 $\{f(x_n)\}$ 的极限存在, 即 $\lim\limits_{x\to+\infty}f(x)$ 存在.
    \item % 10
        {\heiti 证明}\quad 设 $f(x)$ 的最小正周期为 $T$, 对 $\forall \varepsilon > 0$, $\exists x_0 > 0$, 当 $x \in [x_0, x_0 + T]$, 有
        \[
            |f(x)| < \varepsilon.    
        \]
        另一方面, 对 $\forall x' \in [x_0 + (n-1)T, x_0 + nT]\ (n \in \mathrm{Z})$, $\exists x \in [x_0, x_0 + T]$, 使得
        \[
            f(x) = f(x + (n-1)T) = f(x').    
        \]
        而 $\bigcup\limits_{n\in\mathrm{Z}}[x_0 + (n-1)T, x_0 + nT] = \mathrm{R}$. 这表明对 $\forall x \in \mathrm{R}$, 都有 $|f(x)| < \varepsilon$. 因此 $f = 0$.
\end{enumerate}

% \end{document}
        \section{无穷小与无穷大}
            % %@author Shuning Zhang
% %@date 2019-12-05

% \documentclass[12pt]{ctexart}

% \usepackage[top=2cm, bottom=2cm, left=2.5cm, right=2.5cm]{geometry}

% \usepackage{enumerate}
% \usepackage{amsmath, amsfonts, amssymb, amsthm}

% \begin{document}

\pagestyle{empty}

\begin{center}
    {\heiti 练习题 2.6}
\end{center}

\begin{enumerate}
    \item % 1
        \begin{enumerate}[(1)]
            \item % 1.1
                略.
            \item % 1.2
                略.
            \item % 1.3
                略.
            \item % 1.4
                略.
            \item % 1.5
                略.
            \item % 1.6
                略.
            \item % 1.7
                略.
            \item % 1.8
                $\lim\limits_{x\to0}\dfrac{\sqrt{x^2 + x^{1/3}}}{x^{1/6}} = \lim\limits_{x\to0}\sqrt{1 + x^{5/3}} = 1$.
            \item % 1.9
                $\lim\limits_{x\to\infty}\dfrac{\sqrt{x^2 + x^{1/3}}}{x} = \lim\limits_{x\to\infty}\sqrt{1 + x^{-5/3}} = 1$.
            \item % 1.10
                $\lim\limits_{x\to0}\dfrac{\sqrt{1+x} - \sqrt{1-x}}{x} = \lim\limits_{x\to0} \dfrac{2x}{x(\sqrt{1+x} + \sqrt{1-x})} = 1$.
            \item % 1.11
                $\lim\limits_{x\to0^+}\dfrac{\sqrt{1+\sqrt{1+\sqrt{x}}} - \sqrt{2}}{\sqrt{x}} = \lim\limits_{x\to0^+} \dfrac{\sqrt{1 + \sqrt{x}} - 1}{\sqrt{x}\left(\sqrt{1+\sqrt{1+\sqrt{x}}} + \sqrt{2}\right)}$ \\
                $= \lim\limits_{x\to0^+} \dfrac{1}{\left(\sqrt{1+\sqrt{1+\sqrt{x}}} + \sqrt{2}\right)\left(\sqrt{1 + \sqrt{x}} + 1\right)} = \dfrac{1}{4\sqrt{2}}$.
            \item % 1.12
                $\lim\limits_{x\to0}\dfrac{\sqrt{1+\tan x} - \sqrt{1-\sin x}}{x} = \lim\limits_{x\to0}\dfrac{\tan x + \sin x}{x(\sqrt{1+\tan x} + \sqrt{1-\sin x})}$ \\
                $= \lim\limits_{x\to0}\dfrac{1/\cos x + 1}{\sqrt{1+\tan x} + \sqrt{1-\sin x}} = 1$.
            \item % 1.13
                $\lim\limits_{x\to\infty}\dfrac{\sqrt{x + \sqrt{x + \sqrt{x}}}}{\sqrt{x}} = \lim\limits_{x\to\infty}\sqrt{1 + \sqrt{1/x + \sqrt{1/x^3}}} = 1$.
            \item % 1.14
                $\lim\limits_{x\to+\infty}\dfrac{(1+x)(1+x^2)\cdots(1+x^n)}{x^{\frac{n(n+1)}{2}}} = \lim\limits_{x\to+\infty} \dfrac{(1+x)(1+x^2) \cdots (1+x^n)}{x \cdot x^2 \cdots x^n}$ \\
                $= \lim\limits_{x\to+\infty} \left(1 + \dfrac{1}{x}\right)\left(1 + \dfrac{1}{x^2}\right) \cdots \left(1 + \dfrac{1}{x^n}\right) = 1$.
        \end{enumerate}
    \item % 2
        \begin{enumerate}[(1)]
            \item % 2.1
                略.
            \item % 2.2
                略.
            \item % 2.3
                略.
            \item % 2.4
                {\heiti 证明}\quad 根据题意, 有
                \[
                    \frac1\alpha\left(\frac{1}{1+\alpha} - (1-\alpha)\right) = o(1).    
                \]
                对上面等式的左边进行化简, 则有
                \[
                    \frac1\alpha\left(\frac{1}{1+\alpha} - (1-\alpha)\right) = \frac1\alpha \cdot \frac{\alpha^2}{1+\alpha} = \frac{\alpha}{1 + \alpha}.
                \]
                当 $x \to x_0$ 时, 显然有
                \[
                    \frac{\alpha}{1 + \alpha} = 0 = o(1).  
                \]
        \end{enumerate}
    \item % 3
        \begin{enumerate}[(1)]
            \item % 3.1
                $2$;
            \item % 3.2
                $1$;
            \item % 3.3
                $0$;
            \item % 3.4
                $1$;
            \item % 3.5
                $\dfrac{1}{2n}$.
        \end{enumerate}
\end{enumerate}

% \end{document}
        \section{连续函数}
        \section{连续函数与极限计算}
            \input{chapter-2/section-2.8/section-2.8.tex}
        \section{函数的一致连续性}
        \section{有限闭区间上连续函数的性质}
        \section{函数的上极限和下极限}
        \section{混沌现象}
    \chapter{函数的导数}
        \section{导数的定义}
        \section{导数的计算}
        \section{高阶导数}
        \section{微分学的中值定理}
        \section{利用导数研究函数}
        \section{L'Hospital 法则}
        \section{函数作图}
    \chapter{一元微分学的顶峰——Taylor 定理}
        \section{函数的微分}
            % author: Shuning Zhang
% creation date: 2019-12-25
\documentclass[a4paper, 11pt]{ctexart}
\usepackage[top=2cm, bottom=2cm, left=2.5cm, right=2.5cm]{geometry}
\usepackage{enumerate}
\usepackage{amsmath, amssymb}
\newcommand{\diff}{\mathrm{d}\,}
\begin{document}
\pagestyle{empty}
\begin{enumerate}
    \item % 1
        \begin{enumerate}[(1)]
            \item % 1.1
                $\diff y = -\dfrac{1}{x^2}\diff x$;
            \item % 1.2
                $\diff y = \dfrac{a}{1 + (ax+b)^2}\diff x$;
            \item % 1.3
                $\diff y = x\sin x\diff x$;
            \item % 1.4
                $\diff y = \dfrac{1}{\sqrt{x^2 + a^2}}\diff x$.
        \end{enumerate}
    \item % 2
        \begin{enumerate}[(1)]
            \item % 2.1
                $\ln x$;
            \item % 2.2
                $\arctan x$;
            \item % 2.3
                $2\sqrt{x}$;
            \item % 2.4
                $x^2 + x$;
            \item % 2.5
                $\sin x - \cos x$;
            \item % 2.6
                $\ln(x + \sqrt{1+x^2})$;
            \item % 2.7
                $-\dfrac{\mathrm{e}^{-ax}}{a}$;
            \item % 2.8
                $\dfrac{\sin^2x}{2}$;
            \item % 2.9
                $\ln\ln x$;
            \item % 2.10
                $\sqrt{x^2 + a^2}$;
            \item % 2.11
                $-\dfrac{\cos^3x}{3}$;
            \item % 2.12
        \end{enumerate}
    \item % 3
    \item % 4
        \begin{enumerate}[(1)]
            \item % 4.1
                $\dfrac{1}{1 + \mathrm{e}^y}$;
            \item % 4.2
                $\dfrac{y}{1 + y}$;
            \item % 4.3
                $-\dfrac{b^2y}{a^2x}$;
            \item % 4.4
                $-\sqrt{\dfrac yx}$;
            \item % 4.5
                $\dfrac{x^3 + y^2\sqrt{x^2+y^2}}{2xy\sqrt{x^2+y^2} - x^2y}$.
        \end{enumerate}
    \item % 5
\end{enumerate}
\end{document}
        \section{带 Peano 余项的 Taylor 定理}
            % author: Shuning Zhang
% creation date: 2019-12-29
\documentclass[a4paper, 11pt]{ctexart}
\usepackage[top=2cm, bottom=2cm, left=2.5cm, right=2.5cm]{geometry}
\usepackage{enumerate}
\usepackage{amsmath, amssymb}
\begin{document}
\pagestyle{empty}
\begin{enumerate}
    \item % 1
        \begin{enumerate}[(1)]
            \item % 1.1
                $-\dfrac{1}{12}$;
            \item % 1.2
                $1$;
            \item % 1.3
                $\dfrac16$;
            \item % 1.4
                $\ln^2a$.
        \end{enumerate}
    \item % 2
        略.
    \item % 3
        $\sqrt{e}$.
\end{enumerate}
\end{document}
        \section{带 Lagrange 余项和 Cauchy 余项的 Taylor 定理}
    \chapter{求导的逆运算}
        \section{原函数的概念}
        \section{分部积分法和换元法}
        \section{有理函数的原函数}
        \section{可有理化函数的原函数}
    \chapter{函数的积分}
        \section{积分的概念}
        \section{可积函数的性质}
        \section{微积分基本定理}
        \section{分部积分与换元}
        \section{可积分理论}
        \section{Lebesgue 定理}
        \section{反常积分}
        \section{数值积分}
    \chapter{积分学的应用}
        \section{积分学在几何学中的应用}
        \section{物理应用举例}
        \section{面积原理}
        \section{Wallis 公式和 Stirling 公式}
    \chapter{多变量函数的连续性}
        \section{$n$ 维 Euclid 空间}
        \section{$\mathrm{R}^n$ 中点列的极限}
        \section{$\mathrm{R}^n$ 中的开集和闭集}
        \section{列紧集和紧致集}
        \section{集和的连通性}
        \section{多变量函数的极限}
        \section{多变量连续函数}
        \section{连续映射}
    \chapter{多变量函数}
        \section{方向导数和偏导数}
        \section{多变量函数的微分}
        \section{映射的微分}
        \section{复合求导}
        \section{曲线的切线和曲面的切平面}
        \section{隐函数定理}
        \section{隐映射定理}
        \section{逆映射定理}
        \section{高阶偏导数}
        \section{中值定理和 Taylor 定理}
        \section{极值}
        \section{条件极值}
\end{document}
