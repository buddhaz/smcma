\documentclass[a4paper, 12pt]{ctexbook}

\usepackage[top=2cm, bottom = 2cm, left = 2.5cm, right = 2.5cm]{geometry}

\usepackage[bookmarksnumbered=true, colorlinks, linkcolor=black]{hyperref}

% \usepackage{ctex}
\usepackage{amsmath, amsfonts, amssymb}
\usepackage{enumerate}

\newcommand{\arccot}{\mathrm{arccot}\,}

\title{数学分析教程习题解答}
\author{张书宁}
\date{}

\begin{document}
    \maketitle

    \tableofcontents
    \mainmatter

    \chapter{实数和数列极限}
        \section{实数}
        \section{数列和收敛数列}
            % \documentclass[a4paper, 12pt]{article}

% \usepackage[top=2cm, bottom = 2cm, left = 2.5cm, right = 2.5cm]{geometry}

% \usepackage{ctex}
% \usepackage{amsmath, amsfonts, amssymb}
% \usepackage{enumerate}

% \newcommand{\arccot}{\mathrm{arccot}\,}

% \begin{document}

\begin{center}
    {\heiti 练习题 1.2}
\end{center}

\begin{enumerate}
    \item % 1
        \begin{enumerate}[(1)]
            \item % 1.1
                {\heiti 证明}\quad 根据题意, 则有
                \[
                    \left|\frac{1}{1+\sqrt{n}} - 0\right| = \frac{1}{1+\sqrt{n}} < \frac{1}{\sqrt{n}}.    
                \]
                对 $\forall\varepsilon > 0$, 只要取 $N = [1/\varepsilon^2]$, 当 $n > N$ 时, 便有
                \[
                    \left|\frac{1}{1+\sqrt{n}} - 0\right| < \varepsilon. 
                \]
            \item % 1.2
                {\heiti 证明}\quad 根据题意, 则有
                \[
                    \left|\frac{\sin n}{n} - 0\right| = \frac{|\sin n|}{n} \leqslant \frac1n.    
                \]
                对 $\forall\varepsilon > 0$, 只要取 $N = [1/\varepsilon]$, 当 $n > N$ 时, 便有
                \[
                    \left|\frac{\sin n}{n} - 0\right| < \varepsilon.    
                \]
            \item % 1.3
                {\heiti 证明}\quad 利用均值不等式
                \[
                    \sqrt[n]{n!} = \sqrt[n]{1\cdot2\cdots n} \leqslant \frac{1 + 2 + \cdots + n}{n}.   
                \]
                便可得到
                \[
                    \left|\frac{n!}{n^n} - 0\right| = \frac{\sqrt[n]{n!}}{n} \leqslant \frac{1+2+\cdots+n}{n^2} = \frac{\frac{n(n+1)}{2}}{n^2} = \frac12\left(1 + \frac1n\right) < 1 + \frac1n.
                \]
                对 $\forall\varepsilon > 0$, 取 $N = [1/(\varepsilon-1)]$, 当 $n > N$ 时, 便有
                \[
                    \left|\frac{n!}{n^n} - 0\right| < \varepsilon.   
                \]
            \item % 1.4
                {\heiti 证明}\quad 由题意, 可得到
                \[
                    \left|\frac{(-1)^{n-1}}{n} - 0\right| = \frac{|(-1)^{n-1}|}{n} = \frac1n.    
                \]
                对 $\forall\varepsilon > 0$, 取 $N = [1/\varepsilon]$, 当 $n > N$ 时, 便有
                \[
                    \left|\frac{(-1)^{n-1}}{n} - 0\right| < \varepsilon.   
                \]
            \item % 1.5
                {\heiti 证明}\quad 当 $n \geqslant 3$ 时, 有
                \[
                    \left|\frac{2n+3}{5n-10} - \frac25\right| = \frac{2n+3-2(n-2)}{5(n-2)} = \frac{7}{5(n-2)} < \frac{2}{n-2}.    
                \]
                对 $\forall\varepsilon > 0$, 取 $N = [2/\varepsilon + 2]$, 当 $n > N$ 时, 便有
                \[
                    \left|\frac{2n+3}{5n-10} - \frac25\right| < \varepsilon.   
                \]
            \item % 1.6
                {\heiti 证明}\quad 根据题意, 可得到
                \[
                    |0.\underbrace{99\cdots9}_{\text{$n$ 个}} - 1| = 0.\underbrace{00\cdots0}_{\text{$n$ 个}}9\cdots < \frac{1}{10^n}.    
                \]
                对 $\forall\varepsilon > 0$, 取 $N = [\lg(1/\varepsilon)]$, 当 $n > N$ 时, 便有
                \[
                    |0.\underbrace{99\cdots9}_{\text{$n$ 个}} - 1| < \varepsilon.   
                \]
                这便证明了 $\lim\limits_{n\to\infty}0.\underbrace{99\cdots9}_{\text{$n$ 个}} = 1$.
            \item % 1.7
                {\heiti 证明}\quad 因为
                \[
                    \left|\frac{1+2+\cdots+n}{n^2} - \frac12\right| = \frac{n(n+1)-n^2}{2n^2} = \frac{1}{2n}.    
                \]
                所以对 $\forall\varepsilon > 0$, 取 $N = [1/2\varepsilon]$, 当 $n > N$ 时, 便有
                \[
                    \left|\frac{1+2+\cdots+n}{n^2} - \frac12\right| < \varepsilon.   
                \]
            \item % 1.8
                略.
            \item % 1.9
                {\heiti 证明}\quad 利用等式
                \[
                    \arccot x = \frac{\pi}{2} - \arctan x.    
                \]
                则有
                \[
                    \left|\arctan n - \frac{\pi}{2}\right| = \arccot n.    
                \]
                对 $\forall\varepsilon > 0$, 取 $N = [\cot\varepsilon]$, 当 $n > N$ 时, 便有
                \[
                    \left|\arctan n - \frac{\pi}{2}\right| = \arccot n < \varepsilon.   
                \]
            \item 1.10
                % {\heiti 证明} 因为
                % \[
                %     \frac{n^2}{1+n^2} < 1\quad(n\in\mathrm{N}^*).    
                % \]
                % 所以
                % \[
                %     \left|\frac{n^2\arctan n}{1+n^2} - \frac{\pi}{2}\right| < \left|\arctan n - \frac{\pi}{2}\right|.    
                % \]
                % 根据第 $(9)$ 题的结论, 可知
                % \[
                %     \lim_{n\to\infty}\frac{n^2\arctan n}{1+n^2} = \frac{\pi}{2}.    
                % \]
        \end{enumerate}
    \item % 2
        {\heiti 证明}\quad 由题意, 对 $\forall \varepsilon > 0$, $\exists N \in \mathrm{N}^*$, 当 $n > N$ 时,使得不等式
        \begin{equation*}
            |a_n - a| < \varepsilon.
        \end{equation*}
        成立. 另一方面,有 $| |a_n| - |a| | \leqslant \vert a_n - a \vert < \varepsilon$, 即 $\lim\limits_{n\to\infty}\vert a_n \vert = \vert a \vert$.
        
        当 $a_n = (-1)^{n-1}$ 时, 该命题的逆命题为假.
    \item % 3
        {\heiti 证明}\quad 设数列 $\{a_n\}$ 从第 $N$ 项开始后面的项全部等于常数 $c$, 现在来证明 $c$ 就是 $\{a_n\}$ 的极限.
        对 $\forall\varepsilon > 0$, 要使 $|a_n - c| < \varepsilon$ 成立, 只要 $n \geqslant N$ 即可.
        因为当 $n \geqslant N$ 时,有
        \[
            |a_n - c| = |c - c| = 0 < \varepsilon.    
        \]
        这便证明了 $\lim\limits_{n\to\infty}a_n = c$.
    \item 4
    \item % 5
        对 $\forall N \in \mathrm{N}^*$, 依然 $\exists M > 0$, 即便 $n > N$, 也有不等式
        \[
            |a_n - a| \geqslant M    
        \]
        成立.
    \item 6
    \item 7
\end{enumerate}

% \end{document}

        \section{收敛数列的性质}
            % \documentclass[a4paper, 12pt]{article}

% \usepackage[top=2cm, bottom = 2cm, left = 2.5cm, right = 2.5cm]{geometry}

% \usepackage{ctex}
% \usepackage{amsmath, amsfonts, amssymb}
% \usepackage{enumerate}

% \begin{document}

\begin{center}
    {\heiti 练习题 1.3}
\end{center}

\begin{enumerate}
    \item 略.
    \item
        {\heiti 证明}\quad 因为 $\lim\limits_{n\to\infty}a_n = a$, 所以 $\lim\limits_{n\to\infty}\dfrac{1}{a_n} = \dfrac1a$,
        由定理 1.3.2 可知 $\left\{\dfrac{1}{a_n}\right\}$ 是有界的, 即
        \[
            \frac{1}{|a_n|}\leqslant\frac1M.
        \]
        对 $\forall \varepsilon > 0$, $\exists N \in \mathrm{N}^*$, 当 $n > N$ 时,便有
        \begin{equation*}
            |a_n - a| < \frac{M\varepsilon}{2}.
        \end{equation*}
        同时有
        \begin{align*}
            \left| \frac{a_{n+1}}{a_n} - 1 \right| &= \left| \frac{a_{n+1} - a + a - a_n}{a_n} \right| \\
                                                   &\leqslant \frac{|a_{n+1} - a| + |a_n - a|}{|a_n|} \\
                                                   &< \frac1M\left( \frac{M\varepsilon}{2} + \frac{M\varepsilon}{2} \right) = \varepsilon.
        \end{align*}
        这便证明了 $\lim\limits_{n\to\infty}\dfrac{a_{n+1}}{a_n} = 1$.
    
        当 $a_n = q^n\ (0 < |q| < 1)$ 时, $\lim\limits_{n\to\infty}\dfrac{a_{n+1}}{a_n} = q$.
    \item 
        \begin{enumerate}[(1)]
            \item
                $ \lim\limits_{n\to\infty}\dfrac{3^n + (-2)^n}{3^{n+1} + (-2)^{n+1}} = \lim\limits_{n\to\infty}\dfrac{1 + \left(-\dfrac 23\right)^n}{3 - 2\left(-\dfrac 23\right)^{n}} = \dfrac 13 $.
            \item
                $ \lim\limits_{n\to\infty}\left( \dfrac{1 + 2 + \cdots + n}{n + 2} - \dfrac n2 \right) = \lim\limits_{n\to\infty}\dfrac{2\cdot\dfrac n2 \left(n + 1\right) - n^2 - 2n}{2\left(n + 2\right)} = -\dfrac 12 $.
            \item
                $ \lim\limits_{n\to\infty}\sqrt{n}\left(\sqrt{n + 1} - \sqrt{n}\right) = \lim\limits_{n\to\infty}\dfrac{\sqrt{n}}{\sqrt{n + 1} + \sqrt{n}} = \dfrac 12 $.
            \item
                先对通项进行化简,
                \begin{align*}
                    (\sqrt{n^2 + n} - n)^{\frac 1n} &= \left(\frac{n}{\sqrt{n^2 +n} + n}\right)^{\frac 1n} \\
                                                    &= \left(\frac{1}{\sqrt{1 + \frac 1n} + 1}\right)^{\frac 1n}.
                \end{align*}
                易知
                \[
                    \frac 13 < \frac{1}{\sqrt{1 + \frac 1n} + 1} < \frac 12.    
                \]
            
                由夹逼原理可得 $\lim\limits_{n\to\infty}(\sqrt{n^2 + n} - n)^{\frac 1n} = 1$.
            \item 
                根据题意有 $\dfrac 12 \leqslant 1 - \dfrac 1n < 1\ (n \geqslant 2)$.
                由夹逼原理可得 $\lim\limits_{n\to\infty}\left(1 - \dfrac 1n\right)^{\frac 1n} = 1$.
            \item 
                根据题意有 $2 < n^2 - n + 2 = n^2 - (n - 2) \leqslant n^2\ (n \geqslant 2)$.
                由夹逼原理可得 \[\lim\limits_{n\to\infty}(n^2 - n + 2)^{\frac 1n} = 1.\]
            \item 
                因为 $\dfrac{\pi}{4} < \arctan n < \dfrac{\pi}{2}\ (n\in\mathrm{N}^*)$, 所以 $\lim\limits_{n\to\infty}(\arctan n)^{\frac{1}{n}} = 1$.
            \item 
                根据题意有 $1 \leqslant (2\sin^2n + \cos^2n) = (\sin^2n + 1) \leqslant 2$.
                由夹逼原理可得 \[\lim\limits_{n\to\infty}(2\sin^2n + \cos^2n)^{\frac{1}{n}} = 1.\]
        \end{enumerate}
    \item
        \begin{enumerate}[(1)]
            \item 
                $\lim\limits_{n\to\infty}\dfrac{\frac{1-a^n}{1-a}}{\frac{1-b^n}{1-b}} = \dfrac{1-b}{1-a}$.
            \item 
                令 $a_n = \dfrac{1}{1\cdot2} + \dfrac{1}{2\cdot3} + \cdots + \dfrac{1}{n\left(n+1\right)}$, 对 $a_n$ 进行化简, 则有
                \begin{align*}
                    a_n &= \dfrac{1}{1\cdot2} + \dfrac{1}{2\cdot3} + \cdots + \dfrac{1}{n\left(n+1\right)} \\
                        &= 1 - \dfrac 12 + \dfrac 12 - \dfrac 13 + \cdots + \dfrac 1n - \dfrac{1}{n+1} \\
                        &= 1 - \dfrac{1}{n+1}.
                \end{align*}
                因此 $\lim\limits_{n\to\infty}a_n = 1$.
            \item
                令 $a_n = (1-1/2^2)(1-1/3^2)\cdots(1-1/n^2)$, 对 $a_n$ 进行化简, 则有
                \begin{align*}
                    a_n &= \left(1-\frac{1}{2^2}\right)\left(1-\frac{1}{3^2}\right)\cdots\left(1-\frac{1}{n^2}\right) \\
                        &= \frac{2^2-1}{2^2} \cdot \frac{3^2-1}{3^2} \cdot \cdots \cdot \frac{n^2-1}{n^2} \\
                        &= \frac{(2+1)(3+1)\cdots(n+1)}{n!} \cdot \frac{(2-1)(3-1)\cdots(n-1)}{n!} \\
                        &= \frac{n+1}{2} \cdot \frac{1}{n} = \frac 12\left(1 + \frac 1n \right).
                \end{align*}
                因此 $\lim\limits_{n\to\infty}a_n = 1/2$.
            \item
                先对通项进行化简,
                \begin{align*}
                        & \left(1 - \frac{1}{1+2}\right)\left(1 - \frac{1}{1+2+3}\right)\cdots\left(1-\frac{1}{1+2+\cdots+n}\right) \\
                    ={} & \left(\frac{2}{3}\right)\left(\frac{5}{6}\right) \left(\frac{9}{10}\right) \cdots\left(1-\frac{2}{n(n+1)}\right) \\
                    ={} & \left(\frac{4}{6}\right) \left(\frac{10}{12}\right) \left(\frac{18}{20}\right) \cdots \left(\frac{(n-1)(n+2)}{n(n+1)}\right) \\
                    ={} & \left(\frac12 \cdot \frac43\right) \left(\frac23\cdot\frac54\right) \left(\frac34\cdot\frac65\right) \cdots \left(\frac{n-1}{n} \cdot \frac{n+2}{n+1}\right) \\
                    ={} & \left(\frac12 \cdot \frac23 \cdot \frac34 \cdot \cdots \cdot \frac{n-1}{n} \right)\left(\frac43 \cdot \frac54 \cdot \frac65 \cdot \cdots \cdots \frac{n+2}{n+1} \right) \\
                    ={} & \frac{1}{n} \cdot \frac{n+2}{3} = \frac 13 \left(1 + \frac 2n\right).
                \end{align*}
                因此
                \begin{equation*}
                    \lim\limits_{n\to\infty}\left(1 - \frac{1}{1+2}\right)\left(1 - \frac{1}{1+2+3}\right)\cdots\left(1-\frac{1}{1+2+\cdots+n}\right) = \frac13.
                \end{equation*}
            \item 
                因为
                \[
                    \frac{n}{2} \leqslant |1 - 2 + 3 - 4 + \cdots + (-1)^{n-1}n| \leqslant \frac{n+1}{2},
                \]
                由夹逼原理即可得到
                \[
                    \lim\limits_{n\to\infty}\frac{|1 - 2 + 3 - 4 + \cdots + (-1)^{n-1}n|}{n} = \frac 12.
                \]
            \item
                先对通项进行化简,
                \begin{align*}
                        & (1 + x)(1 + x^2) \cdots (1+x^{2^{n-1}}) \\
                    ={} & \frac{1}{1-x}(1 - x)(1 + x)(1 + x^2) \cdots (1 + x^{2^{n-1}}) \\
                    ={} & \frac{1}{1-x}(1 - x^{2^{n-1}}).
                \end{align*}
                因此
                \[
                    \lim\limits_{n\to\infty}(1 + x)(1 + x^2)\cdots(1 + x^{2^{n-1}}) = \frac{1}{1-x}.
                \]
        \end{enumerate}
    \item % 5
        \begin{enumerate}[(1)]
            \item % 5.1
                因为 $0 \leqslant a^n \leqslant b^n$, 所以 $b^n \leqslant a^n + b^n \leqslant 2b^n$, 那么
                \[
                    b \leqslant (a^n + b^n)^{1/n} \leqslant 2^{1/n} b.
                \]
                由夹逼原理即可得到 $\lim\limits_{n\to\infty}(a^n + b^n)^{1/n} = b$.
            \item % 5.2
                设 $a_j = \max(a_1, a_2, \cdots, a_m)$, 则有
                \[
                    0 \leqslant \underbrace{a_1^2 + a_2^2 + \cdots + a_{j-1}^2 + a_{j+1}^2 + \cdots + a_m^2}_{\text{$m-1$ 项}} \leqslant (m-1)a_j^n,    
                \]
                那么
                \[
                    a_j^n \leqslant a_1^n + a_2^n + \cdots + a_m^n \leqslant ma_j^n.    
                \]
                由夹逼原理即可得到
                \[
                    \lim_{n\to\infty}(a_1^n + a_2^n + \cdots + a_m^n)^{1/n} = a_j = \max(a_1, a_2, \cdots, a_m).
                \]
        \end{enumerate}
    \item
        {\heiti 证明}\quad 利用不等式 $a - 1 < [a] \leqslant a$, 则有
        \begin{equation*}
            a_n - \frac 1n = \frac{na_n - 1}{n} < \frac{[na_n]}{n} \leqslant \frac{na_n}{n} = a_n,
        \end{equation*}
        由夹逼原理即可得到 $\lim\limits_{n\to\infty}\dfrac{[na_n]}{n} = a$.
    \item 
        {\heiti 证明}\quad 利用均值不等式
        \[\frac{n}{\frac{1}{a_1} + \frac{1}{a_2} + \cdots + \frac{1}{a_n}} \leqslant \sqrt[n]{a_1a_2\cdots a_n} \leqslant \frac{a_1 + a_2 + \cdots + a_n}{n}.\]
        先看右边的不等式, 由例 4 可得
        \[
            \lim_{n\to\infty}\dfrac{a_1 + a_2 + \cdots + a_n}{n} = a.
        \]
        再看左边的不等式, 将例 4 中的 $a_n$ 换为 $1/a_n$, 则有
        \[
            \lim_{n\to\infty}\dfrac{\frac{1}{a_1} + \frac{1}{a_2} + \cdots + \frac{1}{a_n}}{n} = \dfrac 1a,
        \]
        那么
        \[
            \lim_{n\to\infty}\dfrac{n}{\frac{1}{a_1} + \frac{1}{a_2} + \cdots + \frac{1}{a_n}} = a.
        \]
    
        因此 $\lim\limits_{n\to\infty}\sqrt[n]{a_1a_2\cdots a_n} = a$.
    \item
        \begin{enumerate}[(1)]
            \item 
                {\heiti 证明}\quad 记 $b_n = a_{n+1}/a_n$, 则有 $\lim\limits_{n\to\infty}b_n = a$, 那么
                \begin{align*}
                    \lim_{n\to\infty}\sqrt[n]{a_n} &= \lim_{n\to\infty}\sqrt[n]{\frac{a_n}{a_1}} \\
                                                   &= \lim_{n\to\infty}\sqrt[n]{\frac{a_2}{a_1} \cdot \frac{a_3}{a_2} \cdots \frac{a_n}{a_{n-1}}} \\
                                                   &= \lim_{n\to\infty}\sqrt[n]{b_1 \cdot b_2 \cdots b_n} = a.
                \end{align*}
            \item 
                {\heiti 证明}\quad 对数列 $\{a_n\}$, 令 $a_1 = a$, $a_2 = a_3 = \cdots = a_n = 1$ 即可.
            \item 
                {\heiti 证明}\quad 对数列 $\{a_n\}$, 令 $a_1 = n$, $a_2 = a_3 = \cdots = a_n = 1$ 即可.
            \item 
                {\heiti 证明}\quad 设 $a_n = 1/n$, 则有 $\lim\limits_{n\to\infty}a_n = 0$, 那么
                \begin{align*}
                    \lim_{n\to\infty}(n!)^{-\frac{1}{n}} &= \lim_{n\to\infty}\left(\frac{1}{n!}\right)^\frac{1}{n} \\
                                                         &= \lim_{n\to\infty}\sqrt[n]{\frac{1}{1} \cdot \frac{1}{2} \cdots \frac{1}{n}} \\
                                                         &= \lim_{n\to\infty}\sqrt[n]{a_1 \cdot a_2 \cdots a_n} = 0.
                \end{align*}
        \end{enumerate}
    \item
        {\heiti 证明}\quad 利用不等式
        \[
            \frac{\frac 1n + \frac 1n + \cdots \frac 1n}{n} \leqslant \frac{1 + \frac 12 + \cdots + \frac 1n}{n} \leqslant \sqrt{\frac{1 + \frac{1}{2^2} + \cdots + \frac{1}{n^2}}{n}}.    
        \]
        因为
        \begin{align*}
            1 + \frac{1}{2^2} + \cdots + \frac{1}{n^2} &\leqslant 1 + \frac{1}{1 \cdot 2} + \cdots + \frac{1}{(n-1) \cdot n} \\
            &= 1 + 1 - \frac 12 + \cdots + \frac{1}{n - 1} - \frac 1n \\
            &= 2 - \frac 1n < 2.
        \end{align*}
        所以
        \[
            \frac 1n \leqslant \frac{1 + \frac 12 + \cdots + \frac 1n}{n} < \sqrt{\frac 2n}.
        \]

        由夹逼原理可得 $\lim\limits_{n\to\infty}\dfrac{1 + \frac 12 + \cdots + \frac 1n}{n} = 0$.
    \item % 10
        {\heiti 证明}\quad 设 $a_n = a_0\sqrt{n} + \cdots + a_p\sqrt{n}$, $b_n = a_0\sqrt{n + p} + \cdots + a_p\sqrt{n + p}$. 则有
        \[
            a_n \leqslant a_0\sqrt{n} + a_1\sqrt{n + 1} + \cdots + a_p\sqrt{n + p} \leqslant b_n.
        \]
        而 $\lim\limits_{n\to\infty}a_n = \lim\limits_{n\to\infty}a_n = 0$, 因此由夹逼原理可得
        \[
            \lim\limits_{n\to\infty}(a_0\sqrt{n} + a_1\sqrt{n + 1} + \cdots + a_p\sqrt{n + p}) = 0.    
        \]
    \item % 11
        {\heiti 证明}\quad 因为
        \begin{align*}
                & \lim_{n\to\infty}\frac{a_1 + a_2 + \cdots + a_{2n}}{2n} \\
            ={} & \frac12\lim_{n\to\infty}\frac{(a_1 + a_3 + \cdots + a_{2n-1}) + (a_2 + a_4 + \cdots + a_{2n})}{n} \\
            ={} & \frac12\lim_{n\to\infty}\left(\frac{a_1 + a_3 + \cdots + a_{2n-1}}{n} + \frac{a_2 + a_4 + \cdots + a_{2n}}{n}\right) \\
            ={} & \frac12\left(\lim_{n\to\infty}\frac{a_1 + a_3 + \cdots + a_{2n-1}}{n} + \lim_{n\to\infty}\frac{a_2 + a_4 + \cdots + a_{2n}}{n}\right).
        \end{align*}
        已知 $\lim\limits_{n\to\infty}a_{2n-1} = a$, $\lim\limits_{n\to\infty}a_{2n} = b$, 由例 4 可得
        \begin{gather*}
            \lim_{n\to\infty}\frac{a_1 + a_3 + \cdots + a_{2n-1}}{n} = a, \\
            \lim_{n\to\infty}\frac{a_2 + a_4 + \cdots + a_{2n}}{n} = b.
        \end{gather*}
        即
        \[
            \lim_{n\to\infty}\frac{a_1 + a_2 + \cdots + a_{n}}{n} = \frac{a+b}{2}.   
        \]
    \item % 12
        {\heiti 证明}\quad 令 $a_n = \sum\limits_{i=1}^n( \sqrt{1 + i/n^2} - 1)$, 对 $a_n$ 进行化简, 则有
        \begin{align*}
            a_n &= \sum_{i=1}^n\left( \sqrt{1 + \frac{i}{n^2}} - 1 \right) \\
                &= \sqrt{1 + \frac{1}{n^2}} - 1 + \sqrt{1 + \frac{2}{n^2}} - 1 + \cdots + \sqrt{1 + \frac{1}{n}} - 1 \\
                &= \frac{\frac{1}{n^2}}{\sqrt{1 + \frac{1}{n^2}} + 1} + \frac{\frac{2}{n^2}}{\sqrt{1 + \frac{2}{n^2}} + 1} + \cdots + \frac{\frac{1}{n}}{\sqrt{1 + \frac{1}{n}} + 1}.
        \end{align*}
        显然有
        \begin{equation*}
            \frac{\frac{1}{n^2} \cdot \frac{n(n+1)}{2}}{\sqrt{1 + \frac{1}{n}} + 1} \leqslant a_n \leqslant \frac{\frac{1}{n^2} \cdot \frac{n(n+1)}{2}}{\sqrt{1 + \frac{1}{n^2}} + 1}.
        \end{equation*}
        因此 $\lim\limits_{n\to\infty}a_n = \dfrac 14$.
\end{enumerate}

% \end{document}

        \section{数列极限概念的推广}
            % \documentclass[a4paper, 12pt]{article}

% \usepackage[top=2cm, bottom = 2cm, left = 2.5cm, right = 2.5cm]{geometry}

% \usepackage{ctex}
% \usepackage{amsmath, amsfonts, amssymb}
% \usepackage{enumerate}

% \begin{document}

\begin{center}
    {\heiti 练习题 1.4}
\end{center}

\begin{enumerate}
    \item 1
    \item % 2
        {\heiti 证明}\quad 由题意有
        \[
            a_n = \frac{1 + 2 + \cdots + n}{n} = \frac{\frac{n(n+1)}{2}}{n} = \frac{n + 1}{2}.    
        \]
        对 $\forall A > 0$, 取 $N = [2A - 1]$, 当 $n > N$ 时, 有 $a_n > A$.即
        \[
            \lim\limits_{n\to\infty}\dfrac{1 + 2 + \cdots + n}{n} = +\infty.
        \]
    \item % 3
        {\heiti 证明}\quad 由题意有
        \begin{align*}
            a_n = \frac{1^2 + 2^2 + \cdots + n^2}{n^2}
            = \frac{\frac{n(n+1)(2n+1)}{6}}{n^2} > \frac{\frac{n\cdot n\cdot 2n}{6}}{n^2}
            = \frac{n}{3}.
        \end{align*}
        对 $\forall A > 0$, 取 $N = [3A]$, 当 $n > N$ 时, 有 $a_n > A$. 即
        \[
            \lim\limits_{n\to\infty}\frac{1^2 + 2^2 + \cdots + n^2}{n^2} = +\infty.
        \]
    \item 4
    \item % 5
        {\heiti 证明}\quad 由题意有
        \begin{align*}
            a_n = \frac{1}{\sqrt{n+1}} + \frac{1}{\sqrt{n+2}} + \cdots + \frac{1}{\sqrt{n+n}} \geqslant \frac{n}{\sqrt{n+n}} = \sqrt{\frac{n}{2}}.
        \end{align*}
        对 $\forall A > 0$, 取 $N = [2A^2]$, 当 $n > N$ 时, 有 $a_n > A$. 即
        \[
            \lim\limits_{n\to\infty}\left(\frac{1}{\sqrt{n+1}} + \frac{1}{\sqrt{n+2}} + \cdots + \frac{1}{\sqrt{n+n}}\right) = +\infty.
        \]
\end{enumerate}

% \end{document}

        \section{单调数列}
            % %@author 张书宁

% \documentclass[12pt]{article}

% \usepackage[top=2cm, bottom=2cm, left=2.5cm, right=2.5cm]{geometry}

% \usepackage{ctex}
% \usepackage{amsmath, amssymb}
% \usepackage{enumerate}

% \begin{document}

% \pagestyle{empty}

\begin{center}
    {\heiti 练习题 1.5}
\end{center}

\begin{enumerate}
    \item 
        \begin{enumerate}[(1)]
            \item {\heiti 证明}\quad 显然数列 $\{x_n\}$ 有下界 $0$. 现证明 $\{x_n\}$ 是递减的. 考察 $x_n / x_{n+1}$, 即
                \[
                    \frac{x_n}{x_{n+1}} = \frac{\frac{10}{1}\cdot\frac{11}{3}\cdots\frac{n+9}{2n-1}}{\frac{10}{1}\cdot\frac{11}{3}\cdots\frac{n+9}{2n-1}\cdot\frac{n+10}{2n+1}} = \frac{2n+1}{n+10}.    
                \]
                当 $n \geqslant 9$ 时, 有 $(2n+1)/(n+10) \geqslant 1$, 即 $x_n \geqslant x_{n+1}$. 由定理 1.5.1 可知 $\{x_n\}$ 的极限存在.
            \item {\heiti 证明}\quad 先对通项 $x_n$ 进行化简,
                \begin{align*}
                    x_n &= \left(1 - \frac12\right)\left(1 - \frac13\right)\cdots\left(1 - \frac{1}{n+1}\right) \\
                        &= \frac12 \times \frac23 \times \cdots \times \frac{n}{n+1} \\
                        &= \frac{n!}{(n+1)!} \\
                        &= \frac{1}{n+1}.    
                \end{align*}
                显然 $x_n > 0$, 考察 $x_n/x_{n+1}$, 则有
                \[
                    \left. \frac{x_n}{x_{n+1}} = \frac{1}{n+1} \right/ \frac{1}{n+2} = \frac{n+2}{n+1} > 1.    
                \]
                即 $x_n > x_{n+1}$. 由定理 1.5.1 可知 $\{x_n\}$ 的极限存在.
        \end{enumerate}
    \item {\heiti 证明}\quad 由题意显然知道 $\{x_n\}$ 是一个递增数列. 现在来证明 $\{x_n\}$ 是有上界的.
        已知 $x_1 = \sqrt{2} < 2$. 现假设 $x_{n - 1} < 2$, 那么
        \[
            x_n = \sqrt{2 + \sqrt{x_{n-1}}} < \sqrt{2 + 2} = 2.    
        \]
        由数学归纳法可知 $x_n < 2$ 成立, 即 2 是数列 $\{x_n\}$ 的一个上界. 由定理 1.5.1 可知 $\{x_n\}$ 的极限存在.
    \item {\heiti 证明}\quad 设递增数列 $\{a_n\}$ 有一子列 $\{a_{m_n}\}$ 收敛.
        我们有 $\forall \varepsilon > 0$, $\exists N \in \mathrm{N}^*$, 当 $m_n \geqslant n > N$ 时, 有
        \[
            |a_{m_n} - a| < \varepsilon.    
        \]
        又因数列 $\{a_n\}$ 是递增的, 则有 $a_n \leqslant a_{m_n}$, 因此
        \[
            |a_n - a| \leqslant |a_{m_n} - a| < \varepsilon.    
        \]
        这样就证明了数列 $\{a_n\}$ 也是收敛的.

        若数列 $\{a_n\}$ 是递减的, 那么 $\{-a_n\}$ 就是递增的, 同理可证 $\{-a_n\}$ 存在极限 $a$,
        即 $-a$ 就是 $\{a_n\}$ 的极限.
    \item {\heiti 证明}\quad 考察 $a_n(1 - a_n)$, 则有
        \begin{align*}
            a_n(1 - a_n) &= a_n - a_n^2 + \frac14 - \frac14 \\
                         &= \frac14 - \left(a_n^2 - a_n + \frac14\right) \\
                         &= \frac14 - \left(a_n - \frac12\right)^2 \leqslant \frac14. 
        \end{align*}
        再结合已知条件, 则有 $a_n(1 - a_n) \leqslant 1/4 < a_{n+1}(1 - a_n)$, 即 $a_n < a_{n+1}$.
        又因 $0 < a_n < 1$, 所以数列 $\{a_n\}$ 的极限存在.

        现设 $\{a_n\}$ 的极限为 $a$. 已知不等式
        \[
            a_n(1 - a_n) \leqslant \frac14 < a_{n+1}(1 - a_n)    
        \]
        成立, 由夹逼原理可以得到 $a(1 - a) = 1/4$, 解这个一元二次方程, 即可得到 $a = 1/2$.

    \item {\heiti 证明}\quad 根据均值不等式
        \[
            \frac{n}{1 + \frac12 + \cdots + \frac1n} \leqslant \sqrt[n]{n!} = a_n,    
        \]
        则有
        \begin{align*}
            a_{n+1} - a_n &\geqslant \frac{n+1}{1 + \frac12 + \cdots + \frac{1}{n+1}} - \frac{n}{1 + \frac12 + \cdots + \frac1n} \\
                          &= \frac{(n+1)\left(1 + \frac12 + \cdots \frac1n\right) - n\left(1 + \frac12 + \cdots + \frac{1}{n+1}\right)}{\left(1 + \cdots + \frac1n\right)\left(1 + \cdots + \frac{1}{n+1}\right)} \\
                          &= \frac{\left(n + \frac n2 + \cdots + 1\right) + \left(1 + \frac12 + \cdots + \frac1n\right) - \left(n + \frac n2 + \cdots + 1\right) - \frac{n}{n+1}}{\left(1 + \cdots + \frac1n\right)\left(1 + \cdots + \frac{1}{n+1}\right)} \\
                          &= \frac{\left(1 + \frac12 + \cdots + \frac1n\right) - \frac{n}{n+1}}{\left(1 + \cdots + \frac1n\right)\left(1 + \cdots + \frac{1}{n+1}\right)} \\
                          &= \frac{\left(1 - \frac{1}{n+1}\right) + \left(\frac12 - \frac{1}{n+1}\right) + \cdots + \left(\frac1n - \frac{1}{n+1}\right)}{\left(1 + \cdots + \frac1n\right)\left(1 + \cdots + \frac{1}{n+1}\right)} > 0. \\
        \end{align*}
        因此 $\{(n!)^{1/n}\}$ 是递增数列.
    \item 6
\end{enumerate}
% \end{document}

        \section{自然对数的底 $\mathrm{e}$}
            % \documentclass[12pt, a4paper]{article}

% \usepackage[top=2cm, bottom=2cm, left=2.5cm, right=2.5cm]{geometry}

% \usepackage{ctex}
% \usepackage{amsmath, amssymb}
% \usepackage{enumerate}

% \begin{document}

% \pagestyle{empty}

\begin{center}
    {\heiti 练习题 1.6}
\end{center}

\begin{enumerate}
    \item 
        \begin{enumerate}[(1)]
            \item $\lim\limits_{n\to\infty}\left(1 + \dfrac{1}{n-2}\right)^n = \lim\limits_{n\to\infty}\left(1 + \dfrac{1}{n-2}\right)^{n-2}\left(1 + \dfrac{1}{n-2}\right)^2 = \mathrm{e}$;
            \item $\lim\limits_{n\to\infty}\left(1 - \dfrac{1}{n+3}\right)^n = \lim\limits_{n\to\infty}\left(\dfrac{n+2}{n+3}\right)^n = \lim\limits_{n\to\infty}\dfrac{1}{\left(1 + \frac{1}{n+2}\right)^{n+2}\left(1 + \frac{1}{n+2}\right)^{-2}} = \dfrac{1}{\mathrm{e}}$;
            \item $\lim\limits_{n\to\infty}\left(\dfrac{1+n}{2+n}\right)^n = \lim\limits_{n\to\infty}\dfrac{1}{\left(1 + \frac{1}{n+1}\right)^{n+1} \left(1 + \frac{1}{n+1}\right)^{-1}} = \dfrac{1}{\mathrm{e}}$;
            \item $\lim\limits_{n\to\infty}\left(1 + \dfrac 3n\right)^n = \lim\limits_{n\to\infty}\left(\dfrac{n+3}{n+2}\right)^n\left(\dfrac{n+2}{n+1}\right)^n\left(\dfrac{n+1}{n}\right)^n = \mathrm{e}^3$;
            \item $\lim\limits_{n\to\infty}\left(1 + \dfrac{1}{2n^2}\right)^{4n^2} = \lim\limits_{n\to\infty}\left(1 + \dfrac{1}{2n^2}\right)^{2n^2} \left(1 + \dfrac{1}{2n^2}\right)^{2n^2} = \mathrm{e}^2$.
        \end{enumerate}
    \item {\heiti 证明}\quad 根据题意,则有
        \begin{align*}
                & \lim_{n\to\infty}\left(1 + \frac kn\right)^n \\
            ={} & \lim_{n\to\infty}\left(\frac{n+k}{n}\right)^n \\
            ={} & \lim_{n\to\infty}\underbrace{\left[\frac{n+k}{n+(k-1)}\right]^n\left[\frac{n+(n-1)}{n+(n-2)}\right]^n\cdots\left[\frac{n+1}{n}\right]^n}_{\text{$k$ 项}} \\
            ={} & \lim_{n\to\infty}\underbrace{\frac{\left[1 + \frac{1}{n+(k-1)}\right]^{n+(k-1)}}{\left[1 + \frac{1}{n+(k-1)}\right]^{k-1}} \times \frac{\left[1 + \frac{1}{n+(k-2)}\right]^{n+(k-2)}}{\left[1 + \frac{1}{n+(k-2)}\right]^{k-2}} \times \cdots \times \left[1 + \frac 1n\right]^n}_{\text{$k$ 项}} \\
            ={} & \mathrm{e}^k.
        \end{align*}
        即 $\lim\limits_{n\to\infty}\left(1 + k/n\right)^n = \mathrm{e}^k$.
    \item 略.
    \item 略.
    \item {\heiti 证明}\quad $\because \left\{\left(1 + 1/n\right)^n\right\}$ 是严格递增的数列, $\left\{\left(1 + 1/n\right)^{n+1}\right\}$ 是严格递减的数列,
        而它们的极限都是 $\mathrm{e}$.
        
        $\therefore \left(1 + 1/n\right)^n < \mathrm{e} < \left(1 + 1/n\right)^{n+1}$.
    \item {\heiti 证明}\quad 先证明左边的不等式. 在不等式的两边同乘 $n + 1$, 则有
        \begin{equation*}
            \ln\mathrm{e} = 1 < \ln\left(1 + \frac 1n\right)^{n+1}.    
        \end{equation*}
        由对数函数 $\ln x$ 的严格递增性可知
        \begin{equation*}
            \mathrm{e} < \left(1 + \frac 1n\right)^{n+1}.
        \end{equation*}
        由第 5 题的结论可知上式成立.
        
        采用同样的方法亦可证明右边的不等式.
    \item {\heiti 证明}\quad 先证明右边的不等式. 在不等式的两边同乘 $n$, 则有
        \begin{equation*}
            \ln\left(1 + \frac kn\right)^n < k = k \cdot 1 = k \cdot \ln\mathrm{e} = \ln\mathrm{e}^k.
        \end{equation*}
        由对数函数 $\ln x$ 的严格递增性可知
        \begin{equation*}
            \left(1 + \frac kn\right)^n < \mathrm{e}^k.
        \end{equation*}
        根据第 2 题的结论已经知道 $\lim\limits_{n\to\infty}\left(1 + k/n\right)^n = \mathrm{e}^k$, 再采用第 3 题的证明方法即可证明 $\left\{\left(1 + k/n\right)^n\right\}$ 是严格递增的数列,
        上式得证.
        
        采用同样的证明方法亦可证明左边的不等式.
    \item {\heiti 证明}\quad 先看右边的不等式, 根据第 6 题的结论有
        \[
            \ln\left(1 + \frac 1n\right) < \frac 1n.
        \]
        对上式进行化简, 则有
        \[
            \ln(n+1)-\ln n < \frac 1n.    
        \]
        即
        \begin{equation*}
            \ln2 - \ln1 < 1,
            \ln3 - \ln2 < \frac 12,
            \cdots,
            \ln(n+1) - \ln n < \frac 1n.
        \end{equation*}
        将上列不等式相加, 即可得到
        \[
            \ln(n + 1) < 1 + \frac 12 + \cdots + \frac 1n.    
        \]

        采用同样的证明方法亦可证明左边的不等式.
    \item {\heiti 证明}\quad 先证明 $\{x_n\}$ 是有上界的, 根据第 8 题的结论有
        \begin{align*}
            x_n &< \left(1 + \frac 12 + \cdots \frac 1n\right) - \left(\frac 12 + \frac13 + \cdots + \frac{1}{n+1}\right) \\
            &= 1 - \frac{1}{n+1} \\
            &= \frac{n}{n+1} < 1.
        \end{align*}
        
        再来证明 $\{x_n\}$ 是严格递增的数列. 考察 $x_{n+1} - x_n$, 即
        \begin{align*}
            x_{n+1} - x_n &= \left(1 + \frac12 + \cdots + \frac{1}{n+1} - \ln(n+2)\right) - \left(1 + \frac12 + \cdots + \frac1n - \ln(n+1)\right) \\
            &= \frac{1}{n+1} - \left(\ln(n+2) - \ln(n+1)\right). \\
        \end{align*}
        由第 8 题的证明过程已知 
        \[
            \frac{1}{n+1} > \ln(n+2) - \ln(n+1).
        \]
        因此 $x_{n+1} > x_n$, 所以 $\{x_n\}$ 是严格递增的.
        
        综上所述, 由定理 1.5.1 可知数列 $\{x_n\}$ 的极限存在.
    \item {\heiti 证明}\quad 在等式的两边同时减去 $\ln(n+1)$ 得到
        \begin{equation*}
            1 + \frac12 + \cdots + \frac1n - \ln(n+1) = \ln n - \ln(n+1) + \gamma + \varepsilon_n.
        \end{equation*}
        记等号的左边为 $x_n$, 则有
        \begin{align*}
            x_n &= \ln\frac{n}{n+1} + \gamma + \varepsilon_n \\
            &= -\ln\left(1+\frac1n\right) + \gamma + \varepsilon_n.
        \end{align*}
        现在来证明 $\lim\limits_{n\to\infty}(-\ln(1 + 1/n)) = 0$. 根据第 6 题的结论,即不等式
        \[
            \frac{1}{n+1} < \ln\left(1 + \frac1n\right) < \frac1n    
        \]
        对一切 $n \in \mathrm{N}^*$ 成立. 由夹逼原理可得 $\lim\limits_{n\to\infty}\ln(1 + 1/n) = 0$, 即
        \[
            \lim\limits_{n\to\infty}\left(-\ln\left(1 + \frac1n\right)\right) = 0.  
        \]
        由第 9 题的结论可知
        \[
            \gamma = \lim\limits_{n\to\infty}x_n = \lim\limits_{n\to\infty}\left(-\ln\left(1 + \frac1n\right) + \gamma + \varepsilon_n\right) = \gamma.
        \]
        这就证明了
        \[
            1 + \frac12 + \cdots + \frac1n = \ln n + \gamma + \varepsilon_n.  
        \]
        并且其中的 $\varepsilon_n = \ln(1 + 1/n)$.
    \item {\heiti 证明}\quad 先证左边的不等式. 由第 5 题的结论可知
        \[
            \left(\frac{n+1}{n}\right)^n < \mathrm{e},    
        \]
        即
        \begin{equation*}
            \left(\frac21\right)^1 < \mathrm{e},
            \left(\frac32\right)^2 < \mathrm{e},
            \cdots,
            \left(\frac{n+1}{n}\right)^n < \mathrm{e}.
        \end{equation*}
        将上列不等式相乘, 则有
        \[
            \frac{(n+1)^n}{n!} = \frac21\times\frac{3^2}{2^2}\times\cdots\times\frac{(n+1)^n}{n^n} < \mathrm{e}^n.    
        \]
        通过移项即可得到 $(n+1)^n/\mathrm{e}^n < n!$, 这样就证明了左边的不等式.

        同理可证右边的不等式.
    \item {\heiti 证明}\quad 利用第 11 题的结论, 即对不等式
        \[
            \frac{1}{\mathrm{e}}\left(1 + \frac1n\right) < \frac{\sqrt[n]{n!}}{n} < \frac{1}{\mathrm{e}}\left(1 + \frac1n\right)\sqrt[n]{n+1}  
        \]
        的两边取极限, 其中 $\lim\limits_{n\to\infty}\sqrt[n]{n+1} = 1$. 由夹逼原理即可得到
        \[
            \lim\limits_{n\to\infty}\frac{\sqrt[n]{n!}}{n} = \frac{1}{\mathrm{e}}.    
        \]
    \item 13.
    \item 14.
    \item 15.
    \item {\heiti 证明}\quad.
\end{enumerate}
% \end{document}

        \section{基本列和 Cauchy 收敛原理}
            % %@author 张书宁
% %@date 2019-12-01

% \documentclass[12pt, a4paper]{article}

% \usepackage[top=2cm, bottom=2cm, left=2.5cm, right=2.5cm]{geometry}

% \usepackage{ctex}
% \usepackage{amsmath, amssymb}
% \usepackage{enumerate}

% \begin{document}

% \pagestyle{empty}

\begin{center}
    {\heiti 练习题 1.7}
\end{center}

\begin{enumerate}
    \item 1
    \item 2
    \item 
        \begin{enumerate}[(1)]
            \item {\heiti 证明}\quad 对 $\forall p \in \mathrm{N}^*$, 有
                \begin{align*}
                    |a_{n+p} - a_n| &= \left|(-1)^n\frac{1}{(n+1)^2} + (-1)^{n+1}\frac{1}{(n+2)^2} + \cdots + (-1)^{n+p-1}\frac{1}{(n+p)^2}\right| \\
                                    &\leqslant \frac{1}{(n+1)^2} + \frac{1}{(n+2)^2} + \cdots + \frac{1}{(n+p)^2} \\
                                    &< \frac1n \cdot \frac{1}{n+1} + \frac{1}{n+1} \cdot \frac{1}{n+2} + \cdots + \frac{1}{n+p-1} \cdot \frac{1}{n+p} \\
                                    &= \frac1n - \frac{1}{n+1} + \frac{1}{n+1} - \frac{1}{n+2} + \cdots + \frac{1}{n+p-1} - \frac{1}{n+p} \\
                                    &= \frac1n - \frac{1}{n+p} < \frac1n.
                \end{align*}
                因此只需 $n > N = [1/\varepsilon]$, 即可得出 $\{a_n\}$ 是基本列.
            \item {\heiti 证明}\quad 当 $q = 0$ 时, $\{b_n\}$ 显然收敛到 $a_0$, 所以 $\{b_n\}$ 是基本列. 现设 $0 < |q| < 1$, 对 $\forall p \in \mathrm{N}^*$, 有
                \begin{align*}
                    |b_{n+p} - b_n| &= |a_{n+1}q^{n+1} + a_{n+2}q^{n+2} + \cdots + a_{n+p}q^{n+p}| \\
                                    &= |q|^{n+1} \cdot |a_{n+1} + a_{n+2}q + \cdots + a_{n+p}q^{p-1}| \\
                                    &\leqslant |q|^{n+1}(|a_{n+1}| + |a_{n+2}||q| + \cdots + |a_{n+p}||q|^{p-1}) \\
                                    &\leqslant |q|^{n+1}M (1 + |q| + \cdots + |q|^{p-1}) \\
                                    &=|q|^{n+1}M\frac{1-|q|^p}{1-|q|} \\
                                    &< \frac{|q|^nM}{1-|q|},
                \end{align*}
                其中 $|a_n| \leqslant M\ (n \in \mathrm{N}^*)$. 当 $n$ 满足
                \[
                    n > N = \left[\frac{\ln\frac{(1-|q|)\varepsilon}{M}}{\ln|q|}\right]    
                \]
                的条件时, 即可得出 $\{b_n\}$ 是基本列.
            \item {\heiti 证明}\quad 对 $\forall p \in \mathrm{N}^*$, 有
                \begin{align*}
                    |a_{n+p} - a_n| &= \left| \frac{\sin(n+1)x}{(n+1)^2} + \frac{\sin(n+2)x}{(n+2)^2} + \cdots + \frac{\sin(n+p)x}{(n+p)^2} \right| \\
                                    &\leqslant \frac{|\sin(n+1)x|}{(n+1)^2} + \frac{|\sin(n+2)x|}{(n+2)^2} + \cdots + \frac{|\sin(n+p)x|}{(n+p)^2} \\
                                    &\leqslant \frac{1}{(n+1)^2} + \frac{1}{(n+2)^2} + \cdots + \frac{1}{(n+p)^2} \\
                                    &< \frac1n \cdot \frac{1}{n+1} + \frac{1}{n+1} \cdot \frac{1}{n+2} + \cdots + \frac{1}{n+p-1} \cdot \frac{1}{n+p} \\
                                    &= \frac1n - \frac{1}{n+1} + \frac{1}{n+1} - \frac{1}{n+2} + \cdots + \frac{1}{n+p-1} - \frac{1}{n+p} \\
                                    &= \frac1n - \frac{1}{n+p} < \frac1n.
                \end{align*}
                因此只需 $n > N = [1/\varepsilon]$, 即可得出 $\{a_n\}$ 是基本列.
            \item {\heiti 证明}\quad 对 $\forall p \in \mathrm{N}^*$, 有
                \begin{align*}
                    |a_{n+p} - a_n| &= \left| \frac{\sin(n+1)x}{(n+1)((n+1) + \sin(n+1)x)} + \cdots + \frac{\sin(n+p)x}{(n+p)((n+p) + \sin(n+p)x)} \right| \\
                                    &= \left| \frac{1}{n+1} - \frac{1}{(n+1) + \sin(n+1)x} + \cdots + \frac{1}{n+p} + \frac{1}{(n+p) + \sin(n+p)x} \right| \\
                                    &\leqslant \left| \frac{1}{n+1} - \frac{1}{n+2} + \cdots + \frac{1}{n+p} - \frac{1}{n+p+1} \right| \\
                                    &= \left| \frac{1}{n+1} - \frac{1}{n+p+1} \right| \\
                                    &< \frac{1}{n+1}.
                \end{align*}
                因此只需 $n > N = [1/\varepsilon - 1]$, 即可得出 $\{a_n\}$ 是基本列.
        \end{enumerate}
    \item {\heiti 证明}\quad 考察 $a_{n+1} - a_n$, 那么有
        \begin{align*}
            a_{n+1} - a_n &= \left(|a_2-a_1| + \cdots + |a_{n+1}-a_n|\right) - \left(|a_2-a_1| + \cdots + |a_n-a_{n-1}|\right) \\
                          &= |a_{n+1}-a_n| \geqslant 0,
        \end{align*}
        所以 $\{a_n\}$ 是单调递增的. 又因 $\{a_n\}$ 是有界的, 由定理 1.5.1 可得到 $\{a_n\}$ 收敛.
    \item $\exists M > 0$, 对 $\forall N \in \mathrm{N}^*$, 当 $m, n > N$ 时, 有
        \[
            |a_m - a_n| \geqslant M.   
        \]
    \item 6
\end{enumerate}
% \end{document}

        \section{上确界和下确界}
            % %@author 张书宁
% %@date 2019-12-01

% \documentclass[12pt]{article}

% \usepackage[top=2cm, bottom=2cm, left=2.5cm, right=2.5cm]{geometry}

% \usepackage{ctex}
% \usepackage{amsmath, amssymb}
% \usepackage{enumerate}

% \begin{document}

% \pagestyle{empty}

\begin{center}
    {\heiti 练习题 1.8}
\end{center}

\begin{enumerate}
    \item 
        \begin{enumerate}[(1)]
            \item 令 $E = \{-1, 0, 3, 8, 9, 12\}$, 则有 $\sup E = 12$, $\inf E = -1$.
            \item 令 $E = \{1/n: n \in \mathrm{N}^*\}$, 则有 $\sup E = 1$, $\inf E = 0$.
            \item 令 $E = \{\sqrt{n}: n \in \mathrm{N}^*\}$, 则有 $\sup E = +\infty$, $\inf E = 1$.
            \item 令 $E = \{\sin\dfrac{\pi}{n}: n \in \mathrm{N}^*\}$, 则有 $\sup E = 1$, $\inf E = 0$.
            \item 令 $E = \{x: x^2 - 2x - 3\}$, 则有 $\sup E = 3$, $\inf E = -1$.
            \item 令 $E = \{x: |\ln x| < 1\}$, 则有 $\sup E = \mathrm{e}$, $\inf E = 1/\mathrm{e}$.
        \end{enumerate}
    \item 2
    \item 令 $E = \{ n^{1/n}: n\in\mathrm{N}^* \}$, 可得 $\sup E = \sqrt[3]{3}$, $\inf E = 1$.
    \item 4
    \item 5
\end{enumerate}
% \end{document}

        \section{有限覆盖定理}
        \section{上极限和下极限}
        \section{Stolz 定理}
            % %@author 张书宁
% %@date 2019-12-02

% \documentclass[12pt]{article}

% \usepackage[top=2cm, bottom=2cm, left=2.5cm, right=2.5cm]{geometry}

% \usepackage{ctex}
% \usepackage{amsmath, amssymb}
% \usepackage{enumerate}

% \begin{document}

% \pagestyle{empty}

\begin{center}
    {\heiti 练习题 1.11}
\end{center}

\begin{enumerate}
    \item % 1
        \begin{enumerate}[(1)]
            \item % 1.1
                1;
            \item % 1.2
                1;
            \item % 1.3
                2;
            \item % 1.4
                $2/3$.
        \end{enumerate}
    \item % 2
        利用等式
        \[
            (n!)^{\frac{1}{n^2}} = \mathrm{e}^{\ln(n!)/n^2}.    
        \]
        令 $a_n = \ln n!$, $b_n = n^2$, 则有
        \[
            \lim_{n\to\infty}\frac{a_{n+1} - a_n}{b_{n+1} - b_n} = \lim_{n\to\infty}\frac{\ln(n+1)! - \ln n!}{(n+1)^2 - n^2} = \lim_{n\to\infty}\frac{\ln(n+1)}{2n+1}.
        \]
        再令 $c_n = \ln(n+1)$, $d_n = 2n+1$, 那么
        \[
            \lim_{n\to\infty}\frac{c_{n+1} - c_n}{b_{n+1} - d_n} = \lim_{n\to\infty}\frac{\ln(n+2) - \ln(n+1)}{2n+3 - (2n + 1)} = \frac12\lim_{n\to\infty}\ln\left(1 + \frac{1}{n+1}\right) = 0.    
        \]
        因此 $\lim\limits_{n\to\infty}(\ln n!)/n^2 = 0$, 即 $\lim\limits_{n\to\infty}(n!)^{1/n^2} = 1$.
    \item % 3
        $4/3$.
    \item % 4
        {\heiti 证明}\quad 令 $x_n = a_1 + 2a_2 + \cdots + na_n$, $y_n = n^2$, 显然 $\{y_n\}$ 是严格递增且趋于 $+\infty$.
        则有
        \begin{align*}
            \lim_{n\to\infty}\frac{x_{n+1} - x_n}{y_{n+1} - y_n} &= \lim_{n\to\infty}\frac{(n+1)a_{n+1}}{(n+1)^2 - n^2} \\
                                                                 &= \lim_{n\to\infty}\frac{(n+1)a_{n+1}}{2n + 1} \\
                                                                 &= \lim_{n\to\infty}\frac{(n+1)a_{n+1}}{2(n+1) - 1} \\
                                                                 &= \lim_{n\to\infty}\frac{a_{n+1}}{2 - \frac{1}{n+1}} \\
                                                                 &= \frac a2.
        \end{align*}
        由 Stolz 定理可得
        \[
            \lim_{n\to\infty}\frac{x_n}{y_n} = \lim_{n\to\infty}\frac{a_1 + 2a_2 + \cdots + na_n}{n^2} = \frac a2.    
        \]
    \item % 5
        设 $a_n = (-1)^{n}$, $b_n = n$, 便有
        \[
            \lim_{n\to\infty}\frac{a_n}{b_n} = \lim_{n\to\infty}\frac{(-1)^n}{n} = 0.    
        \]
        那么
        \[
            \lim_{n\to\infty}\frac{a_{n+1} - a_n}{b_{n+1} - b_n} = \lim_{n\to\infty}\frac{(-1)^{n+1} - (-1)^n}{(n+1) - n} = 2\lim_{n\to\infty}(-1)^{n+1}.    
        \]
        但 $\{(-1)^{n+1}\}$ 的极限是不存在的. 
    \item 6
\end{enumerate}

% \end{document}

    \chapter{函数的连续性}
        \section{集合的映射}
        \section{集合的势}
        \section{函数}
        \section{函数的极限}
        \section{极限过程的其他形式}
        \section{无穷小与无穷大}
        \section{连续函数}
        \section{连续函数与极限计算}
        \section{函数的一致连续性}
        \section{有限闭区间上连续函数的性质}
        \section{函数的上极限和下极限}
        \section{混沌现象}
    \chapter{函数的导数}
        \section{导数的定义}
        \section{导数的计算}
        \section{高阶导数}
        \section{微分学的中值定理}
        \section{利用导数研究函数}
        \section{L'Hospital 法则}
        \section{函数作图}
    \chapter{一元微分学的顶峰——Taylor 定理}
        \section{函数的微分}
        \section{带 Peano 余项的 Taylor 定理}
        \section{带 Lagrange 余项和 Cauchy 余项的 Taylor 定理}
    \chapter{求导的逆运算}
        \section{原函数的概念}
        \section{分部积分法和换元法}
        \section{有理函数的原函数}
        \section{可有理化函数的原函数}
    \chapter{函数的积分}
        \section{积分的概念}
        \section{可积函数的性质}
        \section{微积分基本定理}
        \section{分部积分与换元}
        \section{可积分理论}
        \section{Lebesgue 定理}
        \section{反常积分}
        \section{数值积分}
    \chapter{积分学的应用}
        \section{积分学在几何学中的应用}
        \section{物理应用举例}
        \section{面积原理}
        \section{Wallis 公式和 Stirling 公式}
    \chapter{多变量函数的连续性}
        \section{$n$ 维 Euclid 空间}
        \section{$\mathrm{R}^n$ 中点列的极限}
        \section{$\mathrm{R}^n$ 中的开集和闭集}
        \section{列紧集和紧致集}
        \section{集和的连通性}
        \section{多变量函数的极限}
        \section{多变量连续函数}
        \section{连续映射}
    \chapter{多变量函数}
        \section{方向导数和偏导数}
        \section{多变量函数的微分}
        \section{映射的微分}
        \section{复合求导}
        \section{曲线的切线和曲面的切平面}
        \section{隐函数定理}
        \section{隐映射定理}
        \section{逆映射定理}
        \section{高阶偏导数}
        \section{中值定理和 Taylor 定理}
        \section{极值}
        \section{条件极值}
\end{document}
