% %@author Shuning Zhang
% %@date 2019-12-05

% \documentclass[12pt]{ctexart}

% \usepackage[top=2cm, bottom=2cm, left=2.5cm, right=2.5cm]{geometry}

% \usepackage{enumerate}
% \usepackage{amsmath, amsfonts, amssymb, amsthm}

% \begin{document}

\pagestyle{empty}

\begin{center}
    {\heiti 练习题 2.6}
\end{center}

\begin{enumerate}
    \item % 1
        \begin{enumerate}[(1)]
            \item % 1.1
                略.
            \item % 1.2
                略.
            \item % 1.3
                略.
            \item % 1.4
                略.
            \item % 1.5
                略.
            \item % 1.6
                略.
            \item % 1.7
                略.
            \item % 1.8
                $\lim\limits_{x\to0}\dfrac{\sqrt{x^2 + x^{1/3}}}{x^{1/6}} = \lim\limits_{x\to0}\sqrt{1 + x^{5/3}} = 1$.
            \item % 1.9
                $\lim\limits_{x\to\infty}\dfrac{\sqrt{x^2 + x^{1/3}}}{x} = \lim\limits_{x\to\infty}\sqrt{1 + x^{-5/3}} = 1$.
            \item % 1.10
                $\lim\limits_{x\to0}\dfrac{\sqrt{1+x} - \sqrt{1-x}}{x} = \lim\limits_{x\to0} \dfrac{2x}{x(\sqrt{1+x} + \sqrt{1-x})} = 1$.
            \item % 1.11
                $\lim\limits_{x\to0^+}\dfrac{\sqrt{1+\sqrt{1+\sqrt{x}}} - \sqrt{2}}{\sqrt{x}} = \lim\limits_{x\to0^+} \dfrac{\sqrt{1 + \sqrt{x}} - 1}{\sqrt{x}\left(\sqrt{1+\sqrt{1+\sqrt{x}}} + \sqrt{2}\right)}$ \\
                $= \lim\limits_{x\to0^+} \dfrac{1}{\left(\sqrt{1+\sqrt{1+\sqrt{x}}} + \sqrt{2}\right)\left(\sqrt{1 + \sqrt{x}} + 1\right)} = \dfrac{1}{4\sqrt{2}}$.
            \item % 1.12
                $\lim\limits_{x\to0}\dfrac{\sqrt{1+\tan x} - \sqrt{1-\sin x}}{x} = \lim\limits_{x\to0}\dfrac{\tan x + \sin x}{x(\sqrt{1+\tan x} + \sqrt{1-\sin x})}$ \\
                $= \lim\limits_{x\to0}\dfrac{1/\cos x + 1}{\sqrt{1+\tan x} + \sqrt{1-\sin x}} = 1$.
            \item % 1.13
                $\lim\limits_{x\to\infty}\dfrac{\sqrt{x + \sqrt{x + \sqrt{x}}}}{\sqrt{x}} = \lim\limits_{x\to\infty}\sqrt{1 + \sqrt{1/x + \sqrt{1/x^3}}} = 1$.
            \item % 1.14
                $\lim\limits_{x\to+\infty}\dfrac{(1+x)(1+x^2)\cdots(1+x^n)}{x^{\frac{n(n+1)}{2}}} = \lim\limits_{x\to+\infty} \dfrac{(1+x)(1+x^2) \cdots (1+x^n)}{x \cdot x^2 \cdots x^n}$ \\
                $= \lim\limits_{x\to+\infty} \left(1 + \dfrac{1}{x}\right)\left(1 + \dfrac{1}{x^2}\right) \cdots \left(1 + \dfrac{1}{x^n}\right) = 1$.
        \end{enumerate}
    \item % 2
        \begin{enumerate}[(1)]
            \item % 2.1
                略.
            \item % 2.2
                略.
            \item % 2.3
                略.
            \item % 2.4
                {\heiti 证明}\quad 根据题意, 有
                \[
                    \frac1\alpha\left(\frac{1}{1+\alpha} - (1-\alpha)\right) = o(1).    
                \]
                对上面等式的左边进行化简, 则有
                \[
                    \frac1\alpha\left(\frac{1}{1+\alpha} - (1-\alpha)\right) = \frac1\alpha \cdot \frac{\alpha^2}{1+\alpha} = \frac{\alpha}{1 + \alpha}.
                \]
                当 $x \to x_0$ 时, 显然有
                \[
                    \frac{\alpha}{1 + \alpha} = 0 = o(1).  
                \]
        \end{enumerate}
    \item % 3
        \begin{enumerate}[(1)]
            \item % 3.1
                $2$;
            \item % 3.2
                $1$;
            \item % 3.3
                $0$;
            \item % 3.4
                $1$;
            \item % 3.5
                $\dfrac{1}{2n}$.
        \end{enumerate}
\end{enumerate}

% \end{document}