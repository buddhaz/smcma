% %@author Shuning Zhang
% %@date 2019-12-05

% \documentclass[12pt]{ctexart}

% \usepackage[top=2cm, bottom=2cm, left=2.5cm, right=2.5cm]{geometry}

% \usepackage{enumerate}
% \usepackage{amsmath, amsfonts, amssymb, amsthm}

% \begin{document}

% \pagestyle{empty}

\begin{center}
    {\heiti 练习题 2.5}
\end{center}

\begin{enumerate}
    \item % 1
        略.
    \item % 2
        略.
    \item % 3
        {\heiti 证明}\quad 令 $y = \sin\sqrt{x + 1} - \sin\sqrt{x-1}$. 对 $y$ 进行化简, 则有
        \begin{align*}
            y &= \sin\sqrt{x + 1} - \sin\sqrt{x-1} \\
              &= 2\sin\frac{\sqrt{x+1} - \sqrt{x-1}}{2}\cos\frac{\sqrt{x+1} + \sqrt{x-1}}{2} \\
              &= 2\sin\frac{1}{\sqrt{x+1} + \sqrt{x-1}}\cos\frac{\sqrt{x+1} + \sqrt{x-1}}{2}.
        \end{align*}
        因为 $\lim\limits_{n\to+\infty}\sin\dfrac{1}{\sqrt{x+1} + \sqrt{x-1}} = 0$, 而 $-1 \leqslant \cos\dfrac{\sqrt{x+1} + \sqrt{x-1}}{2} \leqslant 1$, 所以
        \[
            \lim_{n\to+\infty} y = 0.    
        \]
    \item % 4
        {\heiti 证明}\quad 因为 $a_1 + a_2 + \cdots + a_n = 0$, 所以 $a_1 = -a_2 - a_3 - \cdots - a_n$, 那么
        \begin{align*}
            \lim_{n\to+\infty}\sum_{i=1}^n a_i\sin\sqrt{x+i} &= \lim_{n\to+\infty}\left(a_1\sin\sqrt{x+1} + \sum_{i=2}^n a_i\sin\sqrt{x+i}\right) \\
                                                             &= \lim_{n\to+\infty}\left(\sin\sqrt{x+1}\sum_{i=2}^n(-a_i) + \sum_{i=2}^n a_i\sin\sqrt{x+i}\right) \\
                                                             &= \lim_{n\to+\infty}\sum_{i=2}^n a_i(\sin\sqrt{x+i} - \sin\sqrt{x+1}) \\
                                                             &= \lim_{n\to+\infty}\sum_{i=2}^n 2a_i\sin\frac{\sqrt{x+i} - \sqrt{x+1}}{2}\cos\frac{\sqrt{x+i} + \sqrt{x+1}}{2} \\
                                                             &= \lim_{n\to+\infty}\sum_{i=2}^n 2a_i\sin\frac{i-1}{2(\sqrt{x+i} + \sqrt{x+1})}\cos\frac{\sqrt{x+i} + \sqrt{x+1}}{2}.
        \end{align*}
        利用第 3 题的结论, 可得
        \begin{align*}
                & \lim_{n\to+\infty}\sum_{i=2}^n 2a_i\sin\frac{i-1}{2(\sqrt{x+i} + \sqrt{x+1})}\cos\frac{\sqrt{x+i} + \sqrt{x+1}}{2} \\
            ={} & \underbrace{0 + 0 + \cdots + 0}_{\text{$n-1$ 个}} \\
            ={} & 0.
        \end{align*}
    \item % 5
        令 $f(n) = \sin(\pi\sqrt{n^2+1})$. 对 $f(n)$ 进行化简, 则有
        \begin{align*}
            f(n) &= \sin(\pi\sqrt{n^2+1}) \\
                 &= \sin(\pi\sqrt{n^2+1} - n\pi + n\pi) \\
                 &= \sin\left(\pi(\sqrt{n^2+1} - n) + n\pi\right) \\
                 &= \sin\left(\pi(\sqrt{n^2+1} - n)\right)\cos n\pi + \cos\left(\pi(\sqrt{n^2+1} - n)\right)\sin n\pi \\
                 &= (-1)^n \sin\left(\pi(\sqrt{n^2+1} - n)\right) \\
                 &= (-1)^n \sin\frac{\pi}{\sqrt{n^2+1} + n}.
        \end{align*}
        因此 $\lim\limits_{n\to\infty}f(n) = 0$.
    \item % 6
        \begin{enumerate}[(1)]
            \item % 6.1
                $1/e^2$;
            \item % 6.2
                $e^{2a}$;
            \item % 6.3
                $1/e^2$;
            \item % 6.4
                $e^{x+2}$.
        \end{enumerate}
    \item % 7
        根据题意, 有
        \begin{align*}
            f(x) &= \lim_{n\to\infty}n^x\left(\left(1+\frac1n\right)^{n+1} - \left(1+\frac1n\right)^n\right) \\
                 &= \lim_{n\to\infty} n^{x-1} \left(1+\frac1n\right)^n \\
                 &= \mathrm{e} \lim_{n\to\infty} n^{x-1}.
        \end{align*}
        因此
        \[
            f(x) =
                \begin{cases}
                    0, & x < 1; \\
                    \mathrm{e}, & x = 1. \\
                \end{cases}
        \]
        当 $x > 1$ 时, $f(x)$ 不存在.
    \item % 8
        {\heiti 证明}\quad 由题目已知条件 $\left|\sum\limits_{i=1}^n a_i\sin ix\right| \leqslant |\sin x|$, 有
        \[
            \frac{\left|\sum\limits_{i=1}^n a_i\sin ix\right|}{|\sin x|} \leqslant 1,   
        \]
        那么
        \[
            \left| \frac{a_1\sin x}{x}\cdot\frac{x}{\sin x} + \frac{a_2\sin2x}{2x}\cdot\frac{2x}{\sin x} + \cdots + \frac{a_n\sin x}{nx}\cdot\frac{nx}{\sin x} \right| \leqslant 1.
        \]
        对上面的不等式取极限, 令 $x \to 0$, 即可得到
        \[
            |a_1 + 2a_2 + \cdots + na_n| \leqslant 1.
        \]
    \item % 9
        {\heiti 证明}\quad 先证明必要性. 设 $\lim\limits_{x\to+\infty}f(x) = l$. 对 $\forall \varepsilon/2 > 0$, $\exists A_1 > 0$, 当 $x_1 > A_1$ 时, 有
        \[
            |f(x_1) - l| < \frac\varepsilon2.    
        \]
        同时, $\exists N_2 > 0$, 当 $x_2 > A_2$ 时, 有
        \[
            |f(x_2) - l| < \frac\varepsilon2.    
        \]
        现取 $A = \max(A_1, A_2)$, 当 $x_1 > A$, $x_2 > A$ 时, 有
        \begin{align*}
            |f(x_1) - f(x_2)| &= |f(x_1) - l + l - f(x_2)| \\
                              &\leqslant |f(x_1) - l| + |f(x_2) - l| \\
                              &< \frac\varepsilon2 + \frac\varepsilon2 = \varepsilon.
        \end{align*}
        再证明充分性. 考察任一发散到 $+\infty$ 的数列 $\{x_n\}$. 因为 $x_n\rightarrow+\infty\ (n\to\infty)$, 所以对 $\forall A > 0$, $\exists N \in \mathrm{N}^*$, 当 $n > N$ 时, 有
        \[
            x_n > A.    
        \]
        进一步, 对 $\forall \varepsilon > 0$, $\exists A > 0$, 对这个 $A$, $\exists N \in \mathrm{N}^*$, 当 $m, n > N$ 时, 有
        \[
            x_m > A, x_n > A,    
        \]
        那么 $|f(x_m) - f(x_n)| < \varepsilon$. 这表明数列 $\{f(x_n)\}$ 是一个基本列. 因此 $\{f(x_n)\}$ 的极限存在, 即 $\lim\limits_{x\to+\infty}f(x)$ 存在.
    \item % 10
        {\heiti 证明}\quad 设 $f(x)$ 的最小正周期为 $T$, 对 $\forall \varepsilon > 0$, $\exists x_0 > 0$, 当 $x \in [x_0, x_0 + T]$, 有
        \[
            |f(x)| < \varepsilon.    
        \]
        另一方面, 对 $\forall x' \in [x_0 + (n-1)T, x_0 + nT]\ (n \in \mathrm{Z})$, $\exists x \in [x_0, x_0 + T]$, 使得
        \[
            f(x) = f(x + (n-1)T) = f(x').    
        \]
        而 $\bigcup\limits_{n\in\mathrm{Z}}[x_0 + (n-1)T, x_0 + nT] = \mathrm{R}$. 这表明对 $\forall x \in \mathrm{R}$, 都有 $|f(x)| < \varepsilon$. 因此 $f = 0$.
\end{enumerate}

% \end{document}