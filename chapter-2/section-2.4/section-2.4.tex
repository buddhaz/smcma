%@author Shuning Zhang
%@date 2019-12-04

\documentclass[12pt]{ctexart}

\usepackage[top=2cm, bottom=2cm, left=2.5cm, right=2.5cm]{geometry}

\usepackage{enumerate}
\usepackage{amsmath, amsfonts, amssymb, amsthm}

\begin{document}

\pagestyle{empty}

\begin{center}
    {\heiti 练习题 2.4}
\end{center}

\begin{enumerate}
    \item % 1
        略.
    \item 2
    \item % 3
        \begin{enumerate}[(1)]
            \item % 3.1
                {\heiti 证明}\quad 对 $\forall \varepsilon > 0$, $\exists \delta > 0$, 当 $0 < |x - x_0| < \delta$ 时, 有
                \[
                    ||f(x)| - |A|| \leqslant |f(x) - A| < \varepsilon.    
                \]
                这便证明了 $\lim\limits_{x\to x_0}|f(x)| = |A|$.
            \item % 3.2
                {\heiti 证明}\quad 因为 $f(x)$ 的极限存在, 所以 $f(x)$ 是有界的, 即 $|f(x)| \leqslant M$.
                对 $\forall \varepsilon > 0$, 取 $\delta = \varepsilon/(M + |A|)$, 当 $0 < |x - x_0| < \delta$ 时, 有
                \[
                    |f^2(x) - A^2| = |f(x) + A||f(x) - A| \leqslant (|f(x)| + |A|)|f(x) - A| < \varepsilon.
                \]
            \item % 3.3
                略.
            \item % 3.4
                略.
        \end{enumerate}
    \item % 4
        \begin{enumerate}[(1)]
            \item % 4.1
                {\heiti 证明}\quad 对 $\forall \varepsilon > 0$, 取 $\delta = \min(1, \varepsilon/19)$, 当 $0 < |x - 2| < \delta$ 时, 有
                \[
                    |x^3 - 8| = |x - 2||x^2 + 2x + 2^2| < 19|x - 2| < \varepsilon.   
                \]
            \item % 4.2
                {\heiti 证明}\quad 对 $\forall \varepsilon > 0$, 取 $\delta = \min(1, 30\varepsilon)$, 当 $0 < |x - 3| < \delta$ 时, 有
                \[
                    \left|\frac{x-3}{x^2-9} - \frac16\right| = \left|\frac{1}{x+3} - \frac16\right| = \frac{|x-3|}{6|x+3|} < \frac{|x-3|}{30} < \varepsilon.
                \]
            \item % 4.3
                {\heiti 证明}\quad 对 $\forall \varepsilon > 0$, 取 $\delta = \min(1, \varepsilon/11)$, 当 $0 < |x - 1| < \delta$ 时, 有
                \begin{align*}
                    \left| \frac{x^4-1}{x-1} - 4 \right| &= |(x^2+1)(x+1) - 4| = |x^3 + x^2 + x - 3| \\
                                                         &= |x^3 - 1 + x^2 - 1 + x - 1| = |x - 1||x^2 + 2x + 3| < 11|x - 1| < \varepsilon.
                \end{align*}
            \item % 4.4
                略.
            \item % 4.5
                略.
        \end{enumerate}
    \item % 5
        \begin{enumerate}[(1)]
            \item % 5.1
                $f(2+) = 4$, $f(2-) = -2a$;
            \item % 5.2
                $a = -2$.
        \end{enumerate}
    \item % 6
        {\heiti 证明}\quad 令 $\lim\limits_{x\to x_0}f(x) = l > a$. 对 $\varepsilon_0 = l - a > 0$, $\exists \delta > 0$, 当 $0 < |x - x_0| < \delta$ 时, 有
        \[
            |f(x) - l| < \varepsilon_0,    
        \]
        即 $a = l - \varepsilon_0 < f(x)$.
    \item % 7
        {\heiti 证明}\quad 对 $\varepsilon_0 = \dfrac{f(x_0+) - f(x_0-)}{2} > 0$, $\exists \delta_1 > 0$, 当 $0 < x_0 - x < \delta_1$ 时, 有
        \[
            |f(x) - f(x_0-)| < \varepsilon_0,    
        \]
        即 $f(x) < f(x_0-) + \varepsilon_0$. 同时, $\exists \delta_2 > 0$, 当 $0 < y - x_0 < \delta_2$ 时, 有
        \[
            |f(y) - f(x_0+)| < \varepsilon_0,    
        \]
        即 $f(x_0+) - \varepsilon_0 < f(y)$. 现取 $\delta = \min(\delta_1, \delta_2)$, 当 $0 < x_0 - x < \delta$, $0 < y - x_0 < \delta$ 时, 有
        \[
            f(x) < f(x_0-) + \varepsilon_0 = f(x_0+) - \varepsilon_0 < f(y).
        \]
    \item 8
    \item % 9
        存在一个 $\varepsilon_0 > 0$, 对任意的 $\delta > 0$, 即使 $0 < |x - x_0| < \delta$, 依然有
        \[
            |f(x) - l| \geqslant \varepsilon_0.
        \]
    \item 10
    \item % 11
        \begin{enumerate}[(1)]
            \item % 11.1
                $-1$;
            \item % 11.2
                $0$;
            \item % 11.3
                $\lim\limits_{n\to1}\dfrac{x^m-1}{x-1} = \lim\limits_{n\to1}\dfrac{1-x^m}{1-x} = \lim\limits_{n\to1}(1 + x + \cdots + x^{m-1}) = m$;
            \item % 11.4
                $\lim\limits_{n\to1}\dfrac{x^m-1}{x^n-1} = \lim\limits_{n\to1}\dfrac{(1-x^m)/(1-x)}{(1-x^n)/(1-x)} = \dfrac mn$;
            \item % 11.5
                $1/2$;
            \item % 11.6
                $1$;
            \item % 11.7
                $1/m$;
            \item % 11.8
                $\dfrac{m(m+1)}{2}$.
        \end{enumerate}
    \item % 12
        \begin{enumerate}[(1)]
            \item % 12.1
                $a/b$;
            \item % 12.2
                $2$;
            \item % 12.3
                $1$;
            \item % 12.4
                $1$;
            \item % 12.5
                $\sin x$;
            \item % 12.6
                $\cos x$;
            \item % 12.7
                $\dfrac{n(n+1)(2n+1)}{12}$;
            \item % 12.8
                $\dfrac{\sin x}{x}$.
        \end{enumerate}
    \item % 13
        \begin{enumerate}[(1)]
            \item % 13.1
                $1$;
            \item % 13.2
                $0$;
            \item % 13.3
                $5/8$;
            \item % 13.4
                $3/2$.
        \end{enumerate}
    \item 14
    \item 15
    \item 16
\end{enumerate}

\end{document}