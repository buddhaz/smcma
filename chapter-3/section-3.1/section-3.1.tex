% author: Shuning Zhang
% creation date: 2019-12-15
\documentclass[a4paper, 11pt]{ctexart}
\usepackage[top=2cm, bottom=2cm, left=2.5cm, right=2.5cm]{geometry}
\usepackage{enumerate}
\usepackage{amsmath, amssymb}
\begin{document}
\pagestyle{empty}
\begin{enumerate}
    \item % 1
        $\lim\limits_{x\to0}\dfrac{f(x)}{x} = \lim\limits_{x\to0}\dfrac{f(x) - f(0)}{x - 0} = f'(0)$.
    \item 2
    \item 3
    \item % 4
        {\heiti 证明}\quad 已知 $f$ 在 $x_0$ 处可导, 即
        \[
            \lim_{h\to0}\frac{f(x_0 + h) - f(x_0)}{h} = f'(x_0),    
        \]
        那么
        \begin{align*}
            \lim_{h\to0}\frac{f(x_0+h) - f(x_0-h)}{2h} &= \lim_{h\to0}\frac{f(x_0+h) - f(x_0) + f(x_0) - f(x_0 - h)}{2h} \\
                                                         &= \frac12\lim_{h\to0}\left(\frac{f(x_0+h) - f(x_0)}{h} + \frac{f(x_0 - h) - f(x_0)}{-h}\right) \\
                                                         &= \frac{f'(x_0) + f'(x_0)}{2} = f'(x_0).
        \end{align*}

        考察 $f(x) = |x|$, 则有
        \[
            \lim_{h\to0}\frac{f(0+h) - f(0-h)}{2h} = \lim_{h\to0}\frac{|0+h| - |0-h|}{2h} = 0.
        \]
        但是 $f(x)$ 在 $x = 0$ 处不可导.
    \item 5
    \item % 6
        {\heiti 证明}\quad 先求抛物线 $y = x^2$ 在点 $(x_1, x_1^2)$ 处切线的斜率, 则有
        \[
            \lim_{h\to0}\frac{(x_1+h)^2 - x_1^2}{h} = \lim_{h\to0}\frac{x_1^2 + 2x_1h + h^2 - x_1^2}{h} = \lim_{h\to0}(2x_1 + h) = 2x_1.    
        \]
        同理, 可求出 $y$ 在点 $(x_2, x_2^2)$ 处切线的斜率为 $2x_2$. 在平面解析几何中, 两直线垂直, 其斜率的乘积为 $-1$, 则有
        \[
            2x_1\cdot2x_2 = -1,    
        \]
        即 $4x_1x_2 + 1 = 0$.
    \item % 7
        $
            \lim\limits_{n\to\infty}\left(\dfrac{f(a + \frac1n)}{f(a)}\right)^n
            = \lim\limits_{n\to\infty}\left(\dfrac{f(a + \frac1n) - f(a) + f(a)}{f(a)}\right)^n
            = \lim\limits_{n\to\infty}\left(1 + \dfrac{f(a + \frac1n) - f(a)}{f(a)}\right)^n \\
            = \lim\limits_{n\to\infty}\left(1 + \dfrac{f(a + \frac1n) - f(a)}{\frac1n} \cdot \dfrac{1}{nf(a)}\right)^n
            = \lim\limits_{n\to\infty}\left(1 + \dfrac{f'(a)}{nf(a)}\right)^n
            = \mathrm{e}^{\frac{f'(a)}{f(a)}}.
        $
    \item 8
    \item % 9
        $f'(0) = \lim\limits_{x\to0}\dfrac{f(x) - f(0)}{x - 0} = \lim\limits_{x\to0}\dfrac{x(x-1)^2(x-2)^3}{x} = -8$;

        $f'(1) = \lim\limits_{x\to1}\dfrac{f(x) - f(1)}{x - 1} = \lim\limits_{x\to0}\dfrac{x(x-1)^2(x-2)^3}{x-1} = 0$;

        $f'(2) = \lim\limits_{x\to2}\dfrac{f(x) - f(2)}{x - 2} = \lim\limits_{x\to1}\dfrac{x(x-1)^2(x-2)^3}{x - 2} = 0$.
    \item % 10
        {\heiti 证明} 考察极限
        \[
            \lim_{x\to0}\frac{f(x) - f(0)}{x - 0} = \lim_{x\to0}\frac{x^\lambda\sin\dfrac1x}{x} = \lim_{x\to0}x^{\lambda - 1}\sin\frac1x.    
        \]
        显然, 当 $\lambda > 1$ 时, 极限 $\lim\limits_{x\to0}x^{\lambda - 1}\sin\dfrac1x = 0$; 当 $\lambda \leqslant 1$ 时, 极限 $\lim\limits_{x\to0}x^{\lambda - 1}\sin\dfrac1x$ 不存在.
\end{enumerate}
\end{document}