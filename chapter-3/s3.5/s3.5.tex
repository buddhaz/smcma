% author: Shuning Zhang
% creation date: 2019-12-23
\documentclass[a4paper, 10pt]{ctexart}
\usepackage[top=2cm, bottom=2cm, left=2.5cm, right=2.5cm]{geometry}
\usepackage{enumerate}
\usepackage{amsmath, amssymb}
\begin{document}
\pagestyle{empty}
\begin{enumerate}
    \item % 1
        略.
    \item % 2
        略.
    \item % 3
    \item % 4
    \item % 5
        {\heiti 证明}\quad 令 $F(x) = f(x)/x$, 那么
        \[
            F'(x) = \frac{xf'(x) - f(x)}{x^2}.    
        \]
        再令 $g(x) = xf'(x) - f(x)$, 那么
        \[
            g'(x) = f'(x) + xf''(x) - f'(x) = xf''(x).    
        \]
        因为 $f'$ 严格递增, 所以 $f'' > 0$, 即 $g' > 0$.
    \item % 6
        {\heiti 证明}\quad 因为 $f'' \geqslant 0$, 所以 $f'$ 是 $\mathrm{R}$ 上的递增函数.
        因此, 存在一点 $x_0$, 使得当 $x \geqslant x_0$ 时, 有
        \[
            f'(x) \geqslant 0,
        \]
        即 $f$ 是 $[x_0, +\infty)$ 上的递增函数. 又因 $f$ 在 $\mathrm{R}$ 上有界, 所以 $f' = 0$, 即 $f$ 是 $\mathrm{R}$ 上的常值函数.
    \item % 7
        {\heiti 证明}\quad 令 $h(x) = g(x) - f(x)$, 则有 $h'(x) = g'(x) - f'(x) \geqslant 0$, 因此 $h'$ 是 $[a, +\infty)$ 上的递增函数, 那么
        \[
            g(x) - f(x) = h(x) \geqslant h(a) = g(a) - f(a),
        \]
        即
        \[
            g(x) - g(a) \geqslant f(x) - f(a).    
        \]
        再令 $H(x) = g(x) + f(x)$, 同样有 $H'(x) = g'(x) + f'(x) \geqslant 0$, 因此 $H'$ 也是 $[a, +\infty)$ 上的递增函数, 那么
        \[
            g(x) + f(x) = H(x) \geqslant H(a) = g(a) + f(a),    
        \]
        即
        \[
            g(x) - g(a) \geqslant -(f(x) - f(a)).     
        \]
        这样就证明了 $g(x) - g(a) \geqslant |f(x) - f(a)|$.
    \item % 8
    \item % 9
    \item % 10
    \item % 11
    \item % 12
    \item % 13
    \item % 14
    \item % 15
    \item % 16
    \item % 17
    \item % 18
    \item % 19
    \item % 20
    \item % 21
    \item % 22
    \item % 23
\end{enumerate}
\end{document}