% author: Shuning Zhang
% creation date: 2019-12-16
\documentclass[a4paper, 11pt]{ctexart}
\usepackage[top=2cm, bottom=2cm, left=2.5cm, right=2.5cm]{geometry}
\usepackage{enumerate}
\usepackage{amsmath, amssymb}
\begin{document}
\pagestyle{empty}
\begin{enumerate}
    \item % 1
        略.
    \item 2
        令 $f(x) = 1 + x + \cdots + x^n$, 即 $f(x) = \dfrac{1-x^{n+1}}{1-x}$.
        \begin{enumerate}[(1)]
            \item % 2.1
                $1 + 2x + \cdots + nx^{n-1} = f'(x) = \dfrac{nx^{n+1} - (n+1)x^n + 1}{(1-x)^2}$;
            \item % 2.2
                $\sum\limits_{i=1}^n\dfrac{i}{2^{i-1}} = f'(1/2) = \dfrac{n}{2^{n-1}} - \dfrac{n+1}{2^{n-2}} + 4$.
            \item 2.3
        \end{enumerate}
    \item 3
    \item % 4
        {\heiti 证明}\quad 设 $T$ 是 $f$ 的最小正周期, 即 $f(x + T) = f(x)$. 那么
        \begin{align*}
            f'(x + T) &= \lim_{h\to0}\frac{f(x+T+h) - f(x+T)}{h} \\
                      &= \lim_{h\to0}\frac{f(x+h) - f(x)}{h} \\
                      &= f'(x).    
        \end{align*}
        因此, $f'$ 也是周期函数.
    \item % 5
        {\heiti 证明}\quad 若 $f$ 是可导的奇函数, 则有
        \begin{align*}
            f'(-x) &= \lim_{h\to0}\frac{f(-x+h) - f(-x)}{h} \\
                   &= \lim_{h\to0}\frac{-f(x-h) + f(x)}{h} \\
                   &= \lim_{h\to0}\frac{f(x-h) - f(x)}{-h} \\
                   &= f'(x).  
        \end{align*}
        若 $f$ 是可导的偶函数, 则有
        \begin{align*}
            f'(-x) &= \lim_{h\to0}\frac{f(-x+h) - f(-x)}{h} \\
                   &= \lim_{h\to0}\frac{f(x-h) - f(x)}{h} \\
                   &= -\lim_{h\to0}\frac{f(x-h) - f(x)}{-h} \\
                   &= -f'(x).
        \end{align*}
    \item 6
    \item 7
        {\heiti 证明}\quad 根据题意, 有
        \begin{align*}
            f'_+(a)f'_-(b) &= \lim_{h\to0^+}\frac{f(a+h) - f(a)}{h}\cdot\lim_{h\to0^-}\frac{f(b+h) - f(b)}{h} \\
                           &= \lim_{h\to0^+}\frac{\left(f(a+h) - f(a)\right)\left(f(b-h) - f(b)\right)}{-h^2} \\
                           &= \lim_{h\to0^+}\frac{f(a+h)f(b-h)}{-h^2} > 0.
        \end{align*}
        上式表明, 存在一个 $h>0$, 使得当 $x_1\in(a, a+h)$, $x_2\in(b-h, b)$ 时, 有
        \[
            f(x_1)f(x_2) < 0.    
        \]
        考虑区间 $[x_1, x_2] \subset [a, b]$, 由零值定理可知, 存在一点 $c \in (x_1, x_2)$, 使得
        \[
            f(c) = 0.    
        \]
    \item 8
    \item 9
    \item 10
\end{enumerate}
\end{document}