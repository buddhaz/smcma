% \documentclass[12pt, a4paper]{article}

% \usepackage[top=2cm, bottom=2cm, left=2.5cm, right=2.5cm]{geometry}

% \usepackage{ctex}
% \usepackage{amsmath, amssymb}
% \usepackage{enumerate}

% \begin{document}

% \pagestyle{empty}

\begin{center}
    {\heiti 练习题 1.6}
\end{center}

\begin{enumerate}
    \item 
        \begin{enumerate}[(1)]
            \item $\lim\limits_{n\to\infty}\left(1 + \dfrac{1}{n-2}\right)^n = \lim\limits_{n\to\infty}\left(1 + \dfrac{1}{n-2}\right)^{n-2}\left(1 + \dfrac{1}{n-2}\right)^2 = \mathrm{e}$;
            \item $\lim\limits_{n\to\infty}\left(1 - \dfrac{1}{n+3}\right)^n = \lim\limits_{n\to\infty}\left(\dfrac{n+2}{n+3}\right)^n = \lim\limits_{n\to\infty}\dfrac{1}{\left(1 + \frac{1}{n+2}\right)^{n+2}\left(1 + \frac{1}{n+2}\right)^{-2}} = \dfrac{1}{\mathrm{e}}$;
            \item $\lim\limits_{n\to\infty}\left(\dfrac{1+n}{2+n}\right)^n = \lim\limits_{n\to\infty}\dfrac{1}{\left(1 + \frac{1}{n+1}\right)^{n+1} \left(1 + \frac{1}{n+1}\right)^{-1}} = \dfrac{1}{\mathrm{e}}$;
            \item $\lim\limits_{n\to\infty}\left(1 + \dfrac 3n\right)^n = \lim\limits_{n\to\infty}\left(\dfrac{n+3}{n+2}\right)^n\left(\dfrac{n+2}{n+1}\right)^n\left(\dfrac{n+1}{n}\right)^n = \mathrm{e}^3$;
            \item $\lim\limits_{n\to\infty}\left(1 + \dfrac{1}{2n^2}\right)^{4n^2} = \lim\limits_{n\to\infty}\left(1 + \dfrac{1}{2n^2}\right)^{2n^2} \left(1 + \dfrac{1}{2n^2}\right)^{2n^2} = \mathrm{e}^2$.
        \end{enumerate}
    \item {\heiti 证明}\quad 根据题意,则有
        \begin{align*}
                & \lim_{n\to\infty}\left(1 + \frac kn\right)^n \\
            ={} & \lim_{n\to\infty}\left(\frac{n+k}{n}\right)^n \\
            ={} & \lim_{n\to\infty}\underbrace{\left[\frac{n+k}{n+(k-1)}\right]^n\left[\frac{n+(n-1)}{n+(n-2)}\right]^n\cdots\left[\frac{n+1}{n}\right]^n}_{\text{$k$ 项}} \\
            ={} & \lim_{n\to\infty}\underbrace{\frac{\left[1 + \frac{1}{n+(k-1)}\right]^{n+(k-1)}}{\left[1 + \frac{1}{n+(k-1)}\right]^{k-1}} \times \frac{\left[1 + \frac{1}{n+(k-2)}\right]^{n+(k-2)}}{\left[1 + \frac{1}{n+(k-2)}\right]^{k-2}} \times \cdots \times \left[1 + \frac 1n\right]^n}_{\text{$k$ 项}} \\
            ={} & \mathrm{e}^k.
        \end{align*}
        即 $\lim\limits_{n\to\infty}\left(1 + k/n\right)^n = \mathrm{e}^k$.
    \item 略.
    \item 略.
    \item {\heiti 证明}\quad $\because \left\{\left(1 + 1/n\right)^n\right\}$ 是严格递增的数列, $\left\{\left(1 + 1/n\right)^{n+1}\right\}$ 是严格递减的数列,
        而它们的极限都是 $\mathrm{e}$.
        
        $\therefore \left(1 + 1/n\right)^n < \mathrm{e} < \left(1 + 1/n\right)^{n+1}$.
    \item {\heiti 证明}\quad 先证明左边的不等式. 在不等式的两边同乘 $n + 1$, 则有
        \begin{equation*}
            \ln\mathrm{e} = 1 < \ln\left(1 + \frac 1n\right)^{n+1}.    
        \end{equation*}
        由对数函数 $\ln x$ 的严格递增性可知
        \begin{equation*}
            \mathrm{e} < \left(1 + \frac 1n\right)^{n+1}.
        \end{equation*}
        由第 5 题的结论可知上式成立.
        
        采用同样的方法亦可证明右边的不等式.
    \item {\heiti 证明}\quad 先证明右边的不等式. 在不等式的两边同乘 $n$, 则有
        \begin{equation*}
            \ln\left(1 + \frac kn\right)^n < k = k \cdot 1 = k \cdot \ln\mathrm{e} = \ln\mathrm{e}^k.
        \end{equation*}
        由对数函数 $\ln x$ 的严格递增性可知
        \begin{equation*}
            \left(1 + \frac kn\right)^n < \mathrm{e}^k.
        \end{equation*}
        根据第 2 题的结论已经知道 $\lim\limits_{n\to\infty}\left(1 + k/n\right)^n = \mathrm{e}^k$, 再采用第 3 题的证明方法即可证明 $\left\{\left(1 + k/n\right)^n\right\}$ 是严格递增的数列,
        上式得证.
        
        采用同样的证明方法亦可证明左边的不等式.
    \item {\heiti 证明}\quad 先看右边的不等式, 根据第 6 题的结论有
        \[
            \ln\left(1 + \frac 1n\right) < \frac 1n.
        \]
        对上式进行化简, 则有
        \[
            \ln(n+1)-\ln n < \frac 1n.    
        \]
        即
        \begin{equation*}
            \ln2 - \ln1 < 1,
            \ln3 - \ln2 < \frac 12,
            \cdots,
            \ln(n+1) - \ln n < \frac 1n.
        \end{equation*}
        将上列不等式相加, 即可得到
        \[
            \ln(n + 1) < 1 + \frac 12 + \cdots + \frac 1n.    
        \]

        采用同样的证明方法亦可证明左边的不等式.
    \item {\heiti 证明}\quad 先证明 $\{x_n\}$ 是有上界的, 根据第 8 题的结论有
        \begin{align*}
            x_n &< \left(1 + \frac 12 + \cdots \frac 1n\right) - \left(\frac 12 + \frac13 + \cdots + \frac{1}{n+1}\right) \\
            &= 1 - \frac{1}{n+1} \\
            &= \frac{n}{n+1} < 1.
        \end{align*}
        
        再来证明 $\{x_n\}$ 是严格递增的数列. 考察 $x_{n+1} - x_n$, 即
        \begin{align*}
            x_{n+1} - x_n &= \left(1 + \frac12 + \cdots + \frac{1}{n+1} - \ln(n+2)\right) - \left(1 + \frac12 + \cdots + \frac1n - \ln(n+1)\right) \\
            &= \frac{1}{n+1} - \left(\ln(n+2) - \ln(n+1)\right). \\
        \end{align*}
        由第 8 题的证明过程已知 
        \[
            \frac{1}{n+1} > \ln(n+2) - \ln(n+1).
        \]
        因此 $x_{n+1} > x_n$, 所以 $\{x_n\}$ 是严格递增的.
        
        综上所述, 由定理 1.5.1 可知数列 $\{x_n\}$ 的极限存在.
    \item {\heiti 证明}\quad 在等式的两边同时减去 $\ln(n+1)$ 得到
        \begin{equation*}
            1 + \frac12 + \cdots + \frac1n - \ln(n+1) = \ln n - \ln(n+1) + \gamma + \varepsilon_n.
        \end{equation*}
        记等号的左边为 $x_n$, 则有
        \begin{align*}
            x_n &= \ln\frac{n}{n+1} + \gamma + \varepsilon_n \\
            &= -\ln\left(1+\frac1n\right) + \gamma + \varepsilon_n.
        \end{align*}
        现在来证明 $\lim\limits_{n\to\infty}(-\ln(1 + 1/n)) = 0$. 根据第 6 题的结论,即不等式
        \[
            \frac{1}{n+1} < \ln\left(1 + \frac1n\right) < \frac1n    
        \]
        对一切 $n \in \mathrm{N}^*$ 成立. 由夹逼原理可得 $\lim\limits_{n\to\infty}\ln(1 + 1/n) = 0$, 即
        \[
            \lim\limits_{n\to\infty}\left(-\ln\left(1 + \frac1n\right)\right) = 0.  
        \]
        由第 9 题的结论可知
        \[
            \gamma = \lim\limits_{n\to\infty}x_n = \lim\limits_{n\to\infty}\left(-\ln\left(1 + \frac1n\right) + \gamma + \varepsilon_n\right) = \gamma.
        \]
        这就证明了
        \[
            1 + \frac12 + \cdots + \frac1n = \ln n + \gamma + \varepsilon_n.  
        \]
        并且其中的 $\varepsilon_n = \ln(1 + 1/n)$.
    \item {\heiti 证明}\quad 先证左边的不等式. 由第 5 题的结论可知
        \[
            \left(\frac{n+1}{n}\right)^n < \mathrm{e},    
        \]
        即
        \begin{equation*}
            \left(\frac21\right)^1 < \mathrm{e},
            \left(\frac32\right)^2 < \mathrm{e},
            \cdots,
            \left(\frac{n+1}{n}\right)^n < \mathrm{e}.
        \end{equation*}
        将上列不等式相乘, 则有
        \[
            \frac{(n+1)^n}{n!} = \frac21\times\frac{3^2}{2^2}\times\cdots\times\frac{(n+1)^n}{n^n} < \mathrm{e}^n.    
        \]
        通过移项即可得到 $(n+1)^n/\mathrm{e}^n < n!$, 这样就证明了左边的不等式.

        同理可证右边的不等式.
    \item {\heiti 证明}\quad 利用第 11 题的结论, 即对不等式
        \[
            \frac{1}{\mathrm{e}}\left(1 + \frac1n\right) < \frac{\sqrt[n]{n!}}{n} < \frac{1}{\mathrm{e}}\left(1 + \frac1n\right)\sqrt[n]{n+1}  
        \]
        的两边取极限, 其中 $\lim\limits_{n\to\infty}\sqrt[n]{n+1} = 1$. 由夹逼原理即可得到
        \[
            \lim\limits_{n\to\infty}\frac{\sqrt[n]{n!}}{n} = \frac{1}{\mathrm{e}}.    
        \]
    \item 13
    \item 14
    \item % 15
        {\heiti 证明}\quad 利用第 9 题的结论, 则有
        \[
            1 + \frac12 + \cdots + \frac1n = \ln n + \gamma + \varepsilon_n.    
        \]
        那么
        \[
            1 + \frac12 + \cdots + \frac{1}{2n} = \ln 2n + \gamma + \varepsilon_{2n}.    
        \]
        两式相减, 得到
        \[
            \frac{1}{n+1} + \frac{1}{n+2} + \cdots + \frac{1}{n + n} = \ln 2n - \ln n + \varepsilon_{2n} - \varepsilon_n = \ln2 + \varepsilon_{2n} - \varepsilon_n.     
        \]
        对上式取极限, 便可得到
        \[
            \lim_{n\to\infty} \left(\frac{1}{n+1} + \frac{1}{n+2} + \cdots + \frac{1}{n+n}\right) = \ln2.    
        \]
    \item 16
\end{enumerate}
% \end{document}
