% \documentclass[a4paper, 12pt]{article}

% \usepackage[top=2cm, bottom = 2cm, left = 2.5cm, right = 2.5cm]{geometry}

% \usepackage{ctex}
% \usepackage{amsmath, amsfonts, amssymb}
% \usepackage{enumerate}

% \newcommand{\arccot}{\mathrm{arccot}\,}

% \begin{document}

\begin{center}
    {\heiti 练习题 1.2}
\end{center}

\begin{enumerate}
    \item % 1
        \begin{enumerate}[(1)]
            \item % 1.1
                {\heiti 证明}\quad 根据题意, 则有
                \[
                    \left|\frac{1}{1+\sqrt{n}} - 0\right| = \frac{1}{1+\sqrt{n}} < \frac{1}{\sqrt{n}}.    
                \]
                对 $\forall\varepsilon > 0$, 只要取 $N = [1/\varepsilon^2]$, 当 $n > N$ 时, 便有
                \[
                    \left|\frac{1}{1+\sqrt{n}} - 0\right| < \varepsilon. 
                \]
            \item % 1.2
                {\heiti 证明}\quad 根据题意, 则有
                \[
                    \left|\frac{\sin n}{n} - 0\right| = \frac{|\sin n|}{n} \leqslant \frac1n.    
                \]
                对 $\forall\varepsilon > 0$, 只要取 $N = [1/\varepsilon]$, 当 $n > N$ 时, 便有
                \[
                    \left|\frac{\sin n}{n} - 0\right| < \varepsilon.    
                \]
            \item % 1.3
                {\heiti 证明}\quad 利用均值不等式
                \[
                    \sqrt[n]{n!} = \sqrt[n]{1\cdot2\cdots n} \leqslant \frac{1 + 2 + \cdots + n}{n}.   
                \]
                便可得到
                \[
                    \left|\frac{n!}{n^n} - 0\right| = \frac{\sqrt[n]{n!}}{n} \leqslant \frac{1+2+\cdots+n}{n^2} = \frac{\frac{n(n+1)}{2}}{n^2} = \frac12\left(1 + \frac1n\right) < 1 + \frac1n.
                \]
                对 $\forall\varepsilon > 0$, 取 $N = [1/(\varepsilon-1)]$, 当 $n > N$ 时, 便有
                \[
                    \left|\frac{n!}{n^n} - 0\right| < \varepsilon.   
                \]
            \item % 1.4
                {\heiti 证明}\quad 由题意, 可得到
                \[
                    \left|\frac{(-1)^{n-1}}{n} - 0\right| = \frac{|(-1)^{n-1}|}{n} = \frac1n.    
                \]
                对 $\forall\varepsilon > 0$, 取 $N = [1/\varepsilon]$, 当 $n > N$ 时, 便有
                \[
                    \left|\frac{(-1)^{n-1}}{n} - 0\right| < \varepsilon.   
                \]
            \item % 1.5
                {\heiti 证明}\quad 当 $n \geqslant 3$ 时, 有
                \[
                    \left|\frac{2n+3}{5n-10} - \frac25\right| = \frac{2n+3-2(n-2)}{5(n-2)} = \frac{7}{5(n-2)} < \frac{2}{n-2}.    
                \]
                对 $\forall\varepsilon > 0$, 取 $N = [2/\varepsilon + 2]$, 当 $n > N$ 时, 便有
                \[
                    \left|\frac{2n+3}{5n-10} - \frac25\right| < \varepsilon.   
                \]
            \item % 1.6
                {\heiti 证明}\quad 根据题意, 可得到
                \[
                    |0.\underbrace{99\cdots9}_{\text{$n$ 个}} - 1| = 0.\underbrace{00\cdots0}_{\text{$n$ 个}}9\cdots < \frac{1}{10^n}.    
                \]
                对 $\forall\varepsilon > 0$, 取 $N = [\lg(1/\varepsilon)]$, 当 $n > N$ 时, 便有
                \[
                    |0.\underbrace{99\cdots9}_{\text{$n$ 个}} - 1| < \varepsilon.   
                \]
                这便证明了 $\lim\limits_{n\to\infty}0.\underbrace{99\cdots9}_{\text{$n$ 个}} = 1$.
            \item % 1.7
                {\heiti 证明}\quad 因为
                \[
                    \left|\frac{1+2+\cdots+n}{n^2} - \frac12\right| = \frac{n(n+1)-n^2}{2n^2} = \frac{1}{2n}.    
                \]
                所以对 $\forall\varepsilon > 0$, 取 $N = [1/2\varepsilon]$, 当 $n > N$ 时, 便有
                \[
                    \left|\frac{1+2+\cdots+n}{n^2} - \frac12\right| < \varepsilon.   
                \]
            \item % 1.8
                略.
            \item % 1.9
                {\heiti 证明}\quad 利用等式
                \[
                    \arccot x = \frac{\pi}{2} - \arctan x.    
                \]
                则有
                \[
                    \left|\arctan n - \frac{\pi}{2}\right| = \arccot n.    
                \]
                对 $\forall\varepsilon > 0$, 取 $N = [\cot\varepsilon]$, 当 $n > N$ 时, 便有
                \[
                    \left|\arctan n - \frac{\pi}{2}\right| = \arccot n < \varepsilon.   
                \]
            \item 1.10
                % {\heiti 证明} 因为
                % \[
                %     \frac{n^2}{1+n^2} < 1\quad(n\in\mathrm{N}^*).    
                % \]
                % 所以
                % \[
                %     \left|\frac{n^2\arctan n}{1+n^2} - \frac{\pi}{2}\right| < \left|\arctan n - \frac{\pi}{2}\right|.    
                % \]
                % 根据第 $(9)$ 题的结论, 可知
                % \[
                %     \lim_{n\to\infty}\frac{n^2\arctan n}{1+n^2} = \frac{\pi}{2}.    
                % \]
        \end{enumerate}
    \item % 2
        {\heiti 证明}\quad 由题意, 对 $\forall \varepsilon > 0$, $\exists N \in \mathrm{N}^*$, 当 $n > N$ 时,使得不等式
        \begin{equation*}
            |a_n - a| < \varepsilon.
        \end{equation*}
        成立. 另一方面,有 $| |a_n| - |a| | \leqslant \vert a_n - a \vert < \varepsilon$, 即 $\lim\limits_{n\to\infty}\vert a_n \vert = \vert a \vert$.
        
        当 $a_n = (-1)^{n-1}$ 时, 该命题的逆命题为假.
    \item % 3
        {\heiti 证明}\quad 设数列 $\{a_n\}$ 从第 $N$ 项开始后面的项全部等于常数 $c$, 现在来证明 $c$ 就是 $\{a_n\}$ 的极限.
        对 $\forall\varepsilon > 0$, 要使 $|a_n - c| < \varepsilon$ 成立, 只要 $n \geqslant N$ 即可.
        因为当 $n \geqslant N$ 时,有
        \[
            |a_n - c| = |c - c| = 0 < \varepsilon.    
        \]
        这便证明了 $\lim\limits_{n\to\infty}a_n = c$.
    \item 4
    \item % 5
        对 $\forall N \in \mathrm{N}^*$, 依然 $\exists M > 0$, 即便 $n > N$, 也有不等式
        \[
            |a_n - a| \geqslant M    
        \]
        成立.
    \item 6
    \item 7
\end{enumerate}

% \end{document}
