% % @author Shuning Zhang
% % @date 2019-02-10
% \documentclass[a4paper, 11pt]{ctexart}
% \usepackage{amsfonts, amsmath, amssymb}
% \usepackage{color}
% \usepackage{enumerate}
% \usepackage[bottom=2cm, left=2.5cm, right=2.5cm, top=2cm]{geometry}
% \usepackage{multicol}
% \begin{document}
\begin{enumerate}
    \item % 1
        略.
    \item % 2
        略.
    \item % 3
        略.
    \item % 4
        \begin{enumerate}[(1)]
            \item % 4.1
                {\heiti 证明}\quad \begin{align*}
                    |(f(\boldsymbol{x}) \pm g(\boldsymbol{x})) - (l \pm m)| &= |(f(\boldsymbol{x}) - l) \pm (g(\boldsymbol{x}) - m)| \\
                    &\leqslant |f(\boldsymbol{x}) - l| + |g(\boldsymbol{x}) - m| \\
                    &< \frac{\varepsilon}{2} + \frac{\varepsilon}{2} = \varepsilon.
                \end{align*}
            \item % 4.2
                {\heiti 证明}\quad \begin{align*}
                    |f(\boldsymbol{x})g(\boldsymbol{x}) - lm| &= |f(\boldsymbol{x})g(\boldsymbol{x}) - f(\boldsymbol{x})m + f(\boldsymbol{x})m - lm| \\
                    &= |f(\boldsymbol{x})(g(\boldsymbol{x}) - m) + m(f(\boldsymbol{x}) - l)| \\
                    &\leqslant |f(\boldsymbol{x})||g(\boldsymbol{x}) - m| + |m||f(\boldsymbol{x}) - l| \\
                    &\leqslant M|g(\boldsymbol{x}) - m| + |m||f(\boldsymbol{x}) - l| \\
                    &< M\frac{\varepsilon}{2M} + |m|\frac{\varepsilon}{2|m|} = \varepsilon.
                \end{align*}
            \item % 4.3
                {\heiti 证明}\quad 
                \begin{align*}
                    \left| \frac{f(\boldsymbol{x})}{g(\boldsymbol{x})} - \frac{l}{m} \right| &= \left| \frac{mf(\boldsymbol{x}) - lg(\boldsymbol{x})}{mg(\boldsymbol{x})} \right| \\
                    &= \frac{|mf(\boldsymbol{x}) - ml + ml - lg(\boldsymbol{x})|}{|m||g(\boldsymbol{x})|} \\
                    &\leqslant \frac{|m||f(\boldsymbol{x}) - l| + |l||g(\boldsymbol{x}) - m|}{|m||g(\boldsymbol{x})|} \\
                    &= \frac{|f(\boldsymbol{x}) - l|}{|g(\boldsymbol{x})|} + \frac{|l||g(\boldsymbol{x}) - m|}{|m||g(\boldsymbol{x})|} \\
                    &\leqslant \frac{|f(\boldsymbol{x}) - l|}{M} + \frac{|l||g(\boldsymbol{x}) - m|}{|m|M}.
                \end{align*}
        \end{enumerate}
    \item % 5
        略.
    \item % 6
        {\heiti 证明}\quad 只证明
        \[
            \lim_{x \to 0}\lim_{y \to 0}(x+y)\sin\frac1x\sin\frac1y    
        \]
        不存在.
        固定 $x$ (即将 $x$ 看作一个常数), 那么
        \[
            \lim_{y\to 0}(x + y)\sin\frac1x\sin\frac1y = \sin\frac1x\lim_{y\to 0}(x + y)\sin\frac1y,   
        \]
        其中 $\sin\frac1x$ 是常数, $\sin\frac1y$ 是有界量, 但 $x + y \to x\ (y \to 0)$, 因此极限不存在.
        
        现在来证明 $f(\boldsymbol{x})$ 在趋于 $(0, 0)$ 的极限是 $\boldsymbol{0}$, 那么
        \begin{align*}
            \left| (x+y)\sin\frac1x\sin\frac1y - 0 \right| &= \left| x + y \right|\left|\sin\frac1x\right|\left|\sin\frac1y\right| \\
            &\leqslant |x+y| \\
            &\leqslant |x| + |y|
        \end{align*}
        $0 < \|\boldsymbol{x} - \boldsymbol{0}\| = \|\boldsymbol{x}\| = \sqrt{x^2 + y^2} < \delta = \varepsilon/\sqrt{2}$ 时则上式成立. 因为
        \[
            |x| + |y| \leqslant \sqrt{2}\sqrt{x^2 + y^2} < \sqrt{2}\delta = \varepsilon.    
        \]
        其中 $|x| + |y| \leqslant \sqrt{2}\sqrt{x^2 + y^2}$ 在练习题 8.1 中已经证明过.
    \item % 7
        {\color{red} remained}{\heiti 证明}\quad $\lim\limits_{(x,y)\to(x_0,y_0)}f(x,y) = a$ 存在, 那么对任意的 $\varepsilon > 0$, 存在 $\delta > 0$, 当 $\boldsymbol{x} \in D$ 且 $0 < \|\boldsymbol{x} - \boldsymbol{x}_0\| < \delta\ (\boldsymbol{x}_0 = (x_0, y_0))$ 时, 有
        \[
            |f(\boldsymbol{x}) - a| < \varepsilon.    
        \]
    \item % 8
        {\color{red} remained}
\end{enumerate}
% \end{document}