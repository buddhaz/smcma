% % @author Shuning Zhang
% % @date 2019-02-12
% \documentclass[a4paper, 11pt]{ctexart}
% \usepackage{amsfonts, amsmath, amssymb}
% \usepackage{color}
% \usepackage{enumerate}
% \usepackage[bottom=2cm, left=2.5cm, right=2.5cm, top=2cm]{geometry}
% \usepackage{multicol}
% \begin{document}
\begin{enumerate}
    \item % 1
        \begin{enumerate}[(1)]
            \item % 1.1
                $\{(0,0)\}$;
            \item % 1.2
                $\{(0,0)\}$;
            \item % 1.3
                因为
                \[
                    \lim_{(x,y)\to(0,0)}x\sin\frac1y = 0,    
                \]
                而 $\lim_{(x,y)\to(x,0)}x\sin\frac1y\ (x \neq 0)$ 不存在, 故间断点集为 $\{(x,0) : x \neq 0\}$.
        \end{enumerate}
    \item % 2
        {\color{red} remained}
    \item % 3
        \begin{enumerate}[(1)]
            \item % 3.1
                {\heiti 证明}\quad 考虑集合 $E = \{\boldsymbol{p} \in \mathrm{R}^n : \rho(\boldsymbol{p}, A) > 0\}$, 任取 $\boldsymbol{x} \in E$, 令
                \[
                    r = \frac{\rho(\boldsymbol{x}, A)}{2} > 0,    
                \]
                显然 $B_r(\boldsymbol{x}) \subset E$, 故 $\boldsymbol{x}$ 是 $E$ 的内点, 即 $E$ 是一个开集. 另外
                \[
                    E = \mathrm{R}^n \setminus \{\boldsymbol{p} \in \mathrm{R}^n : \rho(\boldsymbol{p}, A) = 0\} = (A^c)^\circ.    
                \]
                因此 $\{\boldsymbol{p} \in \mathrm{R}^n : \rho(\boldsymbol{p}, A) = 0\} = \overline{A}$.
            \item % 3.2
                {\color{red} remained unfinished}{\heiti 证明}\quad 对任意的 $\boldsymbol{x} \in A$, 有
                \begin{align*}
                    \|\boldsymbol{p} - \boldsymbol{q}\| &= \|\boldsymbol{p} - \boldsymbol{x} + \boldsymbol{x} - \boldsymbol{q}\| \\
                    &\geqslant \|\boldsymbol{p} - \boldsymbol{x}\| - \|\boldsymbol{q} - \boldsymbol{x}\|.
                \end{align*}
                而
                \begin{align*}
                    |\|\boldsymbol{p} - \boldsymbol{x}\| - \|\boldsymbol{q} - \boldsymbol{x}\|| &\geqslant |\inf_{a\in A}\|\boldsymbol{p} - \boldsymbol{a}\| - \inf_{a\in A}\|\boldsymbol{q} - \boldsymbol{a}\|| \\
                    &= |\rho(\boldsymbol{p}, A) - \rho(\boldsymbol{q}, A)|{\color{red} ?}
                \end{align*}
        \end{enumerate}
    \item % 4
        \begin{enumerate}[(1)]
            \item % 4.1
                {\heiti 证明}\quad 令 $f(\boldsymbol{x}) = \rho(\boldsymbol{x}, B)$, 由第 3 题的 (2) 可知 $f(\boldsymbol{x})$ 是 $A$ 上的连续函数.
                而 $A$ 又是紧集, 故存在一点 $\boldsymbol{a}$, 使得
                \[
                    f(\boldsymbol{a}) = \inf{f(A)} = \rho(A, B).    
                \]
            \item % 4.2
                {\heiti 证明}\quad 由 (1) 已经知道存在一点 $\boldsymbol{a} \in A$, 使得 $\rho(\boldsymbol{a}, B) = \rho(A, B)$.
                再考虑函数 $g(\boldsymbol{x}) = \rho(\{\boldsymbol{a}\}, \boldsymbol{x})$, 此时将 $\{\boldsymbol{a}\}$ 是一个独点集, 根据 (1) 的结论, 则
                存在一点 $b \in B$, 使得
                \[
                    g(\boldsymbol{b}) = \inf{g(B)} = \rho(A, B).     
                \]
                合起来, 正是
                \[
                    \rho(\{\boldsymbol{a}\}, \{\boldsymbol{b}\}) = \|\boldsymbol{a} - \boldsymbol{b}\| = \rho(A, B).    
                \]
            \item % 4.3
                {\color{red} remained}
        \end{enumerate}
    \item % 5
        令 $A = \mathrm{R}^n$, $B = \varnothing$, $A$, $B$ 是闭集, 且 $A \cap B = \varnothing$, 那么
        \[
            \rho(A, B) = \inf\{\|\boldsymbol{b}\|: \boldsymbol{b} \in B\} = \|\boldsymbol{0}\| = 0.  
        \]
    \item % 6
        {\heiti 证明}\quad 令 $E = \{\boldsymbol{p} : \rho(\boldsymbol{p}, A) \leqslant c\}$. 因为 $A$ 有界, 故 $E$ 有界.
        那么只需证明 $E$ 是闭集即可. 考虑 $E^c = \{\boldsymbol{q} : \rho(\boldsymbol{q}, A) > c\}$, 任取 $\boldsymbol{x} \in E^c$, 令
        \[
            r = \frac{\rho(\boldsymbol{x}, A) - c}{2} > \frac{c - c}{2} = 0,    
        \]
        显然 $B_r(\boldsymbol{x}) \subset E^c$, 即 $\boldsymbol{x}$ 是 $E^c$ 的一个内点. 因此 $E^c$ 是开集, 故 $E$ 是闭集.
        这样便证明了 $E$ 是有界闭集, 即 $E$ 是紧集 (compact set).
    \item % 7
        {\heiti 证明}\quad 将 $E$ 分解成 $E = A \cup B$, 其中
        \begin{align*}
            A = \{\boldsymbol{a} \in \mathrm{R}^n : f(\boldsymbol{a}) > 0\}, \\
            B = \{\boldsymbol{b} \in \mathrm{R}^n : f(\boldsymbol{b}) < 0\}.    
        \end{align*}
        显然 $A \neq \varnothing$, $B \neq \varnothing$, $A \cap B = \varnothing$. 若 $A$ 中存在极限点 $\boldsymbol{x}$, 则必有 $\boldsymbol{x} \leqslant 0$,
        但 $\boldsymbol{x} \notin B$, 即 $A' \cap B = \varnothing$. 同理也有 $A \cap B' = \varnothing$. 因此 $E$ 是非连通集.
\end{enumerate}
% \end{document}