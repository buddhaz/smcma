% @author Shuning Zhang
% @date 2019-02-12
\documentclass[a4paper, 11pt]{ctexart}
\usepackage{amsfonts, amsmath, amssymb}
\usepackage{color}
\usepackage{enumerate}
\usepackage[bottom=2cm, left=2.5cm, right=2.5cm, top=2cm]{geometry}
\usepackage{multicol}
\begin{document}
\begin{enumerate}
    \item % 1
        \begin{enumerate}[(1)]
            \item % 1.1
                $\{(0,0)\}$;
            \item % 1.2
                $\{(0,0)\}$;
            \item % 1.3
                因为
                \[
                    \lim_{(x,y)\to(0,0)}x\sin\frac1y = 0,    
                \]
                而 $\lim_{(x,y)\to(x,0)}x\sin\frac1y\ (x \neq 0)$ 不存在, 故间断点集为 $\{(x,0) : x \neq 0\}$.
        \end{enumerate}
    \item % 2
        {\heiti 证明} 
    \item % 3
        \begin{enumerate}[(1)]
            \item % 3.1
                {\heiti 证明}\quad 由距离的定义可知 $\mathrm{R}^n = \rho(\boldsymbol{x}, A) \geqslant 0$, 考虑 $B = \{\boldsymbol{x} \in \mathrm{R}^n : \rho(\boldsymbol{x}, A) > 0\}$,
                因为 $B$ 是由到 $A$ 的距离大于 $0$ 的点组成, 故 $\boldsymbol{x} \notin A$, 即 $\boldsymbol{x} \in A^c$. 另外
                \[
                    \rho(\boldsymbol{x}, A) > \rho(\boldsymbol{x}, A) / 2 > 0.    
                \]
                令 $r = \rho(\boldsymbol{x}, A) / 2$, 显然 $B_r(\boldsymbol{x}) \subset B$, 即 $B$ 是全体内点组成, 因此 $B = (A^c)^\circ$, 即
                \[
                    \{\boldsymbol{x} : \rho(\boldsymbol{x}, A) = 0\} = \mathrm{R}^n \setminus \{\boldsymbol{x} : \rho(\boldsymbol{x}, A) > 0\} = \mathrm{R}^n \setminus (A^c)^\circ = \overline{A}.     
                \]
            \item % 3.2
                {\heiti 证明}\quad 令 $f(\boldsymbol{x}) = \rho(\boldsymbol{x}, A)\ (\boldsymbol{x} \in \mathrm{R}^n)$, 对任意的 $\boldsymbol{p}, \boldsymbol{q} \in \mathrm{R}^n$, 有
                \[
                    |f(\boldsymbol{p}) - f(\boldsymbol{q})| = |\rho(\boldsymbol{p}, A) - \rho(\boldsymbol{q}, A)| \leqslant \|\boldsymbol{p} - \boldsymbol{q}\|.    
                \]
                取 $\delta = \varepsilon$, 当 $\|\boldsymbol{p} - \boldsymbol{q}\| < \delta$ 时, 则有
                \[
                    |f(\boldsymbol{p}) - f(\boldsymbol{q})| < \varepsilon,    
                \]
                即 $f(\boldsymbol{x})$ 在 $\mathrm{R}^n$ 上一致连续, 因此 $f(\boldsymbol{x})$ 在 $\mathrm{R}^n$ 上连续.
        \end{enumerate}
    \item % 4
    \item % 5
        令 $A = \mathrm{R}^n$, $B = \varnothing$, $A$, $B$ 是闭集, 且 $A \cap B = \varnothing$, 那么
        \[
            \rho(A, B) = \inf\{\|\boldsymbol{b}\|: \boldsymbol{b} \in B\} = \|\boldsymbol{0}\| = 0.  
        \]
    \item % 6
        {\heiti 证明}\quad 令 $d = c + 1$, 先证明 $E = \{\boldsymbol{a} : \rho(\boldsymbol{a}, A) < d\}$ 是开集. 任取 $\boldsymbol{x} \in E$,
        令
        \[
            r = \frac{\min\{\rho(\boldsymbol{x}, A), d - \rho(\boldsymbol{x}, A)\}}{2},   
        \]
        显然 $B_r(\boldsymbol{x}) \subset E$, 因此 $E$ 是开集. 考虑这样一个开集族的并
        \[
            F = \bigcup_{i\in[c, d]}\{\boldsymbol{x} : \rho(\boldsymbol{x}, A) < i\}.
        \]
        显然 $F$ 是 $\{\boldsymbol{p} : \rho(\boldsymbol{p}, A) < c\}$ 的一个开覆盖. 自然可以从中选出
        \[
            \left\{\boldsymbol{x} : \rho(\boldsymbol{x}, A) < \frac{c + d}{2}\right\}.    
        \] 显然也可以将其覆盖. 因此 $\{\boldsymbol{p} : \rho(\boldsymbol{p}, A) < c\}$ 是紧集. 
    \item % 7
        {\heiti 证明}\quad 将 $E$ 分解成 $E = A \cup B$, 其中
        \begin{align*}
            A = \{\boldsymbol{a} \in \mathrm{R}^n : f(\boldsymbol{a}) > 0\}, \\
            B = \{\boldsymbol{b} \in \mathrm{R}^n : f(\boldsymbol{b}) < 0\}.    
        \end{align*}
        显然 $A \neq \varnothing$, $B \neq \varnothing$, $A \cap B = \varnothing$. 若 $A$ 中存在极限点 $\boldsymbol{x}$, 则必有 $\boldsymbol{x} \leqslant 0$,
        但 $\boldsymbol{x} \notin B$, 即 $A' \cap B = \varnothing$. 同理也有 $A \cap B' = \varnothing$. 因此 $E$ 是非连通集.
\end{enumerate}
\end{document}