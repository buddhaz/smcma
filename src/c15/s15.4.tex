% % @author Shuning Zhang
% % @date 2020-05-10
% \documentclass[a4paper, 11pt]{ctexart}
% \usepackage{amsfonts, amsmath, amssymb, amsthm}
% \usepackage{color}
% \usepackage{enumerate}
% \usepackage[bottom=2cm, left=2.5cm, right=2.5cm, top=2cm]{geometry}
% \usepackage{multicol}
% \begin{document}
\begin{enumerate}
    \item % 1
        \begin{enumerate}[(1)]
            \item % 1.1
                $1/e$, 级数在 $x = \pm1/e$ 处发散.
            \item % 1.2
                $1$, 级数在 $x = \pm1$ 处发散.
            \item % 1.3
            \item % 1.4
                $0$.
        \end{enumerate}
    \item % 2
        \begin{multicols}{3}
            \begin{enumerate}[(1)]
                \item % 2.1
                    $(0, +\infty)$;
                \item % 2.2
                    $(-1, +\infty)$;
                \item % 2.3
                    $\{x: x \not=0\}$.
            \end{enumerate}
        \end{multicols}
    \item % 3
    \item % 4
        \begin{enumerate}[(1)]
            \item % 4.1
                $\dfrac{1}{2}\ln\frac{1+x}{1-x}$;
            \item % 4.2
                $-\arctan{x}$;
            \item % 4.3
        \end{enumerate}
    \item % 5
        \begin{enumerate}[(1)]
            \item % 5.1
            \item % 5.2
                \begin{proof}
                    由例 8 可知
                    \[
                        \sum_{n=1}^\infty\frac{(n+1)(n+2)}{2}x^n = \frac{1}{(1-x)^3}.    
                    \]
                    对上式等号两边求导, 左边
                    \[
                        \left(\sum_{n=1}^\infty\frac{(n+1)(n+2)}{2}x^n\right)' = \sum_{n=0}^\infty\frac{(n+1)(n+2)(n+3)}{2}x^n.    
                    \]
                    右边
                    \[
                        \left(\frac{1}{(1-x)^3}\right)' = \frac{3}{(1-x)^4}.    
                    \]
                    再把上式分子中的 $3$ 除到左边, 即可得要证的等式.
                \end{proof}
        \end{enumerate}
    \item % 6
        提示: 定理 14.3.5.
    \item % 7
    \item % 8
\end{enumerate}
% \end{document}
