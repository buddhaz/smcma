% % @author Shuning Zhang
% % @date 2020-03-23
% \documentclass[a4paper, 11pt]{ctexart}
% \usepackage{amsfonts, amsmath, amssymb, amsthm}
% \usepackage{color}
% \usepackage{enumerate}
% \usepackage[bottom=2cm, left=2.5cm, right=2.5cm, top=2cm]{geometry}
% \usepackage{multicol}
% \begin{document}
\begin{enumerate}
    \item % 1
        \begin{enumerate}[(1)]
            \item % 1.1
                $\dfrac{\pi}{2}a^4$;
            \item % 1.2
                $-2\pi ab$;
            \item % 1.3
                $0$.
        \end{enumerate}
    \item % 2
        {\color{red}remained}
    \item % 3
        \begin{proof}
            \begin{align*}
                A &= \frac12\int_\alpha^\beta(\varphi\psi' - \psi\varphi')\,\mathrm{d}t \\
                &= \frac12\int\limits_{\partial{\Omega}}-y\,\mathrm{d}x + x\,\mathrm{d}y \\
                &= \iint\limits_{\Omega}\,\mathrm{d}x\mathrm{d}y.  \qedhere 
            \end{align*}
        \end{proof}
    \item % 4
        提示: 运用定理 11.3.2 证明 $\dfrac{\partial{Q}}{\partial{x}} = \dfrac{\partial{P}}{\partial{y}}$ 即可.
    \item % 5
        \begin{proof}
            考虑将 $\boldsymbol{a}$ 作为 $x$ 轴的单位向量, 那么 $\cos(\boldsymbol{n}, \boldsymbol{a})$ 正是 $\boldsymbol{n}$ 与 $x$ 轴的夹角.
            再根据第一型曲线积分的 Green 公式, 则有
            \[
                \int_{\Gamma}1 \cdot \cos(\boldsymbol{n}, \boldsymbol{a})\,\mathrm{d}s = \iint\limits_{D}0\,\mathrm{d}x\mathrm{d}y = 0.   
            \]
            将 $1$ 看作 $P(x, y)$.
        \end{proof}
    \item % 6
        令 $P(x, y) = x$, $Q(x, y) = y$, 那么
        \[
            \frac{\partial{P}}{\partial{x}} = 1,\ \frac{\partial{Q}}{\partial{y}} = 1.    
        \]
        根据第一型曲线积分的 Green 公式, 则有
        \begin{align*}
            \int_{\Gamma}(P\cos(\boldsymbol{n}, \boldsymbol{i}) + Q\cos(\boldsymbol{n}, \boldsymbol{j}))\,\mathrm{d}s &= \iint\limits_{D}\left(\frac{\partial{P}}{\partial{x}} + \frac{\partial{Q}}{\partial{y}}\right)\,\mathrm{d}x\mathrm{d}y \\
            &= 2\iint\limits_{D}\,\mathrm{d}x\mathrm{d}y \\
            &= 2\sigma(D).   
        \end{align*}
        其中 $D$ 是由 $\Gamma$ 围成的区域, $\sigma(D)$ 表示 $D$ 的面积.
    \item % 7
        \begin{enumerate}[(1)]
            \item % 7.1
                $\dfrac{13}{3}$;
            \item % 7.2
                $\pi$.
        \end{enumerate}
    \item % 8
        {\color{red}remained}
\end{enumerate}
% \end{document}
