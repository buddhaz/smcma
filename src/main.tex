\documentclass[a4paper, 11pt]{ctexbook}

\usepackage[top=2cm, bottom = 2cm, left = 2.5cm, right = 2.5cm]{geometry}

\usepackage[bookmarksnumbered=true, colorlinks, linkcolor=black]{hyperref}

% \usepackage{ctex}
\usepackage{amsmath, amsfonts, amssymb}
\usepackage{color}
\usepackage{enumerate}

\newcommand{\arccot}{\mathrm{arccot}\,}
\newcommand{\dif}{\mathrm{d}}

\begin{document}
    \chapter{实数和数列极限}
        \section{实数}
        \section{数列和收敛数列}
        \section{收敛数列的性质}
        \section{数列极限概念的推广}
        \section{单调数列}
        \section{自然对数的底 $e$}
        \section{基本列和 Cauchy 收敛原理}
        \section{上确界和下确界}
        \section{有限覆盖定理}
        \section{上极限和下极限}
        \section{Stolz 定理}
    \chapter{函数的连续性}
        \section{集合的映射}
        \section{集合的势}
        \section{函数}
        \section{函数的极限}
        \section{极限过程的其他形式}
        \section{无穷小与无穷大}
        \section{连续函数}
        \section{连续函数与极限计算}
        \section{函数的一致连续性}
        \section{有限闭区间上连续函数的性质}
        \section{函数的上极限和下极限}
        \section{混沌现象}
    \chapter{函数的导数}
        \section{导数的定义}
        \section{导数的计算}
        \section{高阶导数}
        \section{微分学的中值定理}
        \section{利用导数研究函数}
        \section{L'Hospital 法则}
        \section{函数作图}
    \chapter{一元微分学的顶峰——Taylor 定理}
        \section{函数的微分}
            % author: Shuning Zhang
% creation date: 2019-12-25
\documentclass[a4paper, 11pt]{ctexart}
\usepackage[top=2cm, bottom=2cm, left=2.5cm, right=2.5cm]{geometry}
\usepackage{enumerate}
\usepackage{amsmath, amssymb}
\newcommand{\diff}{\mathrm{d}\,}
\begin{document}
\pagestyle{empty}
\begin{enumerate}
    \item % 1
        \begin{enumerate}[(1)]
            \item % 1.1
                $\diff y = -\dfrac{1}{x^2}\diff x$;
            \item % 1.2
                $\diff y = \dfrac{a}{1 + (ax+b)^2}\diff x$;
            \item % 1.3
                $\diff y = x\sin x\diff x$;
            \item % 1.4
                $\diff y = \dfrac{1}{\sqrt{x^2 + a^2}}\diff x$.
        \end{enumerate}
    \item % 2
        \begin{enumerate}[(1)]
            \item % 2.1
                $\ln x$;
            \item % 2.2
                $\arctan x$;
            \item % 2.3
                $2\sqrt{x}$;
            \item % 2.4
                $x^2 + x$;
            \item % 2.5
                $\sin x - \cos x$;
            \item % 2.6
                $\ln(x + \sqrt{1+x^2})$;
            \item % 2.7
                $-\dfrac{\mathrm{e}^{-ax}}{a}$;
            \item % 2.8
                $\dfrac{\sin^2x}{2}$;
            \item % 2.9
                $\ln\ln x$;
            \item % 2.10
                $\sqrt{x^2 + a^2}$;
            \item % 2.11
                $-\dfrac{\cos^3x}{3}$;
            \item % 2.12
        \end{enumerate}
    \item % 3
    \item % 4
        \begin{enumerate}[(1)]
            \item % 4.1
                $\dfrac{1}{1 + \mathrm{e}^y}$;
            \item % 4.2
                $\dfrac{y}{1 + y}$;
            \item % 4.3
                $-\dfrac{b^2y}{a^2x}$;
            \item % 4.4
                $-\sqrt{\dfrac yx}$;
            \item % 4.5
                $\dfrac{x^3 + y^2\sqrt{x^2+y^2}}{2xy\sqrt{x^2+y^2} - x^2y}$.
        \end{enumerate}
    \item % 5
\end{enumerate}
\end{document}
        \section{带 Peano 余项的 Taylor 定理}
            % % author: Shuning Zhang
% % creation date: 2019-12-29
% \documentclass[a4paper, 11pt]{ctexart}
% \usepackage[top=2cm, bottom=2cm, left=2.5cm, right=2.5cm]{geometry}
% \usepackage{enumerate}
% \usepackage{amsmath, amssymb}
% \begin{document}
% \pagestyle{empty}
\begin{enumerate}
    \item % 1
        \begin{enumerate}[(1)]
            \item % 1.1
                $-\dfrac{1}{12}$;
            \item % 1.2
                $1$;
            \item % 1.3
                $\dfrac16$;
            \item % 1.4
                $\ln^2a$.
        \end{enumerate}
    \item % 2
        略.
    \item % 3
        $\sqrt{e}$.
\end{enumerate}
% \end{document}
        \section{带 Lagrange 余项和 Cauchy 余项的 Taylor 定理}
    \chapter{求导的逆运算}
        \section{原函数的概念}
            % % author: Shuning Zhang
% % creation date: 2019-12-30
% \documentclass[a4paper, 11pt]{ctexart}
% \usepackage[top=2cm, bottom=2cm, left=2.5cm, right=2.5cm]{geometry}
% \usepackage{enumerate}
% \usepackage{amsmath, amssymb}
% \begin{document}
% \pagestyle{empty}
\begin{enumerate}
    \item % 1
        \begin{enumerate}[(1)]
            \item % 1.1
                $\displaystyle{
                    \frac{x^{11}}{11} + \frac{4x^6}{6} + 4x + c
                }$;
            \item % 1.2
                $\displaystyle{
                    -\frac1x - 2\ln|x| + x + c
                }$;
            \item % 1.3
                $\displaystyle{
                    \frac{2\sqrt{x^3}}{3} + \frac{2}{\sqrt{x}} + c
                }$;
            \item % 1.4
                $\sinh x + c$;
            \item % 1.5
                $\cosh x + c$;
            \item % 1.6
                $\displaystyle{
                    \frac{1}{2^x5\ln2} - \frac{2}{5^x\ln5} + c
                }$;
            \item % 1.7
                $\displaystyle{
                    \frac{e^{2x}}{2} - e^x + x + c
                }$;
            \item % 1.8
                $
                    \begin{cases}
                        -(\sin x + \cos x), & \text{若} \sin x \geqslant \cos x; \\
                        (\sin x + \cos x), & \text{若} \sin x < \cos x;
                    \end{cases}
                $
            \item % 1.9
                $\displaystyle{
                    \frac{1}{b-a}\ln\left|\frac{x+a}{x+b}\right| + c
                }$;
            \item % 1.10
                $\displaystyle{
                    \frac{x}{2} + \frac14\sin 2x + c
                }$;
            \item % 1.11
                $\displaystyle{
                    \frac{x}{2} - \frac14\sin 2x + c
                }$;
            \item % 1.12
                $\displaystyle{
                    \frac{x^5}{5} - \frac{x^4}{4} + \frac{x^3}{3} - \frac{x^2}{2} + x - \ln|1+x| + c
                }$;
            \item % 1.13
                $\tan x - \cot x + c$;
            \item % 1.14
                $\displaystyle{
                    \frac{x^3}{3} - x + \arctan x + c
                }$.
        \end{enumerate}
\end{enumerate}
% \end{document}
        \section{分部积分法和换元法}
        \section{有理函数的原函数}
        \section{可有理化函数的原函数}
    \chapter{函数的积分}
        \section{积分的概念}
        \section{可积函数的性质}
        \section{微积分基本定理}
        \section{分部积分与换元}
        \section{可积分理论}
        \section{Lebesgue 定理}
        \section{反常积分}
        \section{数值积分}
    \chapter{积分学的应用}
        \section{积分学在几何学中的应用}
        \section{物理应用举例}
        \section{面积原理}
        \section{Wallis 公式和 Stirling 公式}
    \chapter{多变量函数的连续性}
        \section{$n$ 维 Euclid 空间}
            % % @author Shuning Zhang
% % @date 2019-02-05
% \documentclass[a4paper, 11pt]{ctexart}
% \usepackage{amsfonts, amsmath, amssymb}
% \usepackage{enumerate}
% \usepackage[bottom=2cm, left=2.5cm, right=2.5cm, top=2cm]{geometry}
% \usepackage{multicol}
% \begin{document}
\begin{enumerate}
    \item % 1
        略.
    \item % 2
        略.
    \item % 3
        略.
    \item % 4
        略.
    \item % 5
        {\heiti 证明}\quad 用反证法. 假设 $B_r(\boldsymbol{a}) \cap B_r(\boldsymbol{b}) \neq \varnothing$, 那么必存在一个元素 $\boldsymbol{c} \in B_r(\boldsymbol{a}) \cap B_r(\boldsymbol{b})$.
        这表明 $\boldsymbol{c} \in B_r(\boldsymbol{a})$ 且 $\boldsymbol{c} \in B_r(\boldsymbol{b})$, 进一步, 有
        \[
            \| \boldsymbol{a} - \boldsymbol{c} \| < r, \| \boldsymbol{c} - \boldsymbol{b} \| < r. 
        \]
        再根据三角不等式, 则有
        \begin{align*}
            \| \boldsymbol{a} - \boldsymbol{b} \| \leqslant \| \boldsymbol{a} - \boldsymbol{c} \| + \| \boldsymbol{c} - \boldsymbol{b} \| < 2r.
        \end{align*}
        与 $\| \boldsymbol{a} - \boldsymbol{b} \| = 2r$ 矛盾. 因此 $B_r(\boldsymbol{a}) \cap B_r(\boldsymbol{b}) = \varnothing$.
    \item % 6
        提示: 证明左边的不等式时, 利用不等式
        \[
            \frac{x_1 + x_2 + \cdots + x_n}{n} \leqslant \sqrt{\frac{x_1^2 + x_2^2 + \cdots + x_n^2}{n}}.    
        \]
        右边的不等式两边平方即可得证.
    \item % 7
        {\heiti 证明}\quad 左边的不等式是显然的. 现在来证明右边的不等式, 对 $\| \boldsymbol{x} \|$ 进行放大处理, 则有
        \begin{align*}
            \| \boldsymbol{x} \| &= \sqrt{x_1^2 + x_2^2 + \cdots + x_n^2} \\
                                 &\leqslant \sqrt{\max(x_i^2) + \max(x_i^2) + \cdots + \max(x_i^2)} \\
                                 &= \sqrt{n\max(x_i^2)} = \sqrt{n}\max|x_i| \leqslant n\max|x_i|.  
        \end{align*}
\end{enumerate}
% \end{document}
        \section{$\mathrm{R}^n$ 中点列的极限}
            % % @author Shuning Zhang
% % @date 2019-02-05
% \documentclass[a4paper, 11pt]{ctexart}
% \usepackage{amsfonts, amsmath, amssymb}
% \usepackage{enumerate}
% \usepackage[bottom=2cm, left=2.5cm, right=2.5cm, top=2cm]{geometry}
% \usepackage{multicol}
% \begin{document}
\begin{enumerate}
    \item % 1
        {\heiti 证明}\quad 因为
        \[
            \lim_{n\to\infty}\frac1n = 0, \lim_{n\to\infty}\sqrt[n]{n} = 1,    
        \]
        故 $\lim\limits_{n\to\infty}\boldsymbol{x}_n = (0, 1)$.
    \item % 2
        \begin{enumerate}[(1)]
            \item % 2.1
                {\heiti 证明}\quad 只证明 $\lim\limits_{k\to\infty}(\boldsymbol{x}_k + \boldsymbol{y}_k) = \boldsymbol{a} + \boldsymbol{b}$.
                对任意的 $\varepsilon / 2 > 0$, 存在 $N \in \mathrm{N}^*$, 当 $k > N$ 时, 有
                \[
                    \| \boldsymbol{x}_k - a \| < \frac{\varepsilon}{2},\ \| \boldsymbol{y}_k - b \| < \frac{\varepsilon}{2}.    
                \]
                同时, 也有
                \begin{align*}
                    \| (\boldsymbol{x}_k + \boldsymbol{y}_k) - (\boldsymbol{a} + \boldsymbol{b}) \| &= \| (\boldsymbol{x}_k - \boldsymbol{a}) + (\boldsymbol{y}_k - \boldsymbol{b}) \| \\
                    &\leqslant \| \boldsymbol{x}_k - \boldsymbol{a} \| + \| \boldsymbol{y}_k - \boldsymbol{b} \| \\
                    &< \frac{\varepsilon}{2} + \frac{\varepsilon}{2} = \varepsilon.
                \end{align*}
                这样便证明了 $\lim\limits_{k\to\infty}(\boldsymbol{x}_k + \boldsymbol{y}_k) = \boldsymbol{a} + \boldsymbol{b}$.
            \item % 2.2
                {\heiti 证明}\quad 对任意的 $|\lambda|\varepsilon > 0$, 存在 $N \in \mathrm{N}^*$, 当 $k > N$ 时, 有
                \[
                    \| \lambda\boldsymbol{x}_k - \lambda\boldsymbol{a} \| = |\lambda|\| \boldsymbol{x}_k - \boldsymbol{a} \| < |\lambda|\varepsilon.    
                \]
        \end{enumerate}
    \item % 3
        {\heiti 证明}\quad 对 $\lim\limits_{k\to\infty}\boldsymbol{x}_k = \boldsymbol{l}$, 则有对任意的 $\varepsilon > 0$, 存在 $K \in \mathrm{N}^*$, 当 $k > K$ 时, 有
        \[
            \| \boldsymbol{x}_k - \boldsymbol{l} \| < \varepsilon,    
        \]
        即对于点列 $\{\boldsymbol{x}_k\}$ 对于 $k > K$ 的点都落在 $B_\varepsilon(\boldsymbol{l})$ 中了.
        对于余下的 $\boldsymbol{x}_1$, $\boldsymbol{x}_2$, $\cdots$, $\boldsymbol{x}_K$ 这 $K$ 个点, 必定存在一个 $B_r(\boldsymbol{l})$, 使得它们都落入其中.
        因此对于点列 $\{\boldsymbol{x}_k\}$ 所有的点都必定落入 $B_{\varepsilon + r}(\boldsymbol{l})$ 中, 即 $\{\boldsymbol{x}_k\}$ 有界.
    \item % 4
        {\heiti 证明}\quad 因为基本点列收敛, 而收敛的点列有界, 因此基本点列有界.
    \item % 5
        {\heiti 证明}\quad 级数收敛, 就是对应的部分和数列收敛, 即 $\lim\limits_{n\to\infty} \sum\limits_{k=1}^n \| \boldsymbol{x}_{k+1} - \boldsymbol{x}_k \| = l$.
        那么对任意的 $\varepsilon > 0$, 存在 $N \in \mathrm{N}^*$, 当 $n>N$ 时, 有
        \[
            \left| \sum_{k=1}^n\| \boldsymbol{x}_{k+1} - \boldsymbol{x}_k \| - l \right| < \varepsilon.   
        \]
        对绝对值里面的部分, 有
        \begin{align*}
            \sum_{k=1}^n\| \boldsymbol{x}_{k+1} - \boldsymbol{x}_k \| - l &= \| \boldsymbol{x}_2 - \boldsymbol{x}_1 \| + \| \boldsymbol{x}_3 - \boldsymbol{x}_2 \| + \cdots + \| \boldsymbol{x}_{n+1} - \boldsymbol{x}_n \| - l \\
            &\geqslant \| \boldsymbol{x}_2 - \boldsymbol{x}_1 + \boldsymbol{x}_3 - \boldsymbol{x}_2 + \cdots + \boldsymbol{x}_{n+1} - \boldsymbol{x}_n \| - l \\
            &= \| \boldsymbol{x}_{n+1} - \boldsymbol{x}_1 \| - l \\
            &\geqslant \| \boldsymbol{x}_{n+1} \| - \| \boldsymbol{x}_1 \| - l \\
            &= \| \boldsymbol{x}_{n+1} - \boldsymbol{l} + \boldsymbol{l} \| - (l + \|\boldsymbol{x}_1\|) \\
            &\geqslant \| \boldsymbol{x}_{n+1} - \boldsymbol{l} \| - \| \boldsymbol{l} \| - (l + \|\boldsymbol{x}_1\|) \\
            &= \| \boldsymbol{x}_{n+1} - \boldsymbol{l} \| - (l + \|\boldsymbol{l}\| + \|\boldsymbol{x}_1\|).
        \end{align*}
        即
        \begin{align*}
            \| \boldsymbol{x}_{n+1} - \boldsymbol{l} \| - (l + \|\boldsymbol{l}\| + \|\boldsymbol{x}_1\|) &\leqslant \sum_{k=1}^n\| \boldsymbol{x}_{k+1} - \boldsymbol{x}_k \| - l \\
            &\leqslant \left|\sum_{k=1}^n\| \boldsymbol{x}_{k+1} - \boldsymbol{x}_k \| - l\right| < \varepsilon.
        \end{align*}
        令 $A = l + \|\boldsymbol{l}\| + \|\boldsymbol{x}_1\|$, $A$ 是一个确切的数, 并将 $A$ 移到右边, 得到
        \[
            \| \boldsymbol{x}_{n+1} - \boldsymbol{l} \| < \varepsilon + A.     
        \]
        这便证明点列 $\{\boldsymbol{x}_i\}$ 收敛.
\end{enumerate}
% \end{document}
        \section{$\mathrm{R}^n$ 中的开集和闭集}
            % @author Shuning Zhang
% @date 2019-02-06
\documentclass[a4paper, 11pt]{ctexart}
\usepackage{amsfonts, amsmath, amssymb}
\usepackage{enumerate}
\usepackage[bottom=2cm, left=2.5cm, right=2.5cm, top=2cm]{geometry}
\usepackage{multicol}
\begin{document}
\begin{enumerate}
    \item % 1
    \item % 2
    \item % 3
    \item % 4
        {\heiti 证明}\quad 因为
        \[
            \boldsymbol{x} \in \Bigl(\bigcup_i E_i\Bigr)^c \Rightarrow \boldsymbol{x} \notin \bigcup_i E_i
            \Rightarrow \boldsymbol{x} \notin E_i \Rightarrow \boldsymbol{x} \in E_i^c
            \Rightarrow \boldsymbol{x} \in \bigcap_i E_i^c,    
        \]
        故 $\displaystyle{\Bigl(\bigcup_i E_i\Bigr)^c \subset \bigcap_i E_i^c}$. 反之, 则有 $\displaystyle{\bigcap_i E_i^c \subset \Bigl(\bigcup_i E_i\Bigr)^c}$.
        因此 $\displaystyle{\Bigl(\bigcup_i E_i\Bigr)^c = \bigcap_i E_i^c}$.
        
        同理可得 $\displaystyle{\Bigl(\bigcap_i E_i\Bigr)^c = \bigcup_i E_i^c}$.
    \item % 5
    \item % 6
    \item % 7
    \item % 8
    \item % 9
    \item % 10
    \item % 11
    \item % 12
    \item % 13
\end{enumerate}
\end{document}
        \section{列紧集和紧致集}
            % % @author Shuning Zhang
% % @date 2019-02-06
% \documentclass[a4paper, 11pt]{ctexart}
% \usepackage{amsfonts, amsmath, amssymb}
% \usepackage{color}
% \usepackage{enumerate}
% \usepackage[bottom=2cm, left=2.5cm, right=2.5cm, top=2cm]{geometry}
% \usepackage{multicol}
% \begin{document}
\begin{enumerate}
    \item % 1
        {\heiti 证明}\quad 因为 $A$ 是紧致集, 故 $A$ 是有界闭集. 根据练习题 8.3 中的第 13 题的结论,
        可知 $P(A)$ 也是有界闭集, 因此 $P(A)$ 也是紧致集.
    \item % 2
        {\heiti 证明}\quad 必要性. 因为 $A \times B$ 是紧致集, 故存在一开集族 $\bigcup_{i \in I}C_i$, 从中选出有限个, 使得
        \[
            A \times B \subset \bigcup_{i}^{n}C_i.    
        \]
        另一方面, 有 $A \subset A \times B$, $B \subset A \times B$, 因此
        \[
            A \subset \bigcup_{i}^{n}C_i,\ B \subset \bigcup_{i}^{n}C_i,   
        \]
        这表明 $A$, $B$ 也被这有限个开集所覆盖, 因此 $A$, $B$ 也是紧致集.

        充分性. 因为 $A$ 和 $B$ 都是紧致集, 即 $A$ 和 $B$ 都是有界闭集. $A \times B$ 显然也有界, 现在只需证 $A \times B$ 是闭集即可.
        显然 $\partial(A \times B) \subset (A \times B)$, 根据练习题 8.3 的第 13 题, 即可知 $A \times B$ 是闭集. 因此 $A \times B$ 也是紧致集.
    \item % 3
        {\heiti 证明}\quad 利用 De Morgan 律. 因为 $\displaystyle{A \bigcap \Bigl(\bigcap_{\alpha}A_\alpha\Bigr) = \varnothing}$, 那么
        \[
            A \subset \Bigl(\bigcap_{\alpha}A_\alpha\Bigr)^c = \bigcup_{\alpha}A_\alpha^c,
        \]
        即 $A$ 被 $\displaystyle{\bigcup_{\alpha}A_\alpha^c}$ 覆盖. 同理, 可得
        \[
            A \subset \Bigl(\bigcap_{i=1}^kA_\alpha\Bigr)^c =  \bigcup_{i=1}^kA_i^c,   
        \]
        即 $A$ 被 $\displaystyle{\bigcup_{\alpha}A_\alpha^c}$ 中有限个开集覆盖. 因此 $A$ 是紧致集.
    \item % 4
        {\heiti\color{red} remained}{\heiti 证明}\quad 先证明 $A$ 是闭集, 对于此题, 证明 $E' \subset E$ 即可, 即证明 $E$ 的全体凝聚点都在 $E$ 中.
        根据题意, $E$ 的每一个凝聚点都在 $E$ 的无穷子集中. 显然将这些无穷子集的并依然是 $E$ 的子集, 即所有凝聚点都在 $E$ 中 ($E' \subset E$), 因此 $E$ 是闭集.
        在证明 $A$ 是有界的, 若 $A$ 无界, 
    \item % 5
        令 $F_i = \mathrm{R}^n \setminus B_i(\boldsymbol{0})$, 显然 $F_i$ 是闭集, 且 $F_{i+1} \subset F_i$, 并有
        \[
            \bigcap_{i=1}^{\infty}F_i = \bigcap_{i=1}^{\infty}\left(\mathrm{R}^n \setminus B_i(\boldsymbol{0})\right) = \varnothing.    
        \]
        
        若 $F_i\ (i = 1, 2, \cdots)$ 为紧致集, 假设 $\displaystyle{\bigcap_{i=1}^{\infty}F_i = \varnothing}$.
        那么 $\displaystyle{\Bigl(\bigcap_{i=1}^{\infty}F_i\Bigr)^c = \bigcup_{i=1}^{\infty}F_i^c = \varnothing^c = \mathrm{R}^n}$, 并对每一个 $F_i$ 取补集, 并令 $E_i = F_i^c$.
        则有 $E_i \subset E_{i+1}$, 且每一个 $E_i$ 都是开集. 考虑 $E_1$, 显然 $E_1 \subset \bigcup_{i=1}^{\infty}F_i^c$, 即 $\bigcup_{i=1}^{\infty}F_i^c$ 是 $E_1$ 的开覆盖,
        从 $\bigcup_{i=1}^{\infty}F_i^c$ 中选出 $E_2$, $E_1 \subset E_2$, 即从中选出一个开集就能覆盖 $E_1$, 说明 $E_1$ 是紧致集, 即 $E_1$ 是有界闭集, 这每一个 $E_i$ 都是开集相矛盾.
        因此 $\displaystyle{\bigcap_{i=1}^{\infty}F_i \neq \varnothing}$.
\end{enumerate}
% \end{document}
        \section{集和的连通性}
            % % @author Shuning Zhang
% % @date 2019-02-10
% \documentclass[a4paper, 11pt]{ctexart}
% \usepackage{amsfonts, amsmath, amssymb}
% \usepackage{color}
% \usepackage{enumerate}
% \usepackage[bottom=2cm, left=2.5cm, right=2.5cm, top=2cm]{geometry}
% \usepackage{multicol}
% \begin{document}
\begin{enumerate}
    \item % 1
        {\heiti 证明}\quad 只需证明连通的开集 $E$ 是道路连通的即可. 任取 $\boldsymbol{p} \in E$, 令
        \begin{gather*}
            A = \{\boldsymbol{x} \in E : \text{$\boldsymbol{x}$ 与 $\boldsymbol{p}$ 能用连续曲线连接}\}, \\
            B = \{\boldsymbol{x} \in E : \text{$\boldsymbol{x}$ 与 $\boldsymbol{p}$ 不能用连续曲线相连}\}.
        \end{gather*}
        那么 $E = A \cup B$. 任取 $\boldsymbol{a} \in A$, 作球 $B_r(\boldsymbol{a})$. 因为 $\boldsymbol{a}$ 与 $\boldsymbol{p}$ 能用连续曲线连接, 而 $B_r(\boldsymbol{a})$ 中的任一点又可以与 $\boldsymbol{a}$ 用连续曲线连接,
        因此 $B_r(\boldsymbol{a})$ 中的任一点都能与 $\boldsymbol{p}$ 相连, 即 $B_r(a) \subset A$. 因此 $\boldsymbol{a}$ 是 $A$ 的一个内点, 那么 $A$ 是开集.

        再证明 $B$ 也是开集. 任取 $\boldsymbol{b} \in B$, 若 $B_r(\boldsymbol{b})$ 中存在一点 $\boldsymbol{x}$ 可与 $\boldsymbol{p}$ 相连, 而 $\boldsymbol{x}$ 又可与 $\boldsymbol{b}$ 相连, 那么 $\boldsymbol{b}$ 就可与 $\boldsymbol{p}$ 相连, 产生矛盾,
        因此 $B_r(\boldsymbol{b})$ 中的任一点都不能与 $\boldsymbol{p}$ 相连, 因此 $B_r(\boldsymbol{b}) \subset B$, 即 $B$ 也是开集.
        
        根据定理 8.5.1 可知连通的开集不能分解成两个非空开集之并. 而 $A$ 显然不空 (至少存在一点 $\boldsymbol{p}$), 因此只能 $B = \varnothing$, 即 $E$ 不存在任意两点不能相连.
    \item % 2
        {\heiti 证明}\quad 先证明 $R^n$ 是连通集. 因为 $R^n$ 是道路连通集, 故 $R^n$ 是连通集.
        
        对非空的 $A \subsetneqq R^n$, 则有 $A \cup A^c = R^n$, 且 $A \cap A^c = \varnothing$. 若 $A$ 是既是开集又是集, 那么 $A^c$ 也既是开集又是闭集.
        从开集的角度考虑, 即 $R^n$ 分解成为两个非空开集之并, 与定理 8.5.1 相悖, 因此 $A$ 不可能既是开集又是闭集.
\end{enumerate}
% \end{document}
        \section{多变量函数的极限}
        \section{多变量连续函数}
        \section{连续映射}
    \chapter{多变量函数}
        \section{方向导数和偏导数}
        \section{多变量函数的微分}
        \section{映射的微分}
        \section{复合求导}
        \section{曲线的切线和曲面的切平面}
        \section{隐函数定理}
        \section{隐映射定理}
        \section{逆映射定理}
        \section{高阶偏导数}
        \section{中值定理和 Taylor 定理}
        \section{极值}
        \section{条件极值}
\end{document}
