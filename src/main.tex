\documentclass[a4paper, 11pt]{ctexbook}

\usepackage[top=2cm, bottom = 2cm, left = 2.5cm, right = 2.5cm]{geometry}

\usepackage[bookmarksnumbered=true, colorlinks, linkcolor=black]{hyperref}

% \usepackage{ctex}
\usepackage{amsfonts, amsmath, amssymb, amsthm}
\usepackage{color}
\usepackage{enumerate}
\usepackage{multicol}

\newcommand{\arccot}{\mathrm{arccot}\,}
\newcommand{\dif}{\mathrm{d}}
\newcommand{\diff}{\mathrm{d}}

\begin{document}
    \part{上册}
        \chapter{实数和数列极限}
            \section{实数}
            \section{数列和收敛数列}
            \section{收敛数列的性质}
            \section{数列极限概念的推广}
            \section{单调数列}
            \section{自然对数的底 \texorpdfstring{$e$}{e}}
            \section{基本列和 Cauchy 收敛原理}
            \section{上确界和下确界}
            \section{有限覆盖定理}
            \section{上极限和下极限}
            \section{Stolz 定理}
        \chapter{函数的连续性}
            \section{集合的映射}
            \section{集合的势}
            \section{函数}
            \section{函数的极限}
            \section{极限过程的其他形式}
            \section{无穷小与无穷大}
            \section{连续函数}
            \section{连续函数与极限计算}
            \section{函数的一致连续性}
            \section{有限闭区间上连续函数的性质}
            \section{函数的上极限和下极限}
            \section{混沌现象}
        \chapter{函数的导数}
            \section{导数的定义}
            \section{导数的计算}
            \section{高阶导数}
            \section{微分学的中值定理}
            \section{利用导数研究函数}
            \section{L'Hospital 法则}
            \section{函数作图}
        \chapter{一元微分学的顶峰——Taylor 定理}
            \section{函数的微分}
                % author: Shuning Zhang
% creation date: 2019-12-25
\documentclass[a4paper, 11pt]{ctexart}
\usepackage[top=2cm, bottom=2cm, left=2.5cm, right=2.5cm]{geometry}
\usepackage{enumerate}
\usepackage{amsmath, amssymb}
\newcommand{\diff}{\mathrm{d}\,}
\begin{document}
\pagestyle{empty}
\begin{enumerate}
    \item % 1
        \begin{enumerate}[(1)]
            \item % 1.1
                $\diff y = -\dfrac{1}{x^2}\diff x$;
            \item % 1.2
                $\diff y = \dfrac{a}{1 + (ax+b)^2}\diff x$;
            \item % 1.3
                $\diff y = x\sin x\diff x$;
            \item % 1.4
                $\diff y = \dfrac{1}{\sqrt{x^2 + a^2}}\diff x$.
        \end{enumerate}
    \item % 2
        \begin{enumerate}[(1)]
            \item % 2.1
                $\ln x$;
            \item % 2.2
                $\arctan x$;
            \item % 2.3
                $2\sqrt{x}$;
            \item % 2.4
                $x^2 + x$;
            \item % 2.5
                $\sin x - \cos x$;
            \item % 2.6
                $\ln(x + \sqrt{1+x^2})$;
            \item % 2.7
                $-\dfrac{\mathrm{e}^{-ax}}{a}$;
            \item % 2.8
                $\dfrac{\sin^2x}{2}$;
            \item % 2.9
                $\ln\ln x$;
            \item % 2.10
                $\sqrt{x^2 + a^2}$;
            \item % 2.11
                $-\dfrac{\cos^3x}{3}$;
            \item % 2.12
        \end{enumerate}
    \item % 3
    \item % 4
        \begin{enumerate}[(1)]
            \item % 4.1
                $\dfrac{1}{1 + \mathrm{e}^y}$;
            \item % 4.2
                $\dfrac{y}{1 + y}$;
            \item % 4.3
                $-\dfrac{b^2y}{a^2x}$;
            \item % 4.4
                $-\sqrt{\dfrac yx}$;
            \item % 4.5
                $\dfrac{x^3 + y^2\sqrt{x^2+y^2}}{2xy\sqrt{x^2+y^2} - x^2y}$.
        \end{enumerate}
    \item % 5
\end{enumerate}
\end{document}
            \section{带 Peano 余项的 Taylor 定理}
                % % author: Shuning Zhang
% % creation date: 2019-12-29
% \documentclass[a4paper, 11pt]{ctexart}
% \usepackage[top=2cm, bottom=2cm, left=2.5cm, right=2.5cm]{geometry}
% \usepackage{enumerate}
% \usepackage{amsmath, amssymb}
% \begin{document}
% \pagestyle{empty}
\begin{enumerate}
    \item % 1
        \begin{enumerate}[(1)]
            \item % 1.1
                $-\dfrac{1}{12}$;
            \item % 1.2
                $1$;
            \item % 1.3
                $\dfrac16$;
            \item % 1.4
                $\ln^2a$.
        \end{enumerate}
    \item % 2
        略.
    \item % 3
        $\sqrt{e}$.
\end{enumerate}
% \end{document}
            \section{带 Lagrange 余项和 Cauchy 余项的 Taylor 定理}
        \chapter{求导的逆运算}
            \section{原函数的概念}
                % % author: Shuning Zhang
% % creation date: 2019-12-30
% \documentclass[a4paper, 11pt]{ctexart}
% \usepackage[top=2cm, bottom=2cm, left=2.5cm, right=2.5cm]{geometry}
% \usepackage{enumerate}
% \usepackage{amsmath, amssymb}
% \begin{document}
% \pagestyle{empty}
\begin{enumerate}
    \item % 1
        \begin{enumerate}[(1)]
            \item % 1.1
                $\displaystyle{
                    \frac{x^{11}}{11} + \frac{4x^6}{6} + 4x + c
                }$;
            \item % 1.2
                $\displaystyle{
                    -\frac1x - 2\ln|x| + x + c
                }$;
            \item % 1.3
                $\displaystyle{
                    \frac{2\sqrt{x^3}}{3} + \frac{2}{\sqrt{x}} + c
                }$;
            \item % 1.4
                $\sinh x + c$;
            \item % 1.5
                $\cosh x + c$;
            \item % 1.6
                $\displaystyle{
                    \frac{1}{2^x5\ln2} - \frac{2}{5^x\ln5} + c
                }$;
            \item % 1.7
                $\displaystyle{
                    \frac{e^{2x}}{2} - e^x + x + c
                }$;
            \item % 1.8
                $
                    \begin{cases}
                        -(\sin x + \cos x), & \text{若} \sin x \geqslant \cos x; \\
                        (\sin x + \cos x), & \text{若} \sin x < \cos x;
                    \end{cases}
                $
            \item % 1.9
                $\displaystyle{
                    \frac{1}{b-a}\ln\left|\frac{x+a}{x+b}\right| + c
                }$;
            \item % 1.10
                $\displaystyle{
                    \frac{x}{2} + \frac14\sin 2x + c
                }$;
            \item % 1.11
                $\displaystyle{
                    \frac{x}{2} - \frac14\sin 2x + c
                }$;
            \item % 1.12
                $\displaystyle{
                    \frac{x^5}{5} - \frac{x^4}{4} + \frac{x^3}{3} - \frac{x^2}{2} + x - \ln|1+x| + c
                }$;
            \item % 1.13
                $\tan x - \cot x + c$;
            \item % 1.14
                $\displaystyle{
                    \frac{x^3}{3} - x + \arctan x + c
                }$.
        \end{enumerate}
\end{enumerate}
% \end{document}
            \section{分部积分法和换元法}
            \section{有理函数的原函数}
            \section{可有理化函数的原函数}
        \chapter{函数的积分}
            \section{积分的概念}
            \section{可积函数的性质}
            \section{微积分基本定理}
            \section{分部积分与换元}
            \section{可积分理论}
                % % @author Shuning Zhang
% % @date 2019-02-29
% \documentclass[a4paper, 11pt]{ctexart}
% \usepackage{amsfonts, amsmath, amssymb, amsthm}
% \usepackage{color}
% \usepackage{enumerate}
% \usepackage[bottom=2cm, left=2.5cm, right=2.5cm, top=2cm]{geometry}
% \usepackage{multicol}
% \begin{document}
\begin{enumerate}
    \item % 1
        \begin{proof}
            已知 $f$ 在 $[a, b]$ 上可积, 对任给的 $\varepsilon > 0$, 必存在 $\delta > 0$, 当 $\|\pi\| < \delta$ 时, 有
            \[
                \left| \sum_{i = 1}^nf(\xi_i)\Delta x_i - \int_a^b f\,\mathrm{d}x \right| < \varepsilon.    
            \]
            对这个分割 $\pi$, $p$ 取由每个小区间 $f$ 的下确界组成, $q$ 取由每个小区间 $f$ 的上确界组成, 显然有
            \[
                p \leq f \leq q.
            \]
            由定理 6.5.3 则有
            \[
                \lim_{\|\pi\|\to0}\sum_{i=1}^n\omega_i\Delta x_i = \lim_{\|\pi\|\to0}\sum_{i=1}^n(q_i-p_i)\Delta x_i= 0.   
            \]
            这正是
            \[
                \int_a^b(q(x) - p(x))\,\mathrm{d}x < \varepsilon. \qedhere   
            \]
        \end{proof}
    \item % 2
        {\color{red}unfinished}\begin{proof}
            若 $f$ 是 $[a, b]$ 上的连续函数, 记 $p = f - \varepsilon/3(b-a)$, $q = f + \varepsilon/3(b-a)$, 显然
            \[
                p \leq f \leq q.    
            \]
            而且
            \[
                \int_a^b (q - p)\,\mathrm{d}x = \frac{2\varepsilon}{3(b-a)}\int_a^b\,\mathrm{d}x = \frac23\varepsilon < \varepsilon. \qedhere   
            \]
        \end{proof}
    \item % 3
        提示: 洛必达法则.
    \item % 4
        \begin{proof}
            对区间 $[a, b]$ 作分割, 记每个小区间的上确界为 $M_i$, 则有
            \[
                \sum_{i=1}^n\sigma\Delta x_i \leq \sum_{i=1}^nM_i\Delta_i,    
            \]
            不等式的左边即是 $\sigma(b-a)$, 考察右边的和式, 由定理 6.5.2 即可得
            \[
                \lim_{\|\pi\|\to0}\sum_{i=1}^nM_i\Delta_i = \overline{I}.   
            \]
            而 $f$ 又是 $[a, b]$ 上的可积函数, 再由定理 6.5.3 可知 $\overline{I} = I$, 即
            \[
                \int_a^b f(x)\,\mathrm{d}x = I \geq \sigma(b - a). \qedhere    
            \]
        \end{proof}
\end{enumerate}
% \end{document}

            \section{Lebesgue 定理}
            \section{反常积分}
            \section{数值积分}
        \chapter{积分学的应用}
            \section{积分学在几何学中的应用}
            \section{物理应用举例}
            \section{面积原理}
            \section{Wallis 公式和 Stirling 公式}
        \chapter{多变量函数的连续性}
            \section{\texorpdfstring{$n$}{n} 维 Euclid 空间}
                % % @author Shuning Zhang
% % @date 2019-02-05
% \documentclass[a4paper, 11pt]{ctexart}
% \usepackage{amsfonts, amsmath, amssymb}
% \usepackage{enumerate}
% \usepackage[bottom=2cm, left=2.5cm, right=2.5cm, top=2cm]{geometry}
% \usepackage{multicol}
% \begin{document}
\begin{enumerate}
    \item % 1
        略.
    \item % 2
        略.
    \item % 3
        略.
    \item % 4
        略.
    \item % 5
        {\heiti 证明}\quad 用反证法. 假设 $B_r(\boldsymbol{a}) \cap B_r(\boldsymbol{b}) \neq \varnothing$, 那么必存在一个元素 $\boldsymbol{c} \in B_r(\boldsymbol{a}) \cap B_r(\boldsymbol{b})$.
        这表明 $\boldsymbol{c} \in B_r(\boldsymbol{a})$ 且 $\boldsymbol{c} \in B_r(\boldsymbol{b})$, 进一步, 有
        \[
            \| \boldsymbol{a} - \boldsymbol{c} \| < r, \| \boldsymbol{c} - \boldsymbol{b} \| < r. 
        \]
        再根据三角不等式, 则有
        \begin{align*}
            \| \boldsymbol{a} - \boldsymbol{b} \| \leqslant \| \boldsymbol{a} - \boldsymbol{c} \| + \| \boldsymbol{c} - \boldsymbol{b} \| < 2r.
        \end{align*}
        与 $\| \boldsymbol{a} - \boldsymbol{b} \| = 2r$ 矛盾. 因此 $B_r(\boldsymbol{a}) \cap B_r(\boldsymbol{b}) = \varnothing$.
    \item % 6
        提示: 证明左边的不等式时, 利用不等式
        \[
            \frac{x_1 + x_2 + \cdots + x_n}{n} \leqslant \sqrt{\frac{x_1^2 + x_2^2 + \cdots + x_n^2}{n}}.    
        \]
        右边的不等式两边平方即可得证.
    \item % 7
        {\heiti 证明}\quad 左边的不等式是显然的. 现在来证明右边的不等式, 对 $\| \boldsymbol{x} \|$ 进行放大处理, 则有
        \begin{align*}
            \| \boldsymbol{x} \| &= \sqrt{x_1^2 + x_2^2 + \cdots + x_n^2} \\
                                 &\leqslant \sqrt{\max(x_i^2) + \max(x_i^2) + \cdots + \max(x_i^2)} \\
                                 &= \sqrt{n\max(x_i^2)} = \sqrt{n}\max|x_i| \leqslant n\max|x_i|.  
        \end{align*}
\end{enumerate}
% \end{document}
            \section{\texorpdfstring{$\mathrm{R}^n$}{Rn} 中点列的极限}
                % % @author Shuning Zhang
% % @date 2019-02-05
% \documentclass[a4paper, 11pt]{ctexart}
% \usepackage{amsfonts, amsmath, amssymb}
% \usepackage{enumerate}
% \usepackage[bottom=2cm, left=2.5cm, right=2.5cm, top=2cm]{geometry}
% \usepackage{multicol}
% \begin{document}
\begin{enumerate}
    \item % 1
        {\heiti 证明}\quad 因为
        \[
            \lim_{n\to\infty}\frac1n = 0, \lim_{n\to\infty}\sqrt[n]{n} = 1,    
        \]
        故 $\lim\limits_{n\to\infty}\boldsymbol{x}_n = (0, 1)$.
    \item % 2
        \begin{enumerate}[(1)]
            \item % 2.1
                {\heiti 证明}\quad 只证明 $\lim\limits_{k\to\infty}(\boldsymbol{x}_k + \boldsymbol{y}_k) = \boldsymbol{a} + \boldsymbol{b}$.
                对任意的 $\varepsilon / 2 > 0$, 存在 $N \in \mathrm{N}^*$, 当 $k > N$ 时, 有
                \[
                    \| \boldsymbol{x}_k - a \| < \frac{\varepsilon}{2},\ \| \boldsymbol{y}_k - b \| < \frac{\varepsilon}{2}.    
                \]
                同时, 也有
                \begin{align*}
                    \| (\boldsymbol{x}_k + \boldsymbol{y}_k) - (\boldsymbol{a} + \boldsymbol{b}) \| &= \| (\boldsymbol{x}_k - \boldsymbol{a}) + (\boldsymbol{y}_k - \boldsymbol{b}) \| \\
                    &\leqslant \| \boldsymbol{x}_k - \boldsymbol{a} \| + \| \boldsymbol{y}_k - \boldsymbol{b} \| \\
                    &< \frac{\varepsilon}{2} + \frac{\varepsilon}{2} = \varepsilon.
                \end{align*}
                这样便证明了 $\lim\limits_{k\to\infty}(\boldsymbol{x}_k + \boldsymbol{y}_k) = \boldsymbol{a} + \boldsymbol{b}$.
            \item % 2.2
                {\heiti 证明}\quad 对任意的 $|\lambda|\varepsilon > 0$, 存在 $N \in \mathrm{N}^*$, 当 $k > N$ 时, 有
                \[
                    \| \lambda\boldsymbol{x}_k - \lambda\boldsymbol{a} \| = |\lambda|\| \boldsymbol{x}_k - \boldsymbol{a} \| < |\lambda|\varepsilon.    
                \]
        \end{enumerate}
    \item % 3
        {\heiti 证明}\quad 对 $\lim\limits_{k\to\infty}\boldsymbol{x}_k = \boldsymbol{l}$, 则有对任意的 $\varepsilon > 0$, 存在 $K \in \mathrm{N}^*$, 当 $k > K$ 时, 有
        \[
            \| \boldsymbol{x}_k - \boldsymbol{l} \| < \varepsilon,    
        \]
        即对于点列 $\{\boldsymbol{x}_k\}$ 对于 $k > K$ 的点都落在 $B_\varepsilon(\boldsymbol{l})$ 中了.
        对于余下的 $\boldsymbol{x}_1$, $\boldsymbol{x}_2$, $\cdots$, $\boldsymbol{x}_K$ 这 $K$ 个点, 必定存在一个 $B_r(\boldsymbol{l})$, 使得它们都落入其中.
        因此对于点列 $\{\boldsymbol{x}_k\}$ 所有的点都必定落入 $B_{\varepsilon + r}(\boldsymbol{l})$ 中, 即 $\{\boldsymbol{x}_k\}$ 有界.
    \item % 4
        {\heiti 证明}\quad 因为基本点列收敛, 而收敛的点列有界, 因此基本点列有界.
    \item % 5
        {\heiti 证明}\quad 级数收敛, 就是对应的部分和数列收敛, 即 $\lim\limits_{n\to\infty} \sum\limits_{k=1}^n \| \boldsymbol{x}_{k+1} - \boldsymbol{x}_k \| = l$.
        那么对任意的 $\varepsilon > 0$, 存在 $N \in \mathrm{N}^*$, 当 $n>N$ 时, 有
        \[
            \left| \sum_{k=1}^n\| \boldsymbol{x}_{k+1} - \boldsymbol{x}_k \| - l \right| < \varepsilon.   
        \]
        对绝对值里面的部分, 有
        \begin{align*}
            \sum_{k=1}^n\| \boldsymbol{x}_{k+1} - \boldsymbol{x}_k \| - l &= \| \boldsymbol{x}_2 - \boldsymbol{x}_1 \| + \| \boldsymbol{x}_3 - \boldsymbol{x}_2 \| + \cdots + \| \boldsymbol{x}_{n+1} - \boldsymbol{x}_n \| - l \\
            &\geqslant \| \boldsymbol{x}_2 - \boldsymbol{x}_1 + \boldsymbol{x}_3 - \boldsymbol{x}_2 + \cdots + \boldsymbol{x}_{n+1} - \boldsymbol{x}_n \| - l \\
            &= \| \boldsymbol{x}_{n+1} - \boldsymbol{x}_1 \| - l \\
            &\geqslant \| \boldsymbol{x}_{n+1} \| - \| \boldsymbol{x}_1 \| - l \\
            &= \| \boldsymbol{x}_{n+1} - \boldsymbol{l} + \boldsymbol{l} \| - (l + \|\boldsymbol{x}_1\|) \\
            &\geqslant \| \boldsymbol{x}_{n+1} - \boldsymbol{l} \| - \| \boldsymbol{l} \| - (l + \|\boldsymbol{x}_1\|) \\
            &= \| \boldsymbol{x}_{n+1} - \boldsymbol{l} \| - (l + \|\boldsymbol{l}\| + \|\boldsymbol{x}_1\|).
        \end{align*}
        即
        \begin{align*}
            \| \boldsymbol{x}_{n+1} - \boldsymbol{l} \| - (l + \|\boldsymbol{l}\| + \|\boldsymbol{x}_1\|) &\leqslant \sum_{k=1}^n\| \boldsymbol{x}_{k+1} - \boldsymbol{x}_k \| - l \\
            &\leqslant \left|\sum_{k=1}^n\| \boldsymbol{x}_{k+1} - \boldsymbol{x}_k \| - l\right| < \varepsilon.
        \end{align*}
        令 $A = l + \|\boldsymbol{l}\| + \|\boldsymbol{x}_1\|$, $A$ 是一个确切的数, 并将 $A$ 移到右边, 得到
        \[
            \| \boldsymbol{x}_{n+1} - \boldsymbol{l} \| < \varepsilon + A.     
        \]
        这便证明点列 $\{\boldsymbol{x}_i\}$ 收敛.
\end{enumerate}
% \end{document}
            \section{\texorpdfstring{$\mathrm{R}^n$}{Rn} 中的开集和闭集}
                % @author Shuning Zhang
% @date 2019-02-06
\documentclass[a4paper, 11pt]{ctexart}
\usepackage{amsfonts, amsmath, amssymb}
\usepackage{enumerate}
\usepackage[bottom=2cm, left=2.5cm, right=2.5cm, top=2cm]{geometry}
\usepackage{multicol}
\begin{document}
\begin{enumerate}
    \item % 1
    \item % 2
    \item % 3
    \item % 4
        {\heiti 证明}\quad 因为
        \[
            \boldsymbol{x} \in \Bigl(\bigcup_i E_i\Bigr)^c \Rightarrow \boldsymbol{x} \notin \bigcup_i E_i
            \Rightarrow \boldsymbol{x} \notin E_i \Rightarrow \boldsymbol{x} \in E_i^c
            \Rightarrow \boldsymbol{x} \in \bigcap_i E_i^c,    
        \]
        故 $\displaystyle{\Bigl(\bigcup_i E_i\Bigr)^c \subset \bigcap_i E_i^c}$. 反之, 则有 $\displaystyle{\bigcap_i E_i^c \subset \Bigl(\bigcup_i E_i\Bigr)^c}$.
        因此 $\displaystyle{\Bigl(\bigcup_i E_i\Bigr)^c = \bigcap_i E_i^c}$.
        
        同理可得 $\displaystyle{\Bigl(\bigcap_i E_i\Bigr)^c = \bigcup_i E_i^c}$.
    \item % 5
    \item % 6
    \item % 7
    \item % 8
    \item % 9
    \item % 10
    \item % 11
    \item % 12
    \item % 13
\end{enumerate}
\end{document}
            \section{列紧集和紧致集}
                % % @author Shuning Zhang
% % @date 2019-02-06
% \documentclass[a4paper, 11pt]{ctexart}
% \usepackage{amsfonts, amsmath, amssymb}
% \usepackage{color}
% \usepackage{enumerate}
% \usepackage[bottom=2cm, left=2.5cm, right=2.5cm, top=2cm]{geometry}
% \usepackage{multicol}
% \begin{document}
\begin{enumerate}
    \item % 1
        {\heiti 证明}\quad 因为 $A$ 是紧致集, 故 $A$ 是有界闭集. 根据练习题 8.3 中的第 13 题的结论,
        可知 $P(A)$ 也是有界闭集, 因此 $P(A)$ 也是紧致集.
    \item % 2
        {\heiti 证明}\quad 必要性. 因为 $A \times B$ 是紧致集, 故存在一开集族 $\bigcup_{i \in I}C_i$, 从中选出有限个, 使得
        \[
            A \times B \subset \bigcup_{i}^{n}C_i.    
        \]
        另一方面, 有 $A \subset A \times B$, $B \subset A \times B$, 因此
        \[
            A \subset \bigcup_{i}^{n}C_i,\ B \subset \bigcup_{i}^{n}C_i,   
        \]
        这表明 $A$, $B$ 也被这有限个开集所覆盖, 因此 $A$, $B$ 也是紧致集.

        充分性. 因为 $A$ 和 $B$ 都是紧致集, 即 $A$ 和 $B$ 都是有界闭集. $A \times B$ 显然也有界, 现在只需证 $A \times B$ 是闭集即可.
        显然 $\partial(A \times B) \subset (A \times B)$, 根据练习题 8.3 的第 13 题, 即可知 $A \times B$ 是闭集. 因此 $A \times B$ 也是紧致集.
    \item % 3
        {\heiti 证明}\quad 利用 De Morgan 律. 因为 $\displaystyle{A \bigcap \Bigl(\bigcap_{\alpha}A_\alpha\Bigr) = \varnothing}$, 那么
        \[
            A \subset \Bigl(\bigcap_{\alpha}A_\alpha\Bigr)^c = \bigcup_{\alpha}A_\alpha^c,
        \]
        即 $A$ 被 $\displaystyle{\bigcup_{\alpha}A_\alpha^c}$ 覆盖. 同理, 可得
        \[
            A \subset \Bigl(\bigcap_{i=1}^kA_\alpha\Bigr)^c =  \bigcup_{i=1}^kA_i^c,   
        \]
        即 $A$ 被 $\displaystyle{\bigcup_{\alpha}A_\alpha^c}$ 中有限个开集覆盖. 因此 $A$ 是紧致集.
    \item % 4
        {\heiti\color{red} remained}{\heiti 证明}\quad 先证明 $A$ 是闭集, 对于此题, 证明 $E' \subset E$ 即可, 即证明 $E$ 的全体凝聚点都在 $E$ 中.
        根据题意, $E$ 的每一个凝聚点都在 $E$ 的无穷子集中. 显然将这些无穷子集的并依然是 $E$ 的子集, 即所有凝聚点都在 $E$ 中 ($E' \subset E$), 因此 $E$ 是闭集.
        在证明 $A$ 是有界的, 若 $A$ 无界, 
    \item % 5
        令 $F_i = \mathrm{R}^n \setminus B_i(\boldsymbol{0})$, 显然 $F_i$ 是闭集, 且 $F_{i+1} \subset F_i$, 并有
        \[
            \bigcap_{i=1}^{\infty}F_i = \bigcap_{i=1}^{\infty}\left(\mathrm{R}^n \setminus B_i(\boldsymbol{0})\right) = \varnothing.    
        \]
        
        若 $F_i\ (i = 1, 2, \cdots)$ 为紧致集, 假设 $\displaystyle{\bigcap_{i=1}^{\infty}F_i = \varnothing}$.
        那么 $\displaystyle{\Bigl(\bigcap_{i=1}^{\infty}F_i\Bigr)^c = \bigcup_{i=1}^{\infty}F_i^c = \varnothing^c = \mathrm{R}^n}$, 并对每一个 $F_i$ 取补集, 并令 $E_i = F_i^c$.
        则有 $E_i \subset E_{i+1}$, 且每一个 $E_i$ 都是开集. 考虑 $E_1$, 显然 $E_1 \subset \bigcup_{i=1}^{\infty}F_i^c$, 即 $\bigcup_{i=1}^{\infty}F_i^c$ 是 $E_1$ 的开覆盖,
        从 $\bigcup_{i=1}^{\infty}F_i^c$ 中选出 $E_2$, $E_1 \subset E_2$, 即从中选出一个开集就能覆盖 $E_1$, 说明 $E_1$ 是紧致集, 即 $E_1$ 是有界闭集, 这每一个 $E_i$ 都是开集相矛盾.
        因此 $\displaystyle{\bigcap_{i=1}^{\infty}F_i \neq \varnothing}$.
\end{enumerate}
% \end{document}
            \section{集合的连通性}
                % % @author Shuning Zhang
% % @date 2019-02-10
% \documentclass[a4paper, 11pt]{ctexart}
% \usepackage{amsfonts, amsmath, amssymb}
% \usepackage{color}
% \usepackage{enumerate}
% \usepackage[bottom=2cm, left=2.5cm, right=2.5cm, top=2cm]{geometry}
% \usepackage{multicol}
% \begin{document}
\begin{enumerate}
    \item % 1
        {\heiti 证明}\quad 只需证明连通的开集 $E$ 是道路连通的即可. 任取 $\boldsymbol{p} \in E$, 令
        \begin{gather*}
            A = \{\boldsymbol{x} \in E : \text{$\boldsymbol{x}$ 与 $\boldsymbol{p}$ 能用连续曲线连接}\}, \\
            B = \{\boldsymbol{x} \in E : \text{$\boldsymbol{x}$ 与 $\boldsymbol{p}$ 不能用连续曲线相连}\}.
        \end{gather*}
        那么 $E = A \cup B$. 任取 $\boldsymbol{a} \in A$, 作球 $B_r(\boldsymbol{a})$. 因为 $\boldsymbol{a}$ 与 $\boldsymbol{p}$ 能用连续曲线连接, 而 $B_r(\boldsymbol{a})$ 中的任一点又可以与 $\boldsymbol{a}$ 用连续曲线连接,
        因此 $B_r(\boldsymbol{a})$ 中的任一点都能与 $\boldsymbol{p}$ 相连, 即 $B_r(a) \subset A$. 因此 $\boldsymbol{a}$ 是 $A$ 的一个内点, 那么 $A$ 是开集.

        再证明 $B$ 也是开集. 任取 $\boldsymbol{b} \in B$, 若 $B_r(\boldsymbol{b})$ 中存在一点 $\boldsymbol{x}$ 可与 $\boldsymbol{p}$ 相连, 而 $\boldsymbol{x}$ 又可与 $\boldsymbol{b}$ 相连, 那么 $\boldsymbol{b}$ 就可与 $\boldsymbol{p}$ 相连, 产生矛盾,
        因此 $B_r(\boldsymbol{b})$ 中的任一点都不能与 $\boldsymbol{p}$ 相连, 因此 $B_r(\boldsymbol{b}) \subset B$, 即 $B$ 也是开集.
        
        根据定理 8.5.1 可知连通的开集不能分解成两个非空开集之并. 而 $A$ 显然不空 (至少存在一点 $\boldsymbol{p}$), 因此只能 $B = \varnothing$, 即 $E$ 不存在任意两点不能相连.
    \item % 2
        {\heiti 证明}\quad 先证明 $R^n$ 是连通集. 因为 $R^n$ 是道路连通集, 故 $R^n$ 是连通集.
        
        对非空的 $A \subsetneqq R^n$, 则有 $A \cup A^c = R^n$, 且 $A \cap A^c = \varnothing$. 若 $A$ 是既是开集又是集, 那么 $A^c$ 也既是开集又是闭集.
        从开集的角度考虑, 即 $R^n$ 分解成为两个非空开集之并, 与定理 8.5.1 相悖, 因此 $A$ 不可能既是开集又是闭集.
\end{enumerate}
% \end{document}
            \section{多变量函数的极限}
                % @author Shuning Zhang
% @date 2019-02-10
\documentclass[a4paper, 11pt]{ctexart}
\usepackage{amsfonts, amsmath, amssymb}
\usepackage{color}
\usepackage{enumerate}
\usepackage[bottom=2cm, left=2.5cm, right=2.5cm, top=2cm]{geometry}
\usepackage{multicol}
\begin{document}
    \begin{enumerate}
        \item % 1
            略.
        \item % 2
            略.
        \item % 3
        \item % 4
            \begin{enumerate}[(1)]
                \item % 4.1
                    {\heiti 证明}\quad \begin{align*}
                        |(f(\boldsymbol{x}) \pm g(\boldsymbol{x})) - (l \pm m)| &= |(f(\boldsymbol{x}) - l) \pm (g(\boldsymbol{x}) - m)| \\
                        &\leqslant |f(\boldsymbol{x}) - l| + |g(\boldsymbol{x}) - m| \\
                        &< \frac{\varepsilon}{2} + \frac{\varepsilon}{2} = \varepsilon.
                    \end{align*}
                \item % 4.2
                    {\heiti 证明}\quad \begin{align*}
                        |f(\boldsymbol{x})g(\boldsymbol{x}) - lm| &= |f(\boldsymbol{x})g(\boldsymbol{x}) - f(\boldsymbol{x})m + f(\boldsymbol{x})m - lm| \\
                        &= |f(\boldsymbol{x})(g(\boldsymbol{x}) - m) + m(f(\boldsymbol{x}) - l)| \\
                        &\leqslant |f(\boldsymbol{x})||g(\boldsymbol{x}) - m| + |m||f(\boldsymbol{x}) - l| \\
                        &\leqslant M|g(\boldsymbol{x}) - m| + |m||f(\boldsymbol{x}) - l| \\
                        &< M\frac{\varepsilon}{2M} + |m|\frac{\varepsilon}{2|m|} = \varepsilon.
                    \end{align*}
                \item % 4.3
                    {\heiti 证明}\quad 
                    \begin{align*}
                        \left| \frac{f(\boldsymbol{x})}{g(\boldsymbol{x})} - \frac{l}{m} \right| &= \left| \frac{mf(\boldsymbol{x}) - lg(\boldsymbol{x})}{mg(\boldsymbol{x})} \right| \\
                        &= \frac{|mf(\boldsymbol{x}) - ml + ml - lg(\boldsymbol{x})|}{|m||g(\boldsymbol{x})|} \\
                        &\leqslant \frac{|m||f(\boldsymbol{x}) - l| + |l||g(\boldsymbol{x}) - m|}{|m||g(\boldsymbol{x})|} \\
                        &= \frac{|f(\boldsymbol{x}) - l|}{|g(\boldsymbol{x})|} + \frac{|l||g(\boldsymbol{x}) - m|}{|m||g(\boldsymbol{x})|} \\
                        &\leqslant \frac{|f(\boldsymbol{x}) - l|}{M} + \frac{|l||g(\boldsymbol{x}) - m|}{|m|M}.
                    \end{align*}
            \end{enumerate}
        \item % 5
            略.
        \item % 6
            {\color{red} remained unfinished}{\heiti 证明}\quad 两个累次极限显然不存在, 直接证明, 则有
            \begin{align*}
                \left| (x+y)\sin\frac1x\sin\frac1y - 0 \right| &= \left| x + y \right|\left|\sin\frac1x\right|\left|\sin\frac1y\right| \\
                &\leqslant |x+y| \\
                &\leqslant |x| + |y|
            \end{align*}
        \item % 7
        \item % 8
    \end{enumerate}
\end{document}
            \section{多变量连续函数}
                % @author Shuning Zhang
% @date 2019-02-12
\documentclass[a4paper, 11pt]{ctexart}
\usepackage{amsfonts, amsmath, amssymb}
\usepackage{color}
\usepackage{enumerate}
\usepackage[bottom=2cm, left=2.5cm, right=2.5cm, top=2cm]{geometry}
\usepackage{multicol}
\begin{document}
\begin{enumerate}
    \item % 1
        \begin{enumerate}[(1)]
            \item % 1.1
                $\{(0,0)\}$;
            \item % 1.2
                $\{(0,0)\}$;
            \item % 1.3
                因为
                \[
                    \lim_{(x,y)\to(0,0)}x\sin\frac1y = 0,    
                \]
                而 $\lim_{(x,y)\to(x,0)}x\sin\frac1y\ (x \neq 0)$ 不存在, 故间断点集为 $\{(x,0) : x \neq 0\}$.
        \end{enumerate}
    \item % 2
        {\heiti 证明} 
    \item % 3
        \begin{enumerate}[(1)]
            \item % 3.1
                {\heiti 证明}\quad 由距离的定义可知 $\mathrm{R}^n = \rho(\boldsymbol{x}, A) \geqslant 0$, 考虑 $B = \{\boldsymbol{x} \in \mathrm{R}^n : \rho(\boldsymbol{x}, A) > 0\}$,
                因为 $B$ 是由到 $A$ 的距离大于 $0$ 的点组成, 故 $\boldsymbol{x} \notin A$, 即 $\boldsymbol{x} \in A^c$. 另外
                \[
                    \rho(\boldsymbol{x}, A) > \rho(\boldsymbol{x}, A) / 2 > 0.    
                \]
                令 $r = \rho(\boldsymbol{x}, A) / 2$, 显然 $B_r(\boldsymbol{x}) \subset B$, 即 $B$ 是全体内点组成, 因此 $B = (A^c)^\circ$, 即
                \[
                    \{\boldsymbol{x} : \rho(\boldsymbol{x}, A) = 0\} = \mathrm{R}^n \setminus \{\boldsymbol{x} : \rho(\boldsymbol{x}, A) > 0\} = \mathrm{R}^n \setminus (A^c)^\circ = \overline{A}.     
                \]
            \item % 3.2
                {\heiti 证明}\quad 令 $f(\boldsymbol{x}) = \rho(\boldsymbol{x}, A)\ (\boldsymbol{x} \in \mathrm{R}^n)$, 对任意的 $\boldsymbol{p}, \boldsymbol{q} \in \mathrm{R}^n$, 有
                \[
                    |f(\boldsymbol{p}) - f(\boldsymbol{q})| = |\rho(\boldsymbol{p}, A) - \rho(\boldsymbol{q}, A)| \leqslant \|\boldsymbol{p} - \boldsymbol{q}\|.    
                \]
                取 $\delta = \varepsilon$, 当 $\|\boldsymbol{p} - \boldsymbol{q}\| < \delta$ 时, 则有
                \[
                    |f(\boldsymbol{p}) - f(\boldsymbol{q})| < \varepsilon,    
                \]
                即 $f(\boldsymbol{x})$ 在 $\mathrm{R}^n$ 上一致连续, 因此 $f(\boldsymbol{x})$ 在 $\mathrm{R}^n$ 上连续.
        \end{enumerate}
    \item % 4
    \item % 5
        令 $A = \mathrm{R}^n$, $B = \varnothing$, $A$, $B$ 是闭集, 且 $A \cap B = \varnothing$, 那么
        \[
            \rho(A, B) = \inf\{\|\boldsymbol{b}\|: \boldsymbol{b} \in B\} = \|\boldsymbol{0}\| = 0.  
        \]
    \item % 6
        {\heiti 证明}\quad 令 $d = c + 1$, 先证明 $E = \{\boldsymbol{a} : \rho(\boldsymbol{a}, A) < d\}$ 是开集. 任取 $\boldsymbol{x} \in E$,
        令
        \[
            r = \frac{\min\{\rho(\boldsymbol{x}, A), d - \rho(\boldsymbol{x}, A)\}}{2},   
        \]
        显然 $B_r(\boldsymbol{x}) \subset E$, 因此 $E$ 是开集. 考虑这样一个开集族的并
        \[
            F = \bigcup_{i\in[c, d]}\{\boldsymbol{x} : \rho(\boldsymbol{x}, A) < i\}.
        \]
        显然 $F$ 是 $\{\boldsymbol{p} : \rho(\boldsymbol{p}, A) < c\}$ 的一个开覆盖. 自然可以从中选出
        \[
            \left\{\boldsymbol{x} : \rho(\boldsymbol{x}, A) < \frac{c + d}{2}\right\}.    
        \] 显然也可以将其覆盖. 因此 $\{\boldsymbol{p} : \rho(\boldsymbol{p}, A) < c\}$ 是紧集. 
    \item % 7
        {\heiti 证明}\quad 将 $E$ 分解成 $E = A \cup B$, 其中
        \begin{align*}
            A = \{\boldsymbol{a} \in \mathrm{R}^n : f(\boldsymbol{a}) > 0\}, \\
            B = \{\boldsymbol{b} \in \mathrm{R}^n : f(\boldsymbol{b}) < 0\}.    
        \end{align*}
        显然 $A \neq \varnothing$, $B \neq \varnothing$, $A \cap B = \varnothing$. 若 $A$ 中存在极限点 $\boldsymbol{x}$, 则必有 $\boldsymbol{x} \leqslant 0$,
        但 $\boldsymbol{x} \notin B$, 即 $A' \cap B = \varnothing$. 同理也有 $A \cap B' = \varnothing$. 因此 $E$ 是非连通集.
\end{enumerate}
\end{document}
            \section{连续映射}
        \chapter{多变量函数}
            \section{方向导数和偏导数}
                % % @author Shuning Zhang
% % @date 2019-02-14
% \documentclass[a4paper, 11pt]{ctexart}
% \usepackage{amsfonts, amsmath, amssymb}
% \usepackage{color}
% \usepackage{enumerate}
% \usepackage[bottom=2cm, left=2.5cm, right=2.5cm, top=2cm]{geometry}
% \usepackage{multicol}
% \begin{document}
\begin{enumerate}
    \item % 1
        \begin{multicols}{2}
            \begin{enumerate}[(1)]
                \item % 1.1
                    $\sqrt{2}$;
                \item % 1.2
                    $2/5$.
            \end{enumerate} 
        \end{multicols}
    \item % 2
        $\left(\dfrac{\sqrt{2}}{2}, \dfrac{\sqrt{2}}{2}\right)$,
        $\left(\dfrac{\sqrt{2}}{2}, -\dfrac{\sqrt{2}}{2}\right)$.
    \item % 3
        $(1,0)$, $(0,1)$, $(-1,0)$, $(0,-1)$.
    \item % 4
        空间中 $\boldsymbol{u} = (\sin\theta\cos\varphi, \sin\theta\sin\varphi, \cos\theta)$, 其中 $0 \leqslant \theta \leqslant \pi$, $0 \leqslant \varphi < 2\pi$.
        设 $\boldsymbol{p}_0 = (x_0, y_0, z_0)$ 是平面上一点, 即 $x_0 + y_0 + z_0 = 0$, 那么
        \[
            \boldsymbol{p}_0 + t\boldsymbol{u} = (x_0 + t\sin\theta\cos\varphi, y_0 + t\sin\theta\sin\varphi, z_0 + t\cos\theta).    
        \]
        进一步, 则有
        \begin{align*}
            f(\boldsymbol{p}_0 + t\boldsymbol{u}) &= |(x_0 + t\sin\theta\cos\varphi) + (y_0 + t\sin\theta\sin\varphi) + (z_0 + t\cos\theta)| \\
            &= |t||\sin\theta\cos\varphi + \sin\theta\sin\varphi + \cos\theta|.    
        \end{align*}
        由方向导数的定义, 则有
        \[
            \lim_{t\to0}\frac{f(\boldsymbol{p}_0 + t\boldsymbol{u}) - f(\boldsymbol{p}_0)}{t}
            =
            \lim_{t\to0}\frac{|t||\sin\theta\cos\varphi + \sin\theta\sin\varphi + \cos\theta|}{t}.    
        \]
        要使上面的极限存在, 那么 $|\sin\theta\cos\varphi + \sin\theta\sin\varphi + \cos\theta| = 0$. 解得
        \[
            \theta = \frac{\pi}{2}, \varphi = \frac{3\pi}{4}\ \text{或}\ \theta = \frac{\pi}{2}, \varphi = \frac{7\pi}{4}.    
        \]
        那么方向则为
        \[
            \left(-\frac{\sqrt{2}}{2}, \frac{\sqrt{2}}{2}, 0\right), \left(\frac{\sqrt{2}}{2}, -\frac{\sqrt{2}}{2}, 0\right).    
        \]
    \item % 5
        \begin{enumerate}[(1)]
            \item % 5.1
                $\dfrac{\partial{f}}{\partial{x}}(0,1) = 2$,
                $\dfrac{\partial{f}}{\partial{y}}(0,1) = 2$,
                $\dfrac{\partial{f}}{\partial{x}}(1,2) = \dfrac{5 + \sqrt{5}}{5}$,
                $\dfrac{\partial{f}}{\partial{y}}(1,2) = \dfrac{5 + 2\sqrt{5}}{5}$;
            \item % 5.2
                $\dfrac{\partial{f}}{\partial{x}}(1,2) = \dfrac23$,
                $\dfrac{\partial{f}}{\partial{y}}(1,2) = \dfrac13$;
            \item % 5.3
                $\dfrac{\partial{f}}{\partial{x}}(1,1) = e^2 + 2\cos{1}$,
                $\dfrac{\partial{f}}{\partial{y}}(1,1) = 2e^2 + \cos{1}$.
        \end{enumerate}
    \item % 6
        \begin{enumerate}[(1)]
            \item % 6.1
                $\dfrac{\partial{z}}{\partial{x}} = y + \dfrac1y$,
                $\dfrac{\partial{z}}{\partial{y}} = x - \dfrac{x}{y^2}$;
            \item % 6.2
                $\dfrac{\partial{z}}{\partial{x}} = \dfrac{2x}{y}\sec^2\dfrac{x^2}{y}$,
                $\dfrac{\partial{z}}{\partial{y}} = -\dfrac{x^2}{y^2}\sec^2\dfrac{x^2}{y}$;
            \item % 6.3
                $\dfrac{\partial{z}}{\partial{x}} = yx^{y-1}$,
                $\dfrac{\partial{z}}{\partial{y}} = x^y\ln{x}$;
            \item % 6.4
                $\dfrac{\partial{z}}{\partial{x}} = \dfrac{1}{x + y^2}$,
                $\dfrac{\partial{z}}{\partial{y}} = \dfrac{2y}{x + y^2}$;
            \item % 6.5
                $\dfrac{\partial{z}}{\partial{x}} = -\dfrac{y}{x^2 + y^2}$,
                $\dfrac{\partial{z}}{\partial{y}} = \dfrac{x}{x^2 + y^2}$;
            \item % 6.6
                $\dfrac{\partial{z}}{\partial{x}} = y\cos{xy}$,
                $\dfrac{\partial{z}}{\partial{y}} = x\cos{xy}$;
            \item % 6.7
                $\dfrac{\partial{u}}{\partial{x}} = (3x^2+y^2+z^2)e^{x(x^2+y^2+z^2)}$,
                $\dfrac{\partial{u}}{\partial{y}} = 2xye^{x(x^2+y^2+z^2)}$,
                $\dfrac{\partial{u}}{\partial{z}} = 2xze^{x(x^2+y^2+z^2)}$;
            \item % 6.8
                $\dfrac{\partial{u}}{\partial{x}}=yze^{xyz}$,
                $\dfrac{\partial{u}}{\partial{y}}=xze^{xyz}$,
                $\dfrac{\partial{u}}{\partial{z}}=xye^{xyz}$;
            \item % 6.9
                $\dfrac{\partial{u}}{\partial{x}} = (yz)x^{yz-1}$,
                $\dfrac{\partial{u}}{\partial{y}} = (z\ln{x})x^{yz}$,
                $\dfrac{\partial{u}}{\partial{z}} = (y\ln{x})x^{yz}$;
            \item % 6.10
                $\dfrac{\partial{u}}{\partial{x}} = \dfrac{1}{x+y^2+z^3}$,
                $\dfrac{\partial{u}}{\partial{y}} = \dfrac{2y}{x+y^2+z^3}$,
                $\dfrac{\partial{u}}{\partial{z}} = \dfrac{3z^2}{x+y^2+z^3}$;
            \item % 6.11
                $\dfrac{\partial{u}}{\partial{x_i}} = \dfrac{1}{x_1 + x_2 + \cdots + x_n}$;
            \item % 6.12
                $\dfrac{\partial{u}}{\partial{x_i}} = \dfrac{2x_i}{\sqrt{1 - (x_1^2 + x_2^2 + \cdots + x_n^2)^2}}$.
        \end{enumerate}
\end{enumerate}
% \end{document}
            \section{多变量函数的微分}
                % @author Shuning Zhang
% @date 2019-02-14
\documentclass[a4paper, 11pt]{ctexart}
\usepackage{amsfonts, amsmath, amssymb}
\usepackage{color}
\usepackage{enumerate}
\usepackage[bottom=2cm, left=2.5cm, right=2.5cm, top=2cm]{geometry}
\usepackage{multicol}
\newcommand{\diff}{\mathrm{d}}
\begin{document}
\begin{enumerate}
    \item % 1
    \item % 2
    \item % 3
        {\heiti 证明}\quad 任取 $\boldsymbol{x} = (x_1, x_2), \boldsymbol{h} = (h_1, h_2) \in \mathrm{R}^2$, 则有
        \begin{align*}
            f(\boldsymbol{x} + \boldsymbol{h}) - f(\boldsymbol{x}) &= (x_1 + h_1)(x_2 + h_2) - x_1x_2 \\
            &= x_1h_2 + x_2h_1 + h_1h_2 \\
            &= \lambda_1h_1 + \lambda_2h_2 + h_1h_2 \\
            &= \sum_{i=1}^2\lambda_ih_i + h_1h_2,   
        \end{align*}
        其中 $\lambda_1 = x_2$, $\lambda_2 = x_1$. 只需证明 $h_1h_2 = o(\|\boldsymbol{h}\|)\ (\|\boldsymbol{h}\|\to0)$ 即可, 即证明如下极限:
        \[
            \lim_{\|\boldsymbol{h}\|\to0}\frac{h_1h_2}{\|\boldsymbol{h}\|} = \lim_{(h_1, h_2) \to (0, 0)}\frac{h_1h_2}{\sqrt{h_1^2 + h_2^2}} = 0.    
        \]
        对任意的 $\varepsilon > 0$, 取 $\delta = \varepsilon$, 当 $0 < |\|\boldsymbol{h}\| - 0| = \sqrt{h_1^2 + h_2^2} < \delta$ 时, 则有
        \[
            \left| \frac{h_1h_2}{\sqrt{h_1^2+h_2^2}} - 0 \right| = \frac{|h_1||h_2|}{\sqrt{h_1^2+h_2^2}} \leqslant \sqrt{h_1^2 + h_2^2} < \delta = \varepsilon.   
        \]
        因此便有
        \[
            f(\boldsymbol{x} + \boldsymbol{h}) - f(\boldsymbol{x}) = \sum_{i=1}^2\lambda_ih_i + o(\|\boldsymbol{h}\|)\quad (\|\boldsymbol{h}\| \to 0).
        \]
        这样便证明了 $f(x, y) = xy$ 在 $\mathrm{R}^2$ 上每一点可微.
    \item % 4
        \begin{enumerate}[(1)]
            \item % 4.1
                $\diff{f(1, 2)} = 6\diff{x} - 2\diff{y}$;
            \item % 4.2
                $\diff{f(1, 2, 1)} = (1/2 + e^3\sin1)\diff x + (1/2 + e^3\sin1)\diff y + (-1/2 + e^3\cos1)\diff z$;
            \item % 4.3
                $\diff f(t_1, t_2, \cdots, t_n) = \displaystyle{\sum_{i=1}^n}\frac{t_i}{\sqrt{t_1^2 + t_2^2 + \cdots + t_n^2}}\diff x_i$;
            \item % 4.4
                $\diff f(x_1, x_2, \cdots, x_n) = \cos(x_1 + x_2^2 + \cdots + x_n^n)\displaystyle{\sum_{i=1}^nix_i^{i-1}\diff x_i}$.
        \end{enumerate}
    \item % 5
        \begin{enumerate}[(1)]
            \item % 5.1
                $(2xy^3, 3x^2y^2)$;
            \item % 5.2
                $(2xy\sin{yz}, x^2(\sin{yz}+yz\cos{yz}), x^2y^2\cos{yz})$;
            \item % 5.3
                $\displaystyle{
                    \left(
                        \cos(y-3z) + \frac{y}{\sqrt{1-x^2y^2}}, -x\sin(y-3z) + \frac{x}{\sqrt{1-x^2y^2}}, 3x\sin(y-3z)    
                    \right)
                }$;
            \item % 5.4
                $\dfrac{1}{\|\boldsymbol{x}\|}(x_1, x_2, \cdots, x_n)$.
        \end{enumerate}
    \item % 6
\end{enumerate}
\end{document}

            \section{映射的微分}
                % % @author Shuning Zhang
% % @date 2019-02-16
% \documentclass[a4paper, 11pt]{ctexart}
% \usepackage{amsfonts, amsmath, amssymb, amsthm}
% \usepackage{color}
% \usepackage{enumerate}
% \usepackage[bottom=2cm, left=2.5cm, right=2.5cm, top=2cm]{geometry}
% \usepackage{multicol}
% \begin{document}
\begin{enumerate}
    \item % 1
        \begin{enumerate}[(1)]
            \item % 1.1
                \[
                    \begin{bmatrix}
                        y^2-6x & 2xy \\
                        3 & -10y
                    \end{bmatrix};    
                \]
            \item % 1.2
                \[
                    \begin{bmatrix}
                        yz^2 & xz^2-8y & 2xyz \\
                        3y^2 & 6xy - 2yz & -y^2
                    \end{bmatrix};
                \]
            \item % 1.3
                \[
                    \begin{bmatrix}
                        e^x(\cos{xy} - y\sin{xy}) & -xe^x\sin{xy} \\
                        e^x(\sin{xy} + y\cos{xy}) & xe^x\cos{xy}
                    \end{bmatrix}.    
                \]
        \end{enumerate}
    \item % 2
        \begin{enumerate}[(1)]
            \item % 2.1
                \[
                    \begin{bmatrix}
                        \cos\theta & -r\sin\theta \\
                        \sin\theta & r\cos\theta
                    \end{bmatrix};    
                \]
            \item % 2.2
                \[
                    \begin{bmatrix}
                        \cos\theta & -r\sin\theta & 0 \\
                        \sin\theta & r\cos\theta & 0 \\
                        0 & 0 & 1
                    \end{bmatrix};    
                \]
            \item % 2.3
                \[
                    \begin{bmatrix}
                        \sin\theta\cos\varphi & r\cos\theta\cos\varphi & -r\sin\theta\sin\varphi \\
                        \sin\theta\sin\varphi & r\cos\theta\sin\varphi & r\sin\theta\cos\varphi \\
                        \cos\theta & -r\sin\theta & 0
                    \end{bmatrix}.    
                \]
        \end{enumerate}
    \item % 3
        略.
    \item % 4
        \begin{proof}
            \begin{align*}
                \|\boldsymbol{f}(t)\| = c
                &\Leftrightarrow
                f_1^2(t) + f_2^2(t) + \cdots + f_n^2(t) = c^2 \\
                &\Leftrightarrow
                2f_1(t)f_1'(t) + 2f_2(t)f_2'(t) + \cdots + 2f_n(t)f_n'(t) = 0 \\
                &\Leftrightarrow
                f_1(t)f_1'(t) + f_2(t)f_2'(t) + \cdots + f_n(t)f_n'(t) = 0 \\
                &\Leftrightarrow
                <J\boldsymbol{f}, \boldsymbol{f}> = 0. \qedhere
            \end{align*}
        \end{proof}
    \item % 5
        $\boldsymbol{f}(x, y, z) = \left(\dfrac12\alpha^2(x), \dfrac12\beta^2(y), \dfrac12\gamma^2(z)\right)$.
    \item % 6
        \begin{enumerate}[(1)]
            \item % 6.1
                \begin{proof}
                    $\boldsymbol{f}(\boldsymbol{0}) = \boldsymbol{f}(\lambda\boldsymbol{x} - \lambda\boldsymbol{x}) = \lambda\boldsymbol{f}(\boldsymbol{x}) - \lambda\boldsymbol{f}(\boldsymbol{x}) = \boldsymbol{0}$.
                \end{proof}
            \item % 6.2
                \begin{proof}
                    $\boldsymbol{f}(-\boldsymbol{x}) = \boldsymbol{f}(1 \cdot \boldsymbol{0} + (-1)\boldsymbol{x}) = 1\cdot\boldsymbol{f}(\boldsymbol{0}) + (-1)\boldsymbol{f}(\boldsymbol{x}) = -\boldsymbol{f}(\boldsymbol{x})$.
                \end{proof}
            \item % 6.3
                \begin{proof}
                    对任意 $\boldsymbol{x} = (x_1, x_2, \cdots, x_n) \in \mathrm{R}^n$, 有
                    \begin{align*}
                        \boldsymbol{f}(\boldsymbol{x}) &= \boldsymbol{f}(x_1, x_2, \cdots, x_n) \\
                        &= \boldsymbol{f}(x_1\boldsymbol{e}_1 + x_2\boldsymbol{e}_2 + \cdots + x_n\boldsymbol{e}_n) \\
                        &= x_1\boldsymbol{f}(\boldsymbol{e}_1) + \boldsymbol{f}(x_2\boldsymbol{e}_2 + \cdots + x_n\boldsymbol{e}_n) \\
                        &= x_1\boldsymbol{f}(\boldsymbol{e}_1) + x_2\boldsymbol{f}(\boldsymbol{e}_2) + \cdots + x_n\boldsymbol{f}(\boldsymbol{e}_n). \qedhere
                    \end{align*}
                \end{proof}
        \end{enumerate}
    \item % 7
    \item % 8
        \begin{proof}
            因为 $\boldsymbol{E}(\lambda_1\boldsymbol{x}_1 + \lambda_2\boldsymbol{x}_2) = \lambda_1\boldsymbol{x}_1 + \lambda_2\boldsymbol{x}_2 = \lambda_1\boldsymbol{E}(\boldsymbol{x}_1) + \lambda_2\boldsymbol{E}(\boldsymbol{x}_2)$, 故 $\boldsymbol{E}(\boldsymbol{x})$ 是一个线性映射.
            \[
                \boldsymbol{E}(\boldsymbol{x}) = \boldsymbol{x} = (x_1, x_2, \cdots, x_n) =
                \begin{cases}
                    E_1(\boldsymbol{x}) = x_1, \\
                    E_2(\boldsymbol{x}) = x_2, \\
                    \cdots, \\
                    E_n(\boldsymbol{x}) = x_n.
                \end{cases}    
            \]
            因此
            \[
                \begin{bmatrix}
                    \frac{\partial{E_1}}{x_1} & \frac{\partial{E_1}}{x_2} & \cdots & \frac{\partial{E_1}}{x_n} \\ 
                    \frac{\partial{E_2}}{x_1} & \frac{\partial{E_2}}{x_2} & \cdots & \frac{\partial{E_2}}{x_n} \\
                    \vdots & \vdots &  & \vdots \\ 
                    \frac{\partial{E_n}}{x_1} & \frac{\partial{E_n}}{x_2} & \cdots & \frac{\partial{E_n}}{x_n} \\ 
                \end{bmatrix}
                =
                \begin{bmatrix}
                    1 & 0 & \cdots & 0 \\
                    0 & 1 & \cdots & 0 \\
                    \vdots & \vdots & & \vdots \\
                    0 & 0 & \cdots & 1
                \end{bmatrix}. \qedhere    
            \]
        \end{proof}
\end{enumerate}
% \end{document}

            \section{复合求导}
            \section{曲线的切线和曲面的切平面}
            \section{隐函数定理}
            \section{隐映射定理}
            \section{逆映射定理}
            \section{高阶偏导数}
            \section{中值定理和 Taylor 定理}
            \section{极值}
            \section{条件极值}
    \part{下册}
        \chapter{多重积分}
            \section{矩形区域上的积分}
            \section{Lebesgue 定理}
                % % @author Shuning Zhang
% % @date 2019-03-10
% \documentclass[a4paper, 11pt]{ctexart}
% \usepackage{amsfonts, amsmath, amssymb, amsthm}
% \usepackage{color}
% \usepackage{enumerate}
% \usepackage[bottom=2cm, left=2.5cm, right=2.5cm, top=2cm]{geometry}
% \usepackage{multicol}
% \begin{document}
\begin{enumerate}
    \item % 1
        \begin{proof}
            点列 $\{\boldsymbol{p}_n\}$ 有极限, 即对任意的 $\varepsilon > 0$, 存在 $N > 0$, 当 $n > N$ 时, 有
            \[
                \|\boldsymbol{p}_n - \boldsymbol{l}\| < \varepsilon.    
            \]
            对于 $\{\boldsymbol{p}_1,\cdots,\boldsymbol{p}_N\}$ 这些点组成一个有限集, 因此是一个零面积集.
            对于 $n>N$ 的点, 显然落入开圆 $N_\varepsilon(\boldsymbol{l})$, 且开圆的面积是任意小的.
            因此所有的 $\boldsymbol{p}_n$ 组成一个零面积集. 
        \end{proof}
    \item % 2
        \begin{proof}
            $B$ 可以分解为 $B$ 的部分零聚点和 $B$ 的孤立点的并, 而 $B$ 有界, 故 $B$ 的孤立点是一个至多可数集, 即零面积集. 又 $B'$ 也是零面积集,
            因此 $\bar{B} = B \cup B'$, 也是零面积集.
        \end{proof}
    \item % 3
        \begin{proof}
            $[0, 1]^2$ 上的有理点是一个可数集, 可数集是零测集.
        \end{proof}
    \item % 4
        \begin{proof}
            $f$ 在 $I$ 上可积, 即 $f$ 在 $I$ 上的 $D(f)$ 是一个零测集. 又 $J \subset I$, 故 $f$ 在 $J$ 上的 $D(f)$ 是 $I$ 上的 $D(f)$ 的子集,
            即 $f$ 在 $J$ 上的 $D(f)$ 也是零测集. 因此 $f$ 在 $J$ 上也可积.
        \end{proof}
    \item % 5
        \begin{proof}
            $I$ 不是零面积集, 根据定理 10.2.3 即可知 $\displaystyle{\int_I f\,\mathrm{d}\sigma > 0}$.
        \end{proof}
    \item % 6
        提示: 运用定理 10.2.4.
    \item % 7
        \begin{proof}
            由 $f$ 的定义可知 $f$ 在 $I \setminus B$ 上连续, 在 $B$ 上不连续, 又 $B$ 是一个可数集, 即 $B$ 是一个零测集, 因此 $f$ 在 $I$ 上可积.
        \end{proof}
    \item % 8
        \begin{proof}
            因为 $D(fg) \subset D(f) \cup D(g)$, 所以 $fg$ 在 $I$ 上也可积. 同理 $f/g$ 也在 $I$ 上可积.
        \end{proof}
    \item % 9
        \begin{proof}
            因为 $D(|f|) \subset D(f)$, 因此 $|f|$ 在 $I$ 上也可积. 因为
            \[
                -|f| \leq f \leq |f|,    
            \]
            故
            \[
                -\int_I|f|\,\mathrm{d}\sigma \leq \int_I f \,\mathrm{d}\sigma \leq \int_I|f|\,\mathrm{d}\sigma.    
            \]
            合起来便是
            \[
                \left|\int_If\,\mathrm{d}\sigma\right| \leq \int_I|f|\,\mathrm{d}\sigma. \qedhere    
            \]
        \end{proof}
\end{enumerate}
% \end{document}

            \section{矩形区域上二重积分的计算}
                % % @author Shuning Zhang
% % @date 2019-03-11
% \documentclass[a4paper, 11pt]{ctexart}
% \usepackage{amsfonts, amsmath, amssymb, amsthm}
% \usepackage{color}
% \usepackage{enumerate}
% \usepackage[bottom=2cm, left=2.5cm, right=2.5cm, top=2cm]{geometry}
% \usepackage{multicol}
% \begin{document}
\begin{enumerate}
    \item % 1
        \begin{multicols}{3}
            \begin{enumerate}[(1)]
                \item % 1.1
                    $\pi/12$;
                \item % 1.2
                    $1$;
                \item % 1.3
                    $0$.
            \end{enumerate}
        \end{multicols}
    \item % 2
        $f(b,d)-f(a,d)-f(b,c)+f(a,c)$.
    \item % 3
        \begin{multicols}{2}
            \begin{enumerate}[(1)]
                \item % 3.1
                    $1/3$;
                \item % 3.2
                    {\color{red}remained}
            \end{enumerate}
        \end{multicols}
    \item % 4
    \item % 5
\end{enumerate}
% \end{document}

            \section{有界集合上的二重积分}
                % % @author Shuning Zhang
% % @date 2019-03-11
% \documentclass[a4paper, 11pt]{ctexart}
% \usepackage{amsfonts, amsmath, amssymb, amsthm}
% \usepackage{color}
% \usepackage{enumerate}
% \usepackage[bottom=2cm, left=2.5cm, right=2.5cm, top=2cm]{geometry}
% \usepackage{multicol}
% \begin{document}
\begin{enumerate}
    \item % 1
        \begin{proof}
            由平面点集 $S$ 面积的定义可知
            \[
                \sigma(S) = \int_S1\,\mathrm{d}\sigma = \int_R\chi_S\,\mathrm{d}\sigma.    
            \]
            有面积即表示上式积分可积, 即特征函数 $\chi_S$ 可积, 又特征函数的不连续点的全体 $D(\chi_S) = \partial{S}$,
            因此只能 $\partial{S}$ 是零测集, 又 $\partial{S}$ 是有界闭集, 所以 $\partial{S}$ 是零面积集.
        \end{proof}
    \item % 2
        {\color{red}remained}
    \item % 3
        \begin{proof}
            $S \subset \mathbb{R}^2$ 有面积, 即积分
            \[
                \int_S1\,\mathrm{d}\sigma = \int_R\chi_S\,\mathrm{d}\sigma    
            \]
            存在, 其中 $R$ 是一个闭矩形, 且 $S \subset R^\circ$. 将上式写为 Riemann 和的形式, 则有
            \[
                \sum_{i=1}^n\chi_S(\xi_i)\sigma(R_i).    
            \]
            由于 $\xi_i$ 是任取的, 故对于 $R_i \cap S \not= \varnothing$ 且 $R_i \not\subset S$ 的 $R_i$ 可使得
            \[
                \chi_S(\xi_i) = 0.    
            \]
            因此可忽略那些 $R_i \cap S \not= \varnothing$ 且 $R_i \not\subset S$ 的 $R_i$, 便有
            \[
                \sum_{R_i \cap S \not= \varnothing}\sigma(R_i) = \sum_{R_i \subset S}\sigma(R_i). \qedhere    
            \]
        \end{proof}
\end{enumerate}
% \end{document}

            \section{有界集合上积分的计算}
                % % @author Shuning Zhang
% % @date 2019-03-13
% \documentclass[a4paper, 11pt]{ctexart}
% \usepackage{amsfonts, amsmath, amssymb, amsthm}
% \usepackage{color}
% \usepackage{enumerate}
% \usepackage[bottom=2cm, left=2.5cm, right=2.5cm, top=2cm]{geometry}
% \usepackage{multicol}
% \begin{document}
\begin{enumerate}
    \item % 1
        \begin{multicols}{2}
            \begin{enumerate}[(1)]
                \item % 1.1
                    $-2$;
                \item % 1.2
                    $32/21$;
                \item % 1.3
                    {\color{red}remained}$e-1/e$; 
                \item % 1.4
                    $a^4/2$;
                \item % 1.5
                    $0$;
                \item % 1.6
                    $\cos2 + 2\cos1 - \pi + 3$;
                \item % 1.7
                \item % 1.8
                    $6$.
            \end{enumerate}
        \end{multicols}
    \item % 2
        \begin{enumerate}[(1)]
            \item % 2.1
                $\displaystyle{\int_0^1\,\mathrm{d}y\int_{\sqrt{y}}^1f(x,y)\,\mathrm{d}x}$;
            \item % 2.2
                $\displaystyle{\int_0^1\,\mathrm{d}y\int_{e^y}^ef(x,y)\,\mathrm{d}x}$;
            \item % 2.3
                $\displaystyle{\int_0^1\,\mathrm{d}y\int_y^{\sqrt{y}}}f(x,y)\,\mathrm{d}x$;
            \item % 2.4
                $\displaystyle{\int_a^b\,\mathrm{d}y\int_y^bf(x,y)\,\mathrm{d}x}$;
            \item % 2.5
                $\displaystyle{\int_{-1}^0\,\mathrm{d}y\int_{-\sqrt{1-y^2}}^{\sqrt{1-y^2}}f(x,y)\,\mathrm{d}x + \int_0^1\,\mathrm{d}y\int_{-\sqrt{1-y}}^{\sqrt{1-y}}f(x,y)\,\mathrm{d}x}$;
            \item % 2.6
                $\displaystyle{\int_{-1}^0\,\mathrm{d}y\int_0^{y+1}f(x,y)\,\mathrm{d}x + \int_0^1\,\mathrm{d}y\int_0^{-y+1}f(x,y)\,\mathrm{d}x}$;
            \item % 2.7
                {\color{red}remained}$\displaystyle{\int_{-1}^1\,\mathrm{d}y\int_{0}^{\arccos{y}}f(x,y)\,\mathrm{d}x}$.
        \end{enumerate}
    \item % 3
        \begin{proof}
            改变累次积分的次序, 则有
            \[
                \int_0^af(x)\,\mathrm{d}x\int_0^xf(y)\,\mathrm{d}y = \int_0^af(y)\,\mathrm{d}y\int_y^af(x)\,\mathrm{d}x.   
            \]
            因此
            \begin{align*}
                \int_0^xf(y)\,\mathrm{d}y &= \int_y^af(x)\,\mathrm{d}x \\
                &= \int_y^0f(x)\,\mathrm{d}x + \int_0^af(x)\,\mathrm{d}x \\
                &= -\int_0^yf(x)\,\mathrm{d}x + \int_0^af(x)\,\mathrm{d}x.
            \end{align*}
            将积分变元换为 $t$, 同一个函数 $f$ 对 $0$ 到 $x$ 和 $0$ 到 $y$ 是一样的, 因此
            \[
                \int_0^xf(t)\,\mathrm{d}t = \frac{1}{2}\int_0^af(t)\,\mathrm{d}t.   
            \]
            这样便有
            \[
                \int_0^af(x)\,\mathrm{d}x\int_0^xf(y)\,\mathrm{d}y = \int_0^af(t)\,\mathrm{d}t\int_0^xf(t)\,\mathrm{d}t = \frac12\left(\int_0^af(t)\,\mathrm{d}t\right)^2. \qedhere   
            \]
        \end{proof}
    \item % 4
        \begin{proof}
            改变累次积分的次序即可, 那么
            \[
                \int_0^a\,\mathrm{d}x\int_0^xf(y)\,\mathrm{d}y = \int_0^af(y)\,\mathrm{d}y\int_y^a\,\mathrm{d}x = \int_0^a(a-y)f(y)\,\mathrm{d}y.   
            \]
            再将积分变元换为 $t$ 即可.
        \end{proof}
    \item % 5
    \item % 6
    \item % 7
        \begin{proof}
            根据第 5 题的结论, 则有
            \begin{align*}
                \int_0^a\,\mathrm{d}x\int_0^x\,\mathrm{d}y\int_0^yf(z)\,\mathrm{d}z &= \int_0^af(z)\,\mathrm{d}z\int_z^a\,\mathrm{d}y\int_y^a\,\mathrm{d}x \\
                &= \int_0^af(z)\,\mathrm{d}z\int_z^a(z-y)\,\mathrm{d}z \\
                &= \left.-\frac12\int_0^a(a-y)^2\right|_z^af(z)\,\mathrm{d}z \\
                &= \frac12\int_0^a(a-z)^2f(z)\,\mathrm{d}z. \qedhere  
            \end{align*}
        \end{proof}
    \item % 8
        根据推论 10.4.1, 则有
        \[
            \iint\limits_{x^2+y^2\leq r^2}f(x,y)\,\mathrm{d}x\mathrm{d}y = f(\boldsymbol{\xi})\pi r^2,   
        \]
        其中 $\boldsymbol{\xi} \in \{(x,y) : x^2+y^2 \leq r^2\}$. 因此
        \[
            \lim_{r\to0}\frac{1}{\pi r^2} \iint\limits_{x^2+y^2\leq r^2}f(x,y)\,\mathrm{d}x\mathrm{d}y = \lim_{r\to0}f(\boldsymbol{\xi}) = f(\boldsymbol{\xi}).   
        \]
\end{enumerate}
% \end{document}

            \section{二重积分换元}
                % @author Shuning Zhang
% @date 2019-03-14
\documentclass[a4paper, 11pt]{ctexart}
\usepackage{amsfonts, amsmath, amssymb, amsthm}
\usepackage{color}
\usepackage{enumerate}
\usepackage[bottom=2cm, left=2.5cm, right=2.5cm, top=2cm]{geometry}
\usepackage{multicol}
\begin{document}
\begin{enumerate}
    \item % 1
        \begin{multicols}{2}
            \begin{enumerate}[(1)]
                \item % 1.1
                    $0$;
                \item % 1.2
            \end{enumerate}
        \end{multicols}
    \item % 2
    \item % 3
        \begin{proof}
            令 $x+y=u$, $-x+y=v$, 那么 $-1 \leq u \leq 1$, $-1 \leq v \leq 1$, 而且
            \[
                \left|\frac{\partial(x,y)}{\partial(u,v)}\right| = \frac12.    
            \]
            因此
            \[
                \iint\limits_{|x|+|y|\leq1}f(x+y)\,\mathrm{d}x\mathrm{d}y = \frac12\int_{-1}^{1}f(u)\,\mathrm{d}u\int_{-1}^1\,\mathrm{d}v = \int_{-1}^{1}f(u)\,\mathrm{d}u.   
            \]
            再把积分变元 $u$ 换为 $t$ 即可.
        \end{proof}
    \item % 4
        \begin{proof}
            令 $xy=u$, $y/x=v$, 那么 $1\leq u\leq 2$, $1 \leq v \leq 4$, 而且
            \[
                \left|\frac{\partial(x,y)}{\partial(u,v)}\right| = \frac{1}{2v}.    
            \]
            故
            \[
                \iint\limits_{D}f(xy)\,\mathrm{d}x\mathrm{d}y = \frac12\int_1^2f(u)\,\mathrm{d}u\int_1^4\frac1v\,\mathrm{d}v = \ln2\int_1^2f(u)\,\mathrm{d}u.    
            \]
            再把积分变元 $u$ 换为 $t$ 即可.
        \end{proof}
    \item % 5
        \begin{multicols}{2}
            \begin{enumerate}[(1)]
                \item % 5.1
                    $-6\pi$;
                \item % 5.2
                \item % 5.3
                \item % 5.4
            \end{enumerate}
        \end{multicols}
    \item % 6
        略.
    \item % 7
        $\dfrac{\pi}{6}ab$.
\end{enumerate}
\end{document}

        \chapter{曲线积分}
            \section{第一型曲线积分}
                % % @author Shuning Zhang
% % @date 2020-03-21
% \documentclass[a4paper, 11pt]{ctexart}
% \usepackage{amsfonts, amsmath, amssymb, amsthm}
% \usepackage{color}
% \usepackage{enumerate}
% \usepackage[bottom=2cm, left=2.5cm, right=2.5cm, top=2cm]{geometry}
% \usepackage{multicol}
% \begin{document}
\begin{enumerate}
    \item % 1
        \begin{enumerate}[(1)]
            \item % 1.1
                $2\pi a^{2n+1}$;
            \item % 1.2
                $1 + \sqrt{2}$;
            \item % 1.3
                $\dfrac{\sqrt{(4\pi^2+2)^3} - 2\sqrt{2}}{3}$;
            \item % 1.4
                $\dfrac{2\pi a^3}{3}$;
            \item % 1.5
                $\dfrac{256a^3}{15}$.
        \end{enumerate}
\end{enumerate}
% \end{document}

            \section{第二型曲线积分}
                % % @author Shuning Zhang
% % @date 2020-03-22
% \documentclass[a4paper, 11pt]{ctexart}
% \usepackage{amsfonts, amsmath, amssymb, amsthm}
% \usepackage{color}
% \usepackage{enumerate}
% \usepackage[bottom=2cm, left=2.5cm, right=2.5cm, top=2cm]{geometry}
% \usepackage{multicol}
% \begin{document}
\begin{enumerate}
    \item % 1
        \begin{enumerate}[(1)]
            \item % 1.1
                $2\pi$;
            \item % 1.2
                $0$;
            \item % 1.3
                $-\dfrac{14}{15}$;
            \item % 1.4
                $\dfrac{1}{4}$;
            \item % 1.5
                $24$.
        \end{enumerate}
    \item % 2
        {\color{red}remained}
    \item % 3
        \begin{enumerate}[(1)]
            \item % 3.1
                $\dfrac{91}{120}$;
            \item % 3.2
                $0$.
        \end{enumerate}
    \item % 4
        $-4$.
    \item % 5
        \begin{proof}
            根据第一型和第二型曲线积分的关系可知
            \[
                \int_{\Gamma}\boldsymbol{F} \cdot \mathrm{d}\boldsymbol{r} = \int_{\Gamma} \boldsymbol{F}\cdot \boldsymbol{t}\,\mathrm{d}s.    
            \]
            其中 $\boldsymbol{F}\cdot\boldsymbol{t} = \|\boldsymbol{F}\|\|\boldsymbol{t}\|\cos\theta = \|\boldsymbol{F}\|\cos\theta$. 由此得出
            \[
                -\|\boldsymbol{F}\| \leq \boldsymbol{F}\cdot\boldsymbol{t} \leq \|\boldsymbol{F}\|,    
            \]
            进一步, 有
            \[
                -\int_{\Gamma}\|\boldsymbol{F}\|\,\mathrm{d}s \leq \int_{\Gamma}\boldsymbol{F}\cdot\boldsymbol{t}\,\mathrm{d}s \leq \int_{\Gamma}\|\boldsymbol{F}\|\,\mathrm{d}s.   
            \]
            改为绝对值的形式, 则有
            \[
                \left|\int_{\Gamma}\boldsymbol{F}\cdot\boldsymbol{t}\,\mathrm{d}s\right| \leq \int_{\Gamma}\|\boldsymbol{F}\|\,\mathrm{d}s.   
            \]
            这正是
            \[
                \left|\int_{\Gamma}\boldsymbol{F} \cdot \mathrm{d}\boldsymbol{r}\right| \leq \int_{\Gamma}\|\boldsymbol{F}\|\,\mathrm{d}s. \qedhere  
            \]
            
        \end{proof}
\end{enumerate}
% \end{document}

            \section{Green 公式}
                % % @author Shuning Zhang
% % @date 2020-03-23
% \documentclass[a4paper, 11pt]{ctexart}
% \usepackage{amsfonts, amsmath, amssymb, amsthm}
% \usepackage{color}
% \usepackage{enumerate}
% \usepackage[bottom=2cm, left=2.5cm, right=2.5cm, top=2cm]{geometry}
% \usepackage{multicol}
% \begin{document}
\begin{enumerate}
    \item % 1
        \begin{enumerate}[(1)]
            \item % 1.1
                $\dfrac{\pi}{2}a^4$;
            \item % 1.2
                $-2\pi ab$;
            \item % 1.3
                $0$.
        \end{enumerate}
    \item % 2
        {\color{red}remained}
    \item % 3
        \begin{proof}
            \begin{align*}
                A &= \frac12\int_\alpha^\beta(\varphi\psi' - \psi\varphi')\,\mathrm{d}t \\
                &= \frac12\int\limits_{\partial{\Omega}}-y\,\mathrm{d}x + x\,\mathrm{d}y \\
                &= \iint\limits_{\Omega}\,\mathrm{d}x\mathrm{d}y.  \qedhere 
            \end{align*}
        \end{proof}
    \item % 4
        提示: 运用定理 11.3.2 证明 $\dfrac{\partial{Q}}{\partial{x}} = \dfrac{\partial{P}}{\partial{y}}$ 即可.
    \item % 5
        \begin{proof}
            考虑将 $\boldsymbol{a}$ 作为 $x$ 轴的单位向量, 那么 $\cos(\boldsymbol{n}, \boldsymbol{a})$ 正是 $\boldsymbol{n}$ 与 $x$ 轴的夹角.
            再根据第一型曲线积分的 Green 公式, 则有
            \[
                \int_{\Gamma}1 \cdot \cos(\boldsymbol{n}, \boldsymbol{a})\,\mathrm{d}s = \iint\limits_{D}0\,\mathrm{d}x\mathrm{d}y = 0.   
            \]
            将 $1$ 看作 $P(x, y)$.
        \end{proof}
    \item % 6
        令 $P(x, y) = x$, $Q(x, y) = y$, 那么
        \[
            \frac{\partial{P}}{\partial{x}} = 1,\ \frac{\partial{Q}}{\partial{y}} = 1.    
        \]
        根据第一型曲线积分的 Green 公式, 则有
        \begin{align*}
            \int_{\Gamma}(P\cos(\boldsymbol{n}, \boldsymbol{i}) + Q\cos(\boldsymbol{n}, \boldsymbol{j}))\,\mathrm{d}s &= \iint\limits_{D}\left(\frac{\partial{P}}{\partial{x}} + \frac{\partial{Q}}{\partial{y}}\right)\,\mathrm{d}x\mathrm{d}y \\
            &= 2\iint\limits_{D}\,\mathrm{d}x\mathrm{d}y \\
            &= 2\sigma(D).   
        \end{align*}
        其中 $D$ 是由 $\Gamma$ 围成的区域, $\sigma(D)$ 表示 $D$ 的面积.
    \item % 7
        \begin{enumerate}[(1)]
            \item % 7.1
                $\dfrac{13}{3}$;
            \item % 7.2
                $\pi$.
        \end{enumerate}
    \item % 8
        {\color{red}remained}
\end{enumerate}
% \end{document}

            \section{等周问题}
        \chapter{曲面积分}
        \chapter{场的数学}
        \chapter{数项级数}
            \section{无穷级数的基本性质}
                % @author Shuning Zhang
% @date 2019-02-02
\documentclass[a4paper, 11pt]{ctexart}
\usepackage{amsfonts, amsmath, amssymb}
\usepackage{enumerate}
\usepackage[bottom=2cm, left=2.5cm, right=2.5cm, top=2cm]{geometry}
\usepackage{multicol}
\begin{document}
\begin{enumerate}
    \item % 1
    \item % 2
    \item % 3
    \item % 4
    \item % 5
    \item % 6
    \item % 7
    \item % 8
\end{enumerate}
\end{document}
\end{document}
