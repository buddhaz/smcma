% % @author Shuning Zhang
% % @date 2020-04-06
% \documentclass[a4paper, 11pt]{ctexart}
% \usepackage{amsfonts, amsmath, amssymb, amsthm}
% \usepackage{color}
% \usepackage{enumerate}
% \usepackage[bottom=2cm, left=2.5cm, right=2.5cm, top=2cm]{geometry}
% \usepackage{multicol}
% \begin{document}
\begin{enumerate}
    \item % 1
        不能. 举例: $a_n = -1/n$, $b_n = 1/n^2$.
    \item % 2
        略.
    \item % 3
        \begin{proof}
            根据定理 14.2.3, 则有
            \[
                \lim_{n\to\infty}\frac{a_n^2}{a_n} = \lim_{n\to\infty}a_n = 0.    
            \]
            故 $\sum a_n^2$ 收敛. 举例: $a_n^2 = 1/n^2$.
        \end{proof}
    \item % 4
        \begin{proof}
            因为
            \begin{align*}
                |a_nb_n| = \sqrt{a_n^2b_n^2} \leq \frac{a_n^2 + b_n^2}{2},
            \end{align*}
            故 $\sum|a_nb_n|$ 收敛.
            又因
            \[
                (a_n + b_n)^2 = a_n^2 + b_n^2 + 2a_nb_n \leq a_n^2 + b_n^2 + 2|a_nb_n|,    
            \]
            而 $\sum|a_nb_n|$ 收敛, 故 $\sum(a_n + b_n)^2$ 收敛.
        \end{proof}
    \item % 5
        \begin{proof}
            因为
            \[
                \lim_{n\to\infty}\frac{a_n}{\sqrt{a_na_{n+1}}} = \lim_{n\to\infty}\sqrt{\frac{a_n}{a_{n+1}}} = 1,   
            \]
            故 $\sum a_n$ 和 $\sum\sqrt{a_na_{n+1}}$ 同敛散, 因此 $\sum\sqrt{a_na_{n+1}}$ 收敛. 举例: 当
            \[
                a_n =
                \begin{cases}
                    1 & n\ \text{为奇数}, \\
                    0 & n\ \text{为偶数}.
                \end{cases}    
            \]
            那么 $\sum\sqrt{a_na_{n+1}} = 0$, 但显然 $\sum a_n$ 不收敛.
        \end{proof}
    \item % 6
        \begin{proof}
            显然有不等式
            \[
                \ln(n!) = \ln1 + \ln2 + \cdots + \ln{n} \leq n\ln{n},    
            \]
            因此
            \[
                \frac{1}{n\ln{n}} < \frac{1}{\ln{n!}},    
            \]
            又 $\sum\dfrac{1}{n\ln{n}}$ 可由 Cauchy 积分判别法知道是发散的, 故 $\sum\dfrac{1}{\ln{n!}}$ 也是发散的.
        \end{proof}
    \item % 7
        \begin{proof}
            因为
            \[
                \int_3^{+\infty}\frac{1}{x\ln x\ln\ln x}\,\mathrm{d}x = \int_3^{+\infty}\frac{1}{\ln x\ln\ln x}\,\mathrm{d}(\ln x) = \int_3^{+\infty}\frac{1}{\ln\ln x}\,\mathrm{d}(\ln\ln x)    
            \]
            发散, 故由 Cauchy 积分判别法可知 $\sum\limits_{n=3}^\infty\dfrac{1}{n\ln n\ln\ln n}$ 发散.
        \end{proof}
    \item % 8
        {\color{red}remained}
    \item % 9
        \begin{proof}
            因为
            \[
                n^{-(1+\delta)/2}\sqrt{a_n} = \sqrt{\frac{1}{n^{1+\delta}}\cdot a_n} \leq \frac{\frac{1}{n^{1+\delta}} + a_n}{2},    
            \]
            而 $\sum\dfrac{1}{n^{1+\delta}}$ 和 $\sum a_n$ 收敛, 故 $\sum n^{-(1+\delta)/2}\sqrt{a_n}$ 收敛.

            当 $\delta = 0$ 时, 则有
            \[
                n^{-1/2}\sqrt{a_n} = \sqrt{\frac{a_n}{n}} = \sqrt{\frac1n\cdot a_n} \geq \frac{2}{n + \frac{1}{a_n}},    
            \]
            又因 $\sum a_n$ 收敛, 有极限 $\lim a_n = 0$, 故数列 $\{a_n\}$ 有界, 即 $a_n \leq M$, 那么
            \[
                \frac{2}{n + \frac{1}{a_n}} \geq \frac{2}{n + \frac{1}{M}}.    
            \]
            $\sum\dfrac{2}{n + \frac{1}{M}}$ 显然与 $\sum\dfrac1n$ 同敛散. 故 $\sum\sqrt{\dfrac{a_n}{n}}$ 发散.
        \end{proof}
    \item % 10
        \begin{enumerate}[(1)]
            \item % 10.1
                \begin{proof}
                    对不等式进行化简, 则有
                    \begin{align*}
                        \left(\ln\frac{1}{a_n}\right)(\ln n)^{-1} \geq 1 + \sigma &\Leftrightarrow \ln\frac{1}{a_n} \geq (1+\sigma)\ln n \\
                        &\Leftrightarrow -\ln a_n \geq \ln n^{1+\sigma} \\
                        &\Leftrightarrow \ln a_n \leq -\ln n^{1+\sigma} = \ln\frac{1}{n^{1+\sigma}} \\
                        &\Leftrightarrow a_n \leq \frac{1}{n^{1+\sigma}},
                    \end{align*}
                    又 $\sum\dfrac{1}{n^{1+\sigma}}$ 收敛, 故 $\sum a_n$ 收敛.
                \end{proof}
            \item % 10.2
                \begin{proof}
                    由 (1) 的推导过程可得到
                    \[
                        a_n \geq \frac1n,    
                    \]
                    故 $\sum a_n$ 发散.
                \end{proof}
            \item % 10.3
                略.
        \end{enumerate}
    \item % 11
        \begin{proof}
            因为 $\sum a_n$ 发散, 若 $\lim\limits_{n\to\infty}a_n = +\infty$, 那么
            \[
                \lim_{n\to\infty}\frac{a_n}{1+a_n} = 1,   
            \]
            由收敛级数的必要条件可知 $\sum\dfrac{a_n}{1+a_n}$ 发散. 若 $\lim\limits_{n\to\infty}a_n = l$, $l$ 为一有限数, 那么
            \[
                \lim_{n\to\infty}\frac{\frac{a_n}{1+a_n}}{a_n} = \lim_{n\to\infty}\frac{1}{1+a_n} = \frac{1}{1+l},
            \]
            故 $\sum a_n$ 与 $\sum \dfrac{a_n}{1+a_n}$ 同敛散, 因此 $\sum\dfrac{a_n}{1+a_n}$ 发散.

            对第二个级数, 考虑其与 $\sum\dfrac{1}{n^2}$ 的比的极限形式. 则有
            \[
                \lim_{n\to\infty}\frac{1}{1 + \frac{1}{n^2a_n}},    
            \]
            不论此极限趋于 $0$ 还是有限数都可以表明 $\sum\dfrac{a_n}{1 + n^2a_n}$ 收敛.
        \end{proof}
    \item % 12
        {\color{red}remained}
        \begin{proof}
            先证明 $\sum{a_n^{\alpha+\beta}}$ 收敛, 因为
            \[
                \lim_{n\to\infty}\frac{a_n^{\alpha+\beta}}{a_n} = \lim_{n\to\infty}a_n^{\alpha+\beta-1} = 0.    
            \]
            再证明 $\sum\dfrac{a_n^{\alpha+\beta}}{n^{\alpha+\beta}}$ 收敛, 因为
            \[
                \left.\lim_{n\to\infty}\frac{a_n^{\alpha+\beta}}{n^{\alpha+\beta}} \right/ \frac{1}{n^{\alpha+\beta}} = \lim_{n\to\infty}a_n^{\alpha+\beta} = 0. \qedhere    
            \]
        \end{proof}
\end{enumerate}
% \end{document}
