% % @author Shuning Zhang
% % @date 2019-02-14
% \documentclass[a4paper, 11pt]{ctexart}
% \usepackage{amsfonts, amsmath, amssymb}
% \usepackage{color}
% \usepackage{enumerate}
% \usepackage[bottom=2cm, left=2.5cm, right=2.5cm, top=2cm]{geometry}
% \usepackage{multicol}
% \newcommand{\diff}{\mathrm{d}}
% \begin{document}
\begin{enumerate}
    \item % 1
        {\heiti 证明}\quad 先证明 $f(\boldsymbol{x})$ 在 $(0, 0)$ 各个方向 $(\cos\theta, \sin\theta)$ 的方向导数都存在. 根据方向导数的定义, 则有
        \[
            \lim_{t \to 0}\frac{f(\boldsymbol{0} + t\boldsymbol{u}) - f(\boldsymbol{0})}{t} = \lim_{t\to 0}\frac{t\sin\theta\cos^2\theta}{\sin^2\theta + t^2\cos^2\theta} = 0,    
        \]
        即 $f(\boldsymbol{x})$ 在 $(0, 0)$ 处各方向导数为 0.

        再证明 $f(\boldsymbol{x})$ 在 $(0, 0)$ 处不可微, 证明 $f(\boldsymbol{x})$ 不连续即可. 考虑 $\displaystyle{\left(\frac1i, \frac1i\right)}$ 和 $\displaystyle{\left(\frac1i, \frac{1}{i^2}\right)}$ 两个点列.
        则有
        \begin{gather*}
            \lim_{i\to\infty}\frac{\frac{1}{i^3}}{\frac{1}{i^4} + \frac{1}{i^2}} = 0, \\
            \lim_{i\to\infty}\frac{\frac{1}{i^4}}{\frac{1}{i^4} + \frac{1}{i^4}} = \frac12.   
        \end{gather*}
        因此 $f(\boldsymbol{x})$ 在 $(0, 0)$ 不连续, 故不可微.
    \item % 2
        {\heiti 证明}\quad 根据微分的定义, 那么
        \[
            f(\boldsymbol{0} + \boldsymbol{h}) - f(\boldsymbol{x}) = \sqrt{|h_1h_2|},    
        \] 那么
        \[
            \lim_{\|\boldsymbol{h}\|\to0}\frac{\sqrt{|h_1h_2|}}{\|\boldsymbol{h}\|} = \lim_{(h_1,h_2)\to(0,0)}\sqrt{\frac{|h_1h_2|}{x^2+y^2}},   
        \]
        由 8.2 节的例 2 可知上述极限不存在, 因此 $\sqrt{|h_1h_2|} \neq o(\|\boldsymbol{h}\|)\ (\|\boldsymbol{h}\|\to0)$, 所以 $f(\boldsymbol{x})$ 在 $(0,0)$ 不可微.
    \item % 3
        {\heiti 证明}\quad 任取 $\boldsymbol{x} = (x_1, x_2), \boldsymbol{h} = (h_1, h_2) \in \mathrm{R}^2$, 则有
        \begin{align*}
            f(\boldsymbol{x} + \boldsymbol{h}) - f(\boldsymbol{x}) &= (x_1 + h_1)(x_2 + h_2) - x_1x_2 \\
            &= x_1h_2 + x_2h_1 + h_1h_2 \\
            &= \lambda_1h_1 + \lambda_2h_2 + h_1h_2 \\
            &= \sum_{i=1}^2\lambda_ih_i + h_1h_2,   
        \end{align*}
        其中 $\lambda_1 = x_2$, $\lambda_2 = x_1$. 只需证明 $h_1h_2 = o(\|\boldsymbol{h}\|)\ (\|\boldsymbol{h}\|\to0)$ 即可, 即证明如下极限:
        \[
            \lim_{\|\boldsymbol{h}\|\to0}\frac{h_1h_2}{\|\boldsymbol{h}\|} = \lim_{(h_1, h_2) \to (0, 0)}\frac{h_1h_2}{\sqrt{h_1^2 + h_2^2}} = 0.    
        \]
        对任意的 $\varepsilon > 0$, 取 $\delta = \varepsilon$, 当 $0 < |\|\boldsymbol{h}\| - 0| = \sqrt{h_1^2 + h_2^2} < \delta$ 时, 则有
        \[
            \left| \frac{h_1h_2}{\sqrt{h_1^2+h_2^2}} - 0 \right| = \frac{|h_1||h_2|}{\sqrt{h_1^2+h_2^2}} \leqslant \sqrt{h_1^2 + h_2^2} < \delta = \varepsilon.   
        \]
        因此便有
        \[
            f(\boldsymbol{x} + \boldsymbol{h}) - f(\boldsymbol{x}) = \sum_{i=1}^2\lambda_ih_i + o(\|\boldsymbol{h}\|)\quad (\|\boldsymbol{h}\| \to 0).
        \]
        这样便证明了 $f(x, y) = xy$ 在 $\mathrm{R}^2$ 上每一点可微.
    \item % 4
        \begin{enumerate}[(1)]
            \item % 4.1
                $\diff{f(1, 2)} = 6\diff{x} - 2\diff{y}$;
            \item % 4.2
                $\diff{f(1, 2, 1)} = (1/2 + e^3\sin1)\diff x + (1/2 + e^3\sin1)\diff y + (-1/2 + e^3\cos1)\diff z$;
            \item % 4.3
                $\diff f(t_1, t_2, \cdots, t_n) = \displaystyle{\sum_{i=1}^n}\frac{t_i}{\sqrt{t_1^2 + t_2^2 + \cdots + t_n^2}}\diff x_i$;
            \item % 4.4
                $\diff f(x_1, x_2, \cdots, x_n) = \cos(x_1 + x_2^2 + \cdots + x_n^n)\displaystyle{\sum_{i=1}^nix_i^{i-1}\diff x_i}$.
        \end{enumerate}
    \item % 5
        \begin{enumerate}[(1)]
            \item % 5.1
                $(2xy^3, 3x^2y^2)$;
            \item % 5.2
                $(2xy\sin{yz}, x^2(\sin{yz}+yz\cos{yz}), x^2y^2\cos{yz})$;
            \item % 5.3
                $\displaystyle{
                    \left(
                        \cos(y-3z) + \frac{y}{\sqrt{1-x^2y^2}}, -x\sin(y-3z) + \frac{x}{\sqrt{1-x^2y^2}}, 3x\sin(y-3z)    
                    \right)
                }$;
            \item % 5.4
                $\dfrac{1}{\|\boldsymbol{x}\|}(x_1, x_2, \cdots, x_n)$.
        \end{enumerate}
    \item % 6
        {\heiti 证明}\quad 先证明 $f(\boldsymbol{x})$ 在 $\boldsymbol{0} = (0,0)$ 处可微. 根据微分的定义, 那么
        \begin{align*}
            f(\boldsymbol{0} - \boldsymbol{h}) - f(\boldsymbol{0}) &= (h_1^2 + h_2^2)\sin\frac{1}{h_1^2 + h_2^2} \\
            &= 0h_1 + 0h_2 + (h_1^2 + h_2^2)\sin\frac{1}{h_1^2 + h_2^2} \\
            &= \sum_{i=1}^2\lambda_ih_i + (h_1^2 + h_2^2)\sin\frac{1}{h_1^2 + h_2^2},   
        \end{align*}
        其中 $\lambda_1 = \lambda_2 = 0$. 只需证明 $(h_1^2 + h_2^2)\sin\frac{1}{h_1^2 + h_2^2} = o(\|\boldsymbol{h}\|)\ (\|\boldsymbol{h}\|\to0)$ 即可.
        对任意的 $\varepsilon > 0$, 取 $\delta = \varepsilon$, 当 $0 < |\|\boldsymbol{h}\| - 0| = \|\boldsymbol{h}\| = \sqrt{h_1^2 + h_2^2} < \delta = \varepsilon$, 则有
        \begin{align*}
            \left| \frac{(h_1^2 + h_2^2)\sin\frac{1}{h_1^2 + h_2^2}}{\|\boldsymbol{h}\|} - 0 \right| &= \frac{(h_1^2 + h_2^2)|\sin\frac{1}{h_1^2 + h_2^2}|}{\sqrt{h_1^2 + h_2^2}} \\
            &\leqslant \sqrt{h_1^2 + h_2^2} < \delta = \varepsilon.    
        \end{align*}
        这样便证明 $(h_1^2 + h_2^2)\sin\frac{1}{h_1^2 + h_2^2} = o(\|\boldsymbol{h}\|)\ (\|\boldsymbol{h}\|\to0)$.

        证明 $\frac{\partial f}{\partial x}$ 在 $(0, 0)$ 处不连续即可, 另一个同理. 先求出偏导数, 则有
        \[
            \frac{\partial f}{\partial x} = 2x\left(\sin\frac{1}{x^2+y^2} - \frac{1}{x^2+y^2}\cos\frac{1}{x^2+y^2}\right).   
        \]
        考虑点列 $\displaystyle{\left(\frac{1}{\sqrt{2k\pi}}, \frac{1}{\sqrt{2k\pi}}\right) \to (0,0)\ (k\to\infty)}$, 那么
        \[
            \lim_{k\to\infty} \frac{\partial f}{\partial x}\left(\frac{1}{\sqrt{2k\pi}}, \frac{1}{\sqrt{2k\pi}}\right) = \lim_{k\to\infty}\frac{2}{\sqrt{2k\pi}}((-1)^{k+1}k\pi) = \lim_{k\to\infty}(-1)^{k+1}\sqrt{2k\pi} = +\infty.   
        \]
        因此 $\frac{\partial f}{\partial x}$ 在 $(0, 0)$ 处的极限不存在, 故不连续.
\end{enumerate}
% \end{document}
