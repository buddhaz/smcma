% @author Shuning Zhang
% @date 2019-02-14
\documentclass[a4paper, 11pt]{ctexart}
\usepackage{amsfonts, amsmath, amssymb}
\usepackage{color}
\usepackage{enumerate}
\usepackage[bottom=2cm, left=2.5cm, right=2.5cm, top=2cm]{geometry}
\usepackage{multicol}
\newcommand{\diff}{\mathrm{d}}
\begin{document}
\begin{enumerate}
    \item % 1
    \item % 2
    \item % 3
        {\heiti 证明}\quad 任取 $\boldsymbol{x} = (x_1, x_2), \boldsymbol{h} = (h_1, h_2) \in \mathrm{R}^2$, 则有
        \begin{align*}
            f(\boldsymbol{x} + \boldsymbol{h}) - f(\boldsymbol{x}) &= (x_1 + h_1)(x_2 + h_2) - x_1x_2 \\
            &= x_1h_2 + x_2h_1 + h_1h_2 \\
            &= \lambda_1h_1 + \lambda_2h_2 + h_1h_2 \\
            &= \sum_{i=1}^2\lambda_ih_i + h_1h_2,   
        \end{align*}
        其中 $\lambda_1 = x_2$, $\lambda_2 = x_1$. 只需证明 $h_1h_2 = o(\|\boldsymbol{h}\|)\ (\|\boldsymbol{h}\|\to0)$ 即可, 即证明如下极限:
        \[
            \lim_{\|\boldsymbol{h}\|\to0}\frac{h_1h_2}{\|\boldsymbol{h}\|} = \lim_{(h_1, h_2) \to (0, 0)}\frac{h_1h_2}{\sqrt{h_1^2 + h_2^2}} = 0.    
        \]
        对任意的 $\varepsilon > 0$, 取 $\delta = \varepsilon$, 当 $0 < |\|\boldsymbol{h}\| - 0| = \sqrt{h_1^2 + h_2^2} < \delta$ 时, 则有
        \[
            \left| \frac{h_1h_2}{\sqrt{h_1^2+h_2^2}} - 0 \right| = \frac{|h_1||h_2|}{\sqrt{h_1^2+h_2^2}} \leqslant \sqrt{h_1^2 + h_2^2} < \delta = \varepsilon.   
        \]
        因此便有
        \[
            f(\boldsymbol{x} + \boldsymbol{h}) - f(\boldsymbol{x}) = \sum_{i=1}^2\lambda_ih_i + o(\|\boldsymbol{h}\|)\quad (\|\boldsymbol{h}\| \to 0).
        \]
        这样便证明了 $f(x, y) = xy$ 在 $\mathrm{R}^2$ 上每一点可微.
    \item % 4
        \begin{enumerate}[(1)]
            \item % 4.1
                $\diff{f(1, 2)} = 6\diff{x} - 2\diff{y}$;
            \item % 4.2
                $\diff{f(1, 2, 1)} = (1/2 + e^3\sin1)\diff x + (1/2 + e^3\sin1)\diff y + (-1/2 + e^3\cos1)\diff z$;
            \item % 4.3
                $\diff f(t_1, t_2, \cdots, t_n) = \displaystyle{\sum_{i=1}^n}\frac{t_i}{\sqrt{t_1^2 + t_2^2 + \cdots + t_n^2}}\diff x_i$;
            \item % 4.4
                $\diff f(x_1, x_2, \cdots, x_n) = \cos(x_1 + x_2^2 + \cdots + x_n^n)\displaystyle{\sum_{i=1}^nix_i^{i-1}\diff x_i}$.
        \end{enumerate}
    \item % 5
        \begin{enumerate}[(1)]
            \item % 5.1
                $(2xy^3, 3x^2y^2)$;
            \item % 5.2
                $(2xy\sin{yz}, x^2(\sin{yz}+yz\cos{yz}), x^2y^2\cos{yz})$;
            \item % 5.3
                $\displaystyle{
                    \left(
                        \cos(y-3z) + \frac{y}{\sqrt{1-x^2y^2}}, -x\sin(y-3z) + \frac{x}{\sqrt{1-x^2y^2}}, 3x\sin(y-3z)    
                    \right)
                }$;
            \item % 5.4
                $\dfrac{1}{\|\boldsymbol{x}\|}(x_1, x_2, \cdots, x_n)$.
        \end{enumerate}
    \item % 6
\end{enumerate}
\end{document}
