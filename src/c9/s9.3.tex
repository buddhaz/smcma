% @author Shuning Zhang
% @date 2019-02-16
\documentclass[a4paper, 11pt]{ctexart}
\usepackage{amsfonts, amsmath, amssymb, amsthm}
\usepackage{color}
\usepackage{enumerate}
\usepackage[bottom=2cm, left=2.5cm, right=2.5cm, top=2cm]{geometry}
\usepackage{multicol}
\begin{document}
\begin{enumerate}
    \item % 1
        \begin{enumerate}[(1)]
            \item % 1.1
                \[
                    \begin{bmatrix}
                        y^2-6x & 2xy \\
                        3 & -10y
                    \end{bmatrix};    
                \]
            \item % 1.2
                \[
                    \begin{bmatrix}
                        yz^2 & xz^2-8y & 2xyz \\
                        3y^2 & 6xy - 2yz & -y^2
                    \end{bmatrix};
                \]
            \item % 1.3
                \[
                    \begin{bmatrix}
                        e^x(\cos{xy} - y\sin{xy}) & -xe^x\sin{xy} \\
                        e^x(\sin{xy} + y\cos{xy}) & xe^x\cos{xy}
                    \end{bmatrix}.    
                \]
        \end{enumerate}
    \item % 2
        \begin{enumerate}[(1)]
            \item % 2.1
                \[
                    \begin{bmatrix}
                        \cos\theta & -r\sin\theta \\
                        \sin\theta & r\cos\theta
                    \end{bmatrix};    
                \]
            \item % 2.2
                \[
                    \begin{bmatrix}
                        \cos\theta & -r\sin\theta & 0 \\
                        \sin\theta & r\cos\theta & 0 \\
                        0 & 0 & 1
                    \end{bmatrix};    
                \]
            \item % 2.3
                \[
                    \begin{bmatrix}
                        \sin\theta\cos\varphi & r\cos\theta\cos\varphi & -r\sin\theta\sin\varphi \\
                        \sin\theta\sin\varphi & r\cos\theta\sin\varphi & r\sin\theta\cos\varphi \\
                        \cos\theta & -r\sin\theta & 0
                    \end{bmatrix}.    
                \]
        \end{enumerate}
    \item % 3
        略.
    \item % 4
        \begin{proof}
            \[
                \boldsymbol{f}(t) = \boldsymbol{x} = (x_1, x_2, \cdots, x_n) =
                \begin{cases}
                    f_1(t) = x_1, \\
                    f_2(t) = x_2, \\
                    \cdots, \\
                    f_n(t) = x_n
                \end{cases}    
            \]
        \end{proof}
    \item % 5
        $\boldsymbol{f}(x, y, z) = \left(\dfrac12\alpha^2(x), \dfrac12\beta^2(y), \dfrac12\gamma^2(z)\right)$.
    \item % 6
        \begin{enumerate}[(1)]
            \item % 6.1
                \begin{proof}
                    $\boldsymbol{f}(\boldsymbol{0}) = \boldsymbol{f}(\lambda\boldsymbol{x} - \lambda\boldsymbol{x}) = \lambda\boldsymbol{f}(\boldsymbol{x}) - \lambda\boldsymbol{f}(\boldsymbol{x}) = \boldsymbol{0}$.
                \end{proof}
            \item % 6.2
                \begin{proof}
                    $\boldsymbol{f}(-\boldsymbol{x}) = \boldsymbol{f}(1 \cdot \boldsymbol{0} + (-1)\boldsymbol{x}) = 1\cdot\boldsymbol{f}(\boldsymbol{0}) + (-1)\boldsymbol{f}(\boldsymbol{x}) = -\boldsymbol{f}(\boldsymbol{x})$.
                \end{proof}
            \item % 6.3
                \begin{proof}
                    对任意 $\boldsymbol{x} = (x_1, x_2, \cdots, x_n) \in \mathrm{R}^n$, 有
                    \begin{align*}
                        \boldsymbol{f}(\boldsymbol{x}) &= \boldsymbol{f}(x_1, x_2, \cdots, x_n) \\
                        &= \boldsymbol{f}(x_1\boldsymbol{e}_1 + x_2\boldsymbol{e}_2 + \cdots + x_n\boldsymbol{e}_n) \\
                        &= x_1\boldsymbol{f}(\boldsymbol{e}_1) + \boldsymbol{f}(x_2\boldsymbol{e}_2 + \cdots + x_n\boldsymbol{e}_n) \\
                        &= x_1\boldsymbol{f}(\boldsymbol{e}_1) + x_2\boldsymbol{f}(\boldsymbol{e}_2) + \cdots + x_n\boldsymbol{f}(\boldsymbol{e}_n). \qedhere
                    \end{align*}
                \end{proof}
        \end{enumerate}
    \item % 7
    \item % 8
        \begin{proof}
            因为 $\boldsymbol{E}(\lambda_1\boldsymbol{x}_1 + \lambda_2\boldsymbol{x}_2) = \lambda_1\boldsymbol{x}_1 + \lambda_2\boldsymbol{x}_2 = \lambda_1\boldsymbol{E}(\boldsymbol{x}_1) + \lambda_2\boldsymbol{E}(\boldsymbol{x}_2)$, 故 $\boldsymbol{E}(\boldsymbol{x})$ 是一个线性映射.
            \[
                \boldsymbol{E}(\boldsymbol{x}) = \boldsymbol{x} = (x_1, x_2, \cdots, x_n) =
                \begin{cases}
                    E_1(\boldsymbol{x}) = x_1, \\
                    E_2(\boldsymbol{x}) = x_2, \\
                    \cdots, \\
                    E_n(\boldsymbol{x}) = x_n.
                \end{cases}    
            \]
            因此
            \[
                \begin{bmatrix}
                    \frac{\partial{E_1}}{x_1} & \frac{\partial{E_1}}{x_2} & \cdots & \frac{\partial{E_1}}{x_n} \\ 
                    \frac{\partial{E_2}}{x_1} & \frac{\partial{E_2}}{x_2} & \cdots & \frac{\partial{E_2}}{x_n} \\
                    \vdots & \vdots &  & \vdots \\ 
                    \frac{\partial{E_n}}{x_1} & \frac{\partial{E_n}}{x_2} & \cdots & \frac{\partial{E_n}}{x_n} \\ 
                \end{bmatrix}
                =
                \begin{bmatrix}
                    1 & 0 & \cdots & 0 \\
                    0 & 1 & \cdots & 0 \\
                    \vdots & \vdots & & \vdots \\
                    0 & 0 & \cdots & 1
                \end{bmatrix}. \qedhere    
            \]
        \end{proof}
\end{enumerate}
\end{document}
