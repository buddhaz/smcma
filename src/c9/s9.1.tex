% % @author Shuning Zhang
% % @date 2019-02-14
% \documentclass[a4paper, 11pt]{ctexart}
% \usepackage{amsfonts, amsmath, amssymb}
% \usepackage{color}
% \usepackage{enumerate}
% \usepackage[bottom=2cm, left=2.5cm, right=2.5cm, top=2cm]{geometry}
% \usepackage{multicol}
% \begin{document}
\begin{enumerate}
    \item % 1
        \begin{multicols}{2}
            \begin{enumerate}[(1)]
                \item % 1.1
                    $\sqrt{2}$;
                \item % 1.2
                    $2/5$.
            \end{enumerate} 
        \end{multicols}
    \item % 2
        $\left(\dfrac{\sqrt{2}}{2}, \dfrac{\sqrt{2}}{2}\right)$,
        $\left(\dfrac{\sqrt{2}}{2}, -\dfrac{\sqrt{2}}{2}\right)$.
    \item % 3
        $(1,0)$, $(0,1)$, $(-1,0)$, $(0,-1)$.
    \item % 4
        空间中 $\boldsymbol{u} = (\sin\theta\cos\varphi, \sin\theta\sin\varphi, \cos\theta)$, 其中 $0 \leqslant \theta \leqslant \pi$, $0 \leqslant \varphi < 2\pi$.
        设 $\boldsymbol{p}_0 = (x_0, y_0, z_0)$ 是平面上一点, 即 $x_0 + y_0 + z_0 = 0$, 那么
        \[
            \boldsymbol{p}_0 + t\boldsymbol{u} = (x_0 + t\sin\theta\cos\varphi, y_0 + t\sin\theta\sin\varphi, z_0 + t\cos\theta).    
        \]
        进一步, 则有
        \begin{align*}
            f(\boldsymbol{p}_0 + t\boldsymbol{u}) &= |(x_0 + t\sin\theta\cos\varphi) + (y_0 + t\sin\theta\sin\varphi) + (z_0 + t\cos\theta)| \\
            &= |t||\sin\theta\cos\varphi + \sin\theta\sin\varphi + \cos\theta|.    
        \end{align*}
        由方向导数的定义, 则有
        \[
            \lim_{t\to0}\frac{f(\boldsymbol{p}_0 + t\boldsymbol{u}) - f(\boldsymbol{p}_0)}{t}
            =
            \lim_{t\to0}\frac{|t||\sin\theta\cos\varphi + \sin\theta\sin\varphi + \cos\theta|}{t}.    
        \]
        要使上面的极限存在, 那么 $|\sin\theta\cos\varphi + \sin\theta\sin\varphi + \cos\theta| = 0$. 解得
        \[
            \theta = \frac{\pi}{2}, \varphi = \frac{3\pi}{4}\ \text{或}\ \theta = \frac{\pi}{2}, \varphi = \frac{7\pi}{4}.    
        \]
        那么方向则为
        \[
            \left(-\frac{\sqrt{2}}{2}, \frac{\sqrt{2}}{2}, 0\right), \left(\frac{\sqrt{2}}{2}, -\frac{\sqrt{2}}{2}, 0\right).    
        \]
    \item % 5
        \begin{enumerate}[(1)]
            \item % 5.1
                $\dfrac{\partial{f}}{\partial{x}}(0,1) = 2$,
                $\dfrac{\partial{f}}{\partial{y}}(0,1) = 2$,
                $\dfrac{\partial{f}}{\partial{x}}(1,2) = \dfrac{5 + \sqrt{5}}{5}$,
                $\dfrac{\partial{f}}{\partial{y}}(1,2) = \dfrac{5 + 2\sqrt{5}}{5}$;
            \item % 5.2
                $\dfrac{\partial{f}}{\partial{x}}(1,2) = \dfrac23$,
                $\dfrac{\partial{f}}{\partial{y}}(1,2) = \dfrac13$;
            \item % 5.3
                $\dfrac{\partial{f}}{\partial{x}}(1,1) = e^2 + 2\cos{1}$,
                $\dfrac{\partial{f}}{\partial{y}}(1,1) = 2e^2 + \cos{1}$.
        \end{enumerate}
    \item % 6
        \begin{enumerate}[(1)]
            \item % 6.1
                $\dfrac{\partial{z}}{\partial{x}} = y + \dfrac1y$,
                $\dfrac{\partial{z}}{\partial{y}} = x - \dfrac{x}{y^2}$;
            \item % 6.2
                $\dfrac{\partial{z}}{\partial{x}} = \dfrac{2x}{y}\sec^2\dfrac{x^2}{y}$,
                $\dfrac{\partial{z}}{\partial{y}} = -\dfrac{x^2}{y^2}\sec^2\dfrac{x^2}{y}$;
            \item % 6.3
                $\dfrac{\partial{z}}{\partial{x}} = yx^{y-1}$,
                $\dfrac{\partial{z}}{\partial{y}} = x^y\ln{x}$;
            \item % 6.4
                $\dfrac{\partial{z}}{\partial{x}} = \dfrac{1}{x + y^2}$,
                $\dfrac{\partial{z}}{\partial{y}} = \dfrac{2y}{x + y^2}$;
            \item % 6.5
                $\dfrac{\partial{z}}{\partial{x}} = -\dfrac{y}{x^2 + y^2}$,
                $\dfrac{\partial{z}}{\partial{y}} = \dfrac{x}{x^2 + y^2}$;
            \item % 6.6
                $\dfrac{\partial{z}}{\partial{x}} = y\cos{xy}$,
                $\dfrac{\partial{z}}{\partial{y}} = x\cos{xy}$;
            \item % 6.7
                $\dfrac{\partial{u}}{\partial{x}} = (3x^2+y^2+z^2)e^{x(x^2+y^2+z^2)}$,
                $\dfrac{\partial{u}}{\partial{y}} = 2xye^{x(x^2+y^2+z^2)}$,
                $\dfrac{\partial{u}}{\partial{z}} = 2xze^{x(x^2+y^2+z^2)}$;
            \item % 6.8
                $\dfrac{\partial{u}}{\partial{x}}=yze^{xyz}$,
                $\dfrac{\partial{u}}{\partial{y}}=xze^{xyz}$,
                $\dfrac{\partial{u}}{\partial{z}}=xye^{xyz}$;
            \item % 6.9
                $\dfrac{\partial{u}}{\partial{x}} = (yz)x^{yz-1}$,
                $\dfrac{\partial{u}}{\partial{y}} = (z\ln{x})x^{yz}$,
                $\dfrac{\partial{u}}{\partial{z}} = (y\ln{x})x^{yz}$;
            \item % 6.10
                $\dfrac{\partial{u}}{\partial{x}} = \dfrac{1}{x+y^2+z^3}$,
                $\dfrac{\partial{u}}{\partial{y}} = \dfrac{2y}{x+y^2+z^3}$,
                $\dfrac{\partial{u}}{\partial{z}} = \dfrac{3z^2}{x+y^2+z^3}$;
            \item % 6.11
                $\dfrac{\partial{u}}{\partial{x_i}} = \dfrac{1}{x_1 + x_2 + \cdots + x_n}$;
            \item % 6.12
                $\dfrac{\partial{u}}{\partial{x_i}} = \dfrac{2x_i}{\sqrt{1 - (x_1^2 + x_2^2 + \cdots + x_n^2)^2}}$.
        \end{enumerate}
\end{enumerate}
% \end{document}