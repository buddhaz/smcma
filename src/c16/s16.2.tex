% @author Shuning Zhang
% @date 2020-06-06
\documentclass[a4paper, 11pt]{ctexart}
\usepackage{amsfonts, amsmath, amssymb, amsthm}
\usepackage{color}
\usepackage{enumerate}
\usepackage[bottom=2cm, left=2.5cm, right=2.5cm, top=2cm]{geometry}
\usepackage{multicol}
\begin{document}
\begin{enumerate}
    \item % 1
    \item % 2
    \item % 3
        
        {\color{red}wrong}
    \begin{proof}
            因为 $\displaystyle{\int_a^{+\infty}f(t)\,\mathrm{d}t}$ 收敛, 根据 Cauchy 收敛原理, 对任意的 $\varepsilon > 0$, 存在 $X > a$, 当 $x_2 \geq x_1 > X$ 时, 有
            \[
                \int_{x_1}^{x_2}f(t)\,\mathrm{d}t < \varepsilon.    
            \]
            又 $f(x)$ 递减, 故 $f(x_2) \leq f(t) \leq f(x_1)$, 对前一个不等式作积分, 有
            \[
                f(x_2)(x_2 - x_1) = \int_{x_1}^{x_2}f(x_2)\,\mathrm{d}t < \int_{x_1}^{x_2}f(t)\,\mathrm{d}t.
            \]
            又 $f(x_1) \geq f(x_2) \geq 0$, 故
            \[
                f(x_2)(x_2 - x_1) = x_2f(x_2) - x_1f(x_2) \geq x_2f(x_2) - x_1f(x_1),  
            \] 
            因此
            \[
                x_2f(x_2) - x_1f(x_1) < \int_{x_1}^{x_2}f(t)\,\mathrm{d}t < \varepsilon.   
            \]
            这就是 $\lim\limits_{x\to+\infty}xf(x) = 0$ 的 Cauchy 收敛原理的表述.
        \end{proof}
    \item % 4
        \begin{proof}
            由题意可知 $f(x) \geq 0$, 并且有不等式
            \[
                0 \leq \int_a^{+\infty}f(x)\sin^2x\,\mathrm{d}x \leq \int_a^{+\infty}f(x)\,\mathrm{d}x.    
            \]
            若 $\displaystyle{\int_a^{+\infty}f(x)\,\mathrm{d}x}$ 收敛, $\displaystyle{\int_a^{+\infty}f(x)\sin^2x\,\mathrm{d}x}$ 显然收敛.
            若 $\displaystyle{\int_a^{+\infty}f(x)\,\mathrm{d}x}$ 发散, 而
            \begin{align*}
                \int_a^{+\infty}f(x)\sin^2x\,\mathrm{d}x &= \int_a^{+\infty}f(x)\left(\frac{1-\cos{2x}}{2}\right)\,\mathrm{d}x \\
                &= \frac12\left(\int_a^{+\infty}f(x)\,\mathrm{d}x + \int_a^{+\infty}f(x)\cos{2x}\,\mathrm{d}x\right).  
            \end{align*}
            由 Dirichlet 判别法可知 $\displaystyle{\int_a^{+\infty}f(x)\cos{2x}\,\mathrm{d}x}$ 收敛, 故 $\displaystyle{\int_a^{+\infty}f(x)\sin^2x\,\mathrm{d}x}$ 发散.
        \end{proof}
    \item % 5
    \item % 6
    \item % 7
        \begin{proof}
            因为 $\displaystyle{\int_a^{+\infty}f(x)\,\mathrm{d}x}$ 收敛, 根据 Cauchy 收敛原理, 对 $\forall \varepsilon > 0$, $\exists A_0 > a$,
            当 $A_1, A_2 > A_0$ 时, 有
            \[
                \varepsilon > \left|\int_{A_1}^{A_2}f'(x)\,\mathrm{d}x\right| = |f(A_2) - f(A_1)|,   
            \]
            这表明 $\lim\limits_{x\to+\infty}f(x)$ 极限存在. 若 $\lim\limits_{x\to+\infty}f(x)=l$, $l \neq 0$, 不妨设 $l > 0$, 那么对 $\dfrac{l}{2} > 0$, $\exists X > 0$,
            当 $x > X$ 时, 有
            \[
                |f(x) - l| < \frac{l}{2},    
            \]
            也即是 $\dfrac{l}{2} < f(x) < \dfrac{3l}{2}$. 对前面已经找到的 $X$, 对任意的 $\varepsilon > 0$, 存在 $A_0 > a$ 且 $A_0 > X$, 当 $A_2 > A_1 > A_0$ 时, 有
            \[
                0 < \frac{l}{2}(A_2 - A_1) = \int_{A_1}^{A_2}\dfrac{l}{2}\,\mathrm{d}x < \int_{A_1}^{A_2}f(x)\,\mathrm{d}x < \varepsilon,  
            \]   
            而 $\dfrac{l}{2}(A_2 - A_1)$ 是不可能任意小的, 故出现矛盾, 因此 $l = 0$.
        \end{proof}
\end{enumerate}
\end{document}
