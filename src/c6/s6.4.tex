% author: Shuning Zhang
% creation date: 2019-01-06
\documentclass[a4paper, 11pt]{ctexart}
\usepackage[top=2cm, bottom=2cm, left=2.5cm, right=2.5cm]{geometry}
\usepackage{enumerate}
\usepackage{multicol}
\usepackage{amsmath, amssymb}
\newcommand{\dif}{\mathrm{d}}
\begin{document}
\begin{enumerate}
    \item % 1
        \begin{multicols}{4}
            \begin{enumerate}[(1)]
                \item % 1.1
                \item % 1.2
                    $-4\pi$;
                \item % 1.3
                    $5/3$;
                \item % 1.4
                    $\dfrac{4\pi - 3\sqrt3}{6}$;
                \item % 1.5
                \item % 1.6
                    $2 - 2/e$;
                \item % 1.7
                \item % 1.8
                    $\dfrac{a^4}{16}\pi$;
                \item % 1.9
                    $2 - \pi/2$;
                \item % 1.10
                \item % 1.11
                \item % 1.12
            \end{enumerate}
        \end{multicols}
    \item % 2
        {\heiti 证明}\quad \begin{align*}
            \int_{-a}^a f(x)\,\dif x &= \int_{-a}^0 f(x)\,\dif x + \int_0^a f(x)\,\dif x \\
                                     &= \int_0^a f(-x)\,\dif x + \int_0^a f(x)\,\dif x \\
                                     &= 2\int_0^a f(x)\,\dif x.
        \end{align*}
    \item % 3
        {\heiti 证明}\quad \begin{align*}
            \int_{-a}^a f(x)\,\dif x &= \int_{-a}^0 f(x)\,\dif x + \int_0^a f(x)\,\dif x \\
                                     &= \int_0^a f(-x)\,\dif x + \int_0^a f(x)\,\dif x \\
                                     &= -\int_0^a f(x)\,\dif x + \int_0^a f(x)\,\dif x \\
                                     &= 0.
        \end{align*}
    \item % 4
    \item % 5
    \item % 6
    \item % 7
    \item % 8
    \item % 9
    \item % 10
    \item % 11
    \item % 12
    \item % 13
    \item % 14
\end{enumerate}
\end{document}