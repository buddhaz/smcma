% %@author Shuning Zhang
% %@date 2019-12-03

% \documentclass[12pt]{ctexart}

% \usepackage[top=2cm, bottom=2cm, left=2.5cm, right=2.5cm]{geometry}

% \usepackage{enumerate}
% \usepackage{amsmath, amssymb, amsfonts, amsthm}

% \begin{document}

% \pagestyle{empty}

\begin{center}
    {\heiti 练习题 2.3}
\end{center}

\begin{enumerate}
    \item % 1
        略.
    \item % 2
        {\heiti 证明}\quad 先证明存在性, 用反证法. 假设 $f$ 不存在不动点. 因为 $f \circ f$ 有唯一的不动点 $x$,
        所以 $f(f(x)) = x$. 又因 $f$ 不存在不动点, 所以 $f(x) = y \ne x$. 那么
        \[
            f(y) = f(f(x)) = x,   
        \]
        即 $f(y) = x$. 更进一步, 有 $f(f(y)) = f(x) = y$, 即 $y$ 也是 $f \circ f$ 的一个不动点. 这与题意 $f \circ f$ 有唯一的不动点相矛盾.
        因此 $f$ 必然存在不动点.
        
        再证明唯一性, 依然用反证法. 假设 $x$ 和 $y$ 都是 $f$ 的不动点, 并且 $x \neq y$, 则有 $f(x) = x$, $f(y) = y$.
        那么
        \begin{gather*}
            f(f(x)) = f(x) = x, \\
            f(f(y)) = f(y) = y.
        \end{gather*}
        这与题意 $f \circ f$ 有唯一的不动点相矛盾. 因此 $f$ 的不动点是唯一的.
    \item 3
    \item 4
    \item % 5
        \begin{enumerate}[(1)]
            \item % 5.1
                $f^n(x) = \dfrac{x}{\sqrt{1 + nx^2}}$;
            \item % 5.2
                $f^n(x) = \dfrac{x}{1 + nbx}$.
        \end{enumerate}
    \item % 6
        {\heiti 证明}\quad 根据题意, 有 $f(1) = f(1 + 0) = f(1) + f(0)$, 因此 $f(0) = 0$. 进一步, 有
        \[
            0 = f(0) = f(x - x) = f(x) + f(-x),    
        \]
        即 $f(-x) = -f(x)$. 说明 $f$ 是一个奇函数. 因此只需在 $[0, +\infty)$ 上进行讨论.
        
        设有理数 $x = p/q \geqslant 0$, 即 $p = qx$, 则有
        \[
            f(p) = f(qx) = f(\underbrace{x + x + \cdots + x}_{\text{$q$ 项}}) = qf(x).
        \]
        那么
        \[
            f(x) = \frac{f(p)}{q} = \frac{f(\overbrace{1 + 1 + \cdots + 1}^{\text{$p$ 项}})}{q} = \frac{pf(1)}{q} = \frac pq f(1) = xf(1).    
        \]
    \item % 7
        {\heiti 证明} 根据题意, 对 $\forall y > 0$, 有
        \[
            f(x + y) = f(x).    
        \]
        这正表明 $f$ 在 $\mathrm{R}$ 上对任意的两点 $x$, $x + y$ 有相同的函数值. 因此 $f$ 必是一个常值函数.  
    \item 8
    \item % 9
        {\heiti 证明} 设 $f(x)$ 是定义在 $(-a, a)$ 上的一个函数, 则有
        \[
            f(x) = \frac{f(x) + f(-x) + f(x) - f(-x)}{2} = \frac{f(x) + f(-x)}{2} + \frac{f(x) - f(-x)}{2}.    
        \]
        令 $g(x) = \dfrac{f(x) + f(-x)}{2}$, $h(x) = \dfrac{f(x) - f(-x)}{2}$, 显然有
        \begin{gather*}
            g(-x) = g(x), \\
            h(-x) = -h(x).
        \end{gather*}
    \item % 10
        \begin{enumerate}[(1)]
            \item % 10.1
                {\heiti 证明}\quad 对双曲正弦函数, 有
                \[
                    \sinh(-x) = \frac{\mathrm{e}^{-x} - \mathrm{e}^x}{2} = -\frac{\mathrm{e}^x - \mathrm{e}^{-x}}{2} = -\sinh x.    
                \]
                对双曲余弦函数, 有
                \[
                    \cosh(-x) = \frac{\mathrm{e}^{-x} + \mathrm{e}^x}{2} = \frac{\mathrm{e}^x + \mathrm{e}^{-x}}{2} = \cosh x.     
                \]
            \item % 10.2
                {\heiti 证明}\quad 根据题意, 有
                \begin{align*}
                    \cosh^2x - \sinh^2x &= \frac{(\mathrm{e}^x + \mathrm{e}^{-x})^2}{4} - \frac{(\mathrm{e}^x - \mathrm{e}^{-x})^2}{4} \\
                                        &= \frac{\mathrm{e}^{2x} + \mathrm{e}^{-2x} + 2\mathrm{e}^x\mathrm{e}^{-x} - (\mathrm{e}^{2x} + \mathrm{e}^{-2x} - 2\mathrm{e}^x\mathrm{e}^{-x})}{4} \\
                                        &= \frac{2 \cdot \mathrm{e}^0 + 2 \cdot \mathrm{e}^0}{4} \\
                                        &= 1.
                \end{align*}
        \end{enumerate}
    \item % 11
        对双曲正弦函数, 令 $u = \mathrm{e}^x$, 则有
        \[
            u^2 - 2yu - 1 = 0,    
        \]
        解得 $u = y \pm \sqrt{y^2 + 1}$. 因为 $u = \mathrm{e}^x > 0$, 所以 $u = y + \sqrt{y^2 + 1}$, 即
        \[
            x = \ln(y + \sqrt{y^2 + 1}).    
        \]
        同理, 可求得双曲余弦的反函数为 $x = \ln(y + \sqrt{y^2 - 1})$.
\end{enumerate}

% \end{document}