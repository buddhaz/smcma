% %@author Shuning Zhang
% %@date 2019-12-04

% \documentclass[12pt]{ctexart}

% \usepackage[top=2cm, bottom=2cm, left=2.5cm, right=2.5cm]{geometry}

% \usepackage{enumerate}
% \usepackage{amsmath, amsfonts, amssymb, amsthm}

% \begin{document}

% \pagestyle{empty}

\begin{center}
    {\heiti 练习题 2.4}
\end{center}

\begin{enumerate}
    \item % 1
        对 $\forall \varepsilon > 0$, $\exists \delta > 0$, 当 $0 < x_0 - x < \delta$ 时, 有 $|f(x) - 1| < \varepsilon$.
    \item % 2
    \item % 3
        \begin{enumerate}[(1)]
            \item % 3.1
                {\heiti 证明}\quad 对 $\forall \varepsilon > 0$, $\exists \delta > 0$, 当 $0 < |x - x_0| < \delta$ 时, 有
                \[
                    ||f(x)| - |A|| \leqslant |f(x) - A| < \varepsilon.    
                \]
                这便证明了 $\lim\limits_{x\to x_0}|f(x)| = |A|$.
            \item % 3.2
                {\heiti 证明}\quad 因为 $f(x)$ 的极限存在, 所以 $f(x)$ 是有界的, 即 $|f(x)| \leqslant M$.
                对 $\forall \varepsilon > 0$, 取 $\delta = \varepsilon/(M + |A|)$, 当 $0 < |x - x_0| < \delta$ 时, 有
                \[
                    |f^2(x) - A^2| = |f(x) + A||f(x) - A| \leqslant (|f(x)| + |A|)|f(x) - A| < \varepsilon.
                \]
            \item % 3.3
                略.
            \item % 3.4
                略.
        \end{enumerate}
    \item % 4
        \begin{enumerate}[(1)]
            \item % 4.1
                {\heiti 证明}\quad 对 $\forall \varepsilon > 0$, 取 $\delta = \min(1, \varepsilon/19)$, 当 $0 < |x - 2| < \delta$ 时, 有
                \[
                    |x^3 - 8| = |x - 2||x^2 + 2x + 2^2| < 19|x - 2| < \varepsilon.   
                \]
            \item % 4.2
                {\heiti 证明}\quad 对 $\forall \varepsilon > 0$, 取 $\delta = \min(1, 30\varepsilon)$, 当 $0 < |x - 3| < \delta$ 时, 有
                \[
                    \left|\frac{x-3}{x^2-9} - \frac16\right| = \left|\frac{1}{x+3} - \frac16\right| = \frac{|x-3|}{6|x+3|} < \frac{|x-3|}{30} < \varepsilon.
                \]
            \item % 4.3
                {\heiti 证明}\quad 对 $\forall \varepsilon > 0$, 取 $\delta = \min(1, \varepsilon/11)$, 当 $0 < |x - 1| < \delta$ 时, 有
                \begin{align*}
                    \left| \frac{x^4-1}{x-1} - 4 \right| &= |(x^2+1)(x+1) - 4| = |x^3 + x^2 + x - 3| \\
                                                         &= |x^3 - 1 + x^2 - 1 + x - 1| = |x - 1||x^2 + 2x + 3| < 11|x - 1| < \varepsilon.
                \end{align*}
            \item % 4.4
                略.
            \item % 4.5
                略.
        \end{enumerate}
    \item % 5
        \begin{enumerate}[(1)]
            \item % 5.1
                $f(2+) = 4$, $f(2-) = -2a$;
            \item % 5.2
                $a = -2$.
        \end{enumerate}
    \item % 6
        {\heiti 证明}\quad 令 $\lim\limits_{x\to x_0}f(x) = l > a$. 对 $\varepsilon_0 = l - a > 0$, $\exists \delta > 0$, 当 $0 < |x - x_0| < \delta$ 时, 有
        \[
            |f(x) - l| < \varepsilon_0,    
        \]
        即 $a = l - \varepsilon_0 < f(x)$.
    \item % 7
        {\heiti 证明}\quad 对 $\varepsilon_0 = \dfrac{f(x_0+) - f(x_0-)}{2} > 0$, $\exists \delta_1 > 0$, 当 $0 < x_0 - x < \delta_1$ 时, 有
        \[
            |f(x) - f(x_0-)| < \varepsilon_0,    
        \]
        即 $f(x) < f(x_0-) + \varepsilon_0$. 同时, $\exists \delta_2 > 0$, 当 $0 < y - x_0 < \delta_2$ 时, 有
        \[
            |f(y) - f(x_0+)| < \varepsilon_0,    
        \]
        即 $f(x_0+) - \varepsilon_0 < f(y)$. 现取 $\delta = \min(\delta_1, \delta_2)$, 当 $0 < x_0 - x < \delta$, $0 < y - x_0 < \delta$ 时, 有
        \[
            f(x) < f(x_0-) + \varepsilon_0 = f(x_0+) - \varepsilon_0 < f(y).
        \]
    \item % 8
    \item % 9
        存在一个 $\varepsilon_0 > 0$, 对任意的 $\delta > 0$, 即使 $0 < |x - x_0| < \delta$, 依然有 $|f(x) - l| \geqslant \varepsilon_0$.
    \item % 10
    \item % 11
        \begin{enumerate}[(1)]
            \item % 11.1
                $-1$;
            \item % 11.2
                $0$;
            \item % 11.3
                $\lim\limits_{x\to1}\dfrac{x^m-1}{x-1} = \lim\limits_{x\to1}\dfrac{1-x^m}{1-x} = \lim\limits_{x\to1}(1 + x + \cdots + x^{m-1}) = m$;
            \item % 11.4
                $\lim\limits_{x\to1}\dfrac{x^m-1}{x^n-1} = \lim\limits_{x\to1}\dfrac{(1-x^m)/(1-x)}{(1-x^n)/(1-x)} = \dfrac mn$;
            \item % 11.5
                $1/2$;
            \item % 11.6
                $1$;
            \item % 11.7
                $1/m$;
            \item % 11.8
                $\dfrac{m(m+1)}{2}$.
        \end{enumerate}
    \item % 12
        \begin{enumerate}[(1)]
            \item % 12.1
                $a/b$;
            \item % 12.2
                $2$;
            \item % 12.3
                $1$;
            \item % 12.4
                $1$;
            \item % 12.5
                $\sin x$;
            \item % 12.6
                $\cos x$;
            \item % 12.7
                $\lim\limits_{x\to0}\dfrac{1-\cos x\cos 2x\cdots\cos nx}{x^2}$ \\
                $= \lim\limits_{x\to0} \dfrac{1 - (\cos2x \cdots \cos nx) + (\cos2x \cdots \cos nx) -\cos x \cdots \cos nx}{x^2}$ \\
                $= \lim\limits_{x\to0} \dfrac{1 - \cos2x \cdots \cos nx}{x^2} + \lim\limits_{x\to0} (\cos2x \cdots \cos nx) \dfrac{1 - \cos x}{x^2}$ \\
                $= \lim\limits_{x\to0} \dfrac{1 - \cos3x \cdots \cos nx}{x^2} + \lim\limits_{x\to0} (\cos3x \cdots \cos nx) \dfrac{1 - \cos2x}{x^2} + \dfrac12$ \\
                $= \lim\limits_{x\to0} \dfrac{1 - \cos nx}{x^2} + \dfrac{(n-1)^2}{2} + \dfrac{(n-2)^2}{2} + \cdots + \dfrac12$ \\
                $= \dfrac{1 + 2^2 + \cdots + n^2}{2}$ \\
                $= \dfrac{n(n+1)(2n+1)}{12}$;
            \item % 12.8
                $\lim\limits_{n\to\infty}\cos\dfrac{x}{2}\cos\dfrac{x}{2^2}\cdots\cos\dfrac{x}{2^n}$ \\
                $= \lim\limits_{n\to\infty}\cos\dfrac{x}{2}\cos\dfrac{x}{2^2}\cdots\cos\dfrac{x}{2^n} \cdot \lim\limits_{n\to\infty} \dfrac{\sin(x/2^n)}{x/2^n}$ \\
                $= \lim\limits_{n\to\infty} \dfrac{2^{n-1}}{x} \cos\dfrac{x}{2} \cos\dfrac{x}{2^2} \cdots \cos\dfrac{x}{2^{n-1}}\left(2\cos\dfrac{x}{2^n}\sin\dfrac{x}{2^n}\right)$ \\
                $= \lim\limits_{n\to\infty} \dfrac{2^{n-2}}{x} \cos\dfrac{x}{2} \cos\dfrac{x}{2^2} \cdots \left(2\cos\dfrac{x}{2^{n-1}}\sin\dfrac{x}{2^{n-1}}\right)$ \\
                $= \lim\limits_{n\to\infty} \dfrac1x\left(2\cos\dfrac{x}{2}\sin\dfrac{x}{2}\right)$ \\
                $= \lim\limits_{n\to\infty} \dfrac{\sin x}{x}$ \\
                $= \dfrac{\sin x}{x}$.
        \end{enumerate}
    \item % 13
        \begin{enumerate}[(1)]
            \item % 13.1
                $1$;
            \item % 13.2
                $0$;
            \item % 13.3
                $5/8$;
            \item % 13.4
                $3/2$.
        \end{enumerate}
    \item % 14
        {\heiti 证明}\quad 考察 $n!\mathrm{e}$, 有
        \[
            n!\mathrm{e} = n! \sum_{k=0}^\infty\frac{1}{k!} = n!\left(\sum_{k=0}^n\frac{1}{k!} + \sum_{k=n+1}^\infty\frac{1}{k!}\right) = n! \sum_{k=0}^n\frac{1}{k!} + n!\sum_{k=n+1}^\infty\frac{1}{k!} = A + B.
        \]
        显然 $A \in \mathrm{N}^*$, 对于 $B$, 则有
        \[
            B = n!\sum_{k=n+1}^\infty\frac{1}{k!} = \frac{1}{n+1} + \frac{1}{(n+1)(n+2)} + \cdots.    
        \]
        
        现在来证明 $\dfrac{1}{n+1} < B < \dfrac{1}{n-1}$. 不等式 $\dfrac{1}{n+1} < B$ 是显然成立的, 对于 $B < \dfrac{1}{n-1}$, 有
        \[
            B = \frac{1}{n+1} + \frac{1}{(n+1)(n+2)} + \cdots < \frac1n + \frac{1}{n^2} + \cdots = \sum_{i=1}^\infty\left(\frac1n\right)^i,
        \]
        其中
        \[
            \sum_{i=1}^\infty\left(\frac1n\right)^i = \lim_{m\to\infty}\sum_{i=1}^m\left(\frac1n\right)^i = \lim_{m\to\infty}\frac{\frac1n\left(1 - \left(\frac1n\right)^m\right)}{1 - \frac1n} = \frac{1}{n-1}.   
        \]
        
        现在, 则有
        \begin{align*}
            n\sin(2\pi n!\mathrm{e}) &= n\sin\left(2\pi (A+B)\right) \\
                                     &= n\sin(2A\pi)\cos(2B\pi) + n\sin(2B\pi)\cos(2A\pi) \\
                                     &= n\sin(2B\pi). 
        \end{align*}
        因为 $\dfrac{1}{n+1} < B < \dfrac{1}{n-1}$, 所以 $\lim\limits_{n\to\infty}B = 0$, 那么
        \[
            \lim\limits_{n\to\infty} n\sin(2B\pi) = \lim\limits_{n\to\infty} 2\pi nB \frac{\sin(2B\pi)}{2B\pi} = 2\pi\lim\limits_{n\to\infty}nB. 
        \]
        因为 $\dfrac{n}{n+1} < nB < \dfrac{n}{n-1}$, 所以 $\lim\limits_{n\to\infty}nB = 1$, 即
        \[
            \lim\limits_{n\to\infty} n\sin(2\pi n!\mathrm{e}) = 2\pi.    
        \]
    \item % 15
        对 $t_0 = 0$ 的任何一个 $B_\eta(t_0)$, 都能找到一个无理数 $p/q \in B_\eta(t_0)$, 使得 $g(p/q) = 1/q = x_0$. 这违背了复合函数求极限的条件.
    \item % 16
\end{enumerate}

% \end{document}