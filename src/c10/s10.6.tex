% @author Shuning Zhang
% @date 2019-03-14
\documentclass[a4paper, 11pt]{ctexart}
\usepackage{amsfonts, amsmath, amssymb, amsthm}
\usepackage{color}
\usepackage{enumerate}
\usepackage[bottom=2cm, left=2.5cm, right=2.5cm, top=2cm]{geometry}
\usepackage{multicol}
\begin{document}
\begin{enumerate}
    \item % 1
        \begin{multicols}{2}
            \begin{enumerate}[(1)]
                \item % 1.1
                    $0$;
                \item % 1.2
            \end{enumerate}
        \end{multicols}
    \item % 2
    \item % 3
        \begin{proof}
            令 $x+y=u$, $-x+y=v$, 那么 $-1 \leq u \leq 1$, $-1 \leq v \leq 1$, 而且
            \[
                \left|\frac{\partial(x,y)}{\partial(u,v)}\right| = \frac12.    
            \]
            因此
            \[
                \iint\limits_{|x|+|y|\leq1}f(x+y)\,\mathrm{d}x\mathrm{d}y = \frac12\int_{-1}^{1}f(u)\,\mathrm{d}u\int_{-1}^1\,\mathrm{d}v = \int_{-1}^{1}f(u)\,\mathrm{d}u.   
            \]
            再把积分变元 $u$ 换为 $t$ 即可.
        \end{proof}
    \item % 4
        \begin{proof}
            令 $xy=u$, $y/x=v$, 那么 $1\leq u\leq 2$, $1 \leq v \leq 4$, 而且
            \[
                \left|\frac{\partial(x,y)}{\partial(u,v)}\right| = \frac{1}{2v}.    
            \]
            故
            \[
                \iint\limits_{D}f(xy)\,\mathrm{d}x\mathrm{d}y = \frac12\int_1^2f(u)\,\mathrm{d}u\int_1^4\frac1v\,\mathrm{d}v = \ln2\int_1^2f(u)\,\mathrm{d}u.    
            \]
            再把积分变元 $u$ 换为 $t$ 即可.
        \end{proof}
    \item % 5
        \begin{multicols}{2}
            \begin{enumerate}[(1)]
                \item % 5.1
                    $-6\pi$;
                \item % 5.2
                \item % 5.3
                \item % 5.4
            \end{enumerate}
        \end{multicols}
    \item % 6
        略.
    \item % 7
        $\dfrac{\pi}{6}ab$.
\end{enumerate}
\end{document}
