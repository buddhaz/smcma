% % @author Shuning Zhang
% % @date 2019-03-11
% \documentclass[a4paper, 11pt]{ctexart}
% \usepackage{amsfonts, amsmath, amssymb, amsthm}
% \usepackage{color}
% \usepackage{enumerate}
% \usepackage[bottom=2cm, left=2.5cm, right=2.5cm, top=2cm]{geometry}
% \usepackage{multicol}
% \begin{document}
\begin{enumerate}
    \item % 1
        \begin{proof}
            由平面点集 $S$ 面积的定义可知
            \[
                \sigma(S) = \int_S1\,\mathrm{d}\sigma = \int_R\chi_S\,\mathrm{d}\sigma.    
            \]
            有面积即表示上式积分可积, 即特征函数 $\chi_S$ 可积, 又特征函数的不连续点的全体 $D(\chi_S) = \partial{S}$,
            因此只能 $\partial{S}$ 是零测集, 又 $\partial{S}$ 是有界闭集, 所以 $\partial{S}$ 是零面积集.
        \end{proof}
    \item % 2
        {\color{red}remained}
    \item % 3
        \begin{proof}
            $S \subset \mathbb{R}^2$ 有面积, 即积分
            \[
                \int_S1\,\mathrm{d}\sigma = \int_R\chi_S\,\mathrm{d}\sigma    
            \]
            存在, 其中 $R$ 是一个闭矩形, 且 $S \subset R^\circ$. 将上式写为 Riemann 和的形式, 则有
            \[
                \sum_{i=1}^n\chi_S(\xi_i)\sigma(R_i).    
            \]
            由于 $\xi_i$ 是任取的, 故对于 $R_i \cap S \not= \varnothing$ 且 $R_i \not\subset S$ 的 $R_i$ 可使得
            \[
                \chi_S(\xi_i) = 0.    
            \]
            因此可忽略那些 $R_i \cap S \not= \varnothing$ 且 $R_i \not\subset S$ 的 $R_i$, 便有
            \[
                \sum_{R_i \cap S \not= \varnothing}\sigma(R_i) = \sum_{R_i \subset S}\sigma(R_i). \qedhere    
            \]
        \end{proof}
\end{enumerate}
% \end{document}
