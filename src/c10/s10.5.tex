% % @author Shuning Zhang
% % @date 2019-03-13
% \documentclass[a4paper, 11pt]{ctexart}
% \usepackage{amsfonts, amsmath, amssymb, amsthm}
% \usepackage{color}
% \usepackage{enumerate}
% \usepackage[bottom=2cm, left=2.5cm, right=2.5cm, top=2cm]{geometry}
% \usepackage{multicol}
% \begin{document}
\begin{enumerate}
    \item % 1
        \begin{multicols}{2}
            \begin{enumerate}[(1)]
                \item % 1.1
                    $-2$;
                \item % 1.2
                    $32/21$;
                \item % 1.3
                    {\color{red}remained}$e-1/e$; 
                \item % 1.4
                    $a^4/2$;
                \item % 1.5
                    $0$;
                \item % 1.6
                    $\cos2 + 2\cos1 - \pi + 3$;
                \item % 1.7
                \item % 1.8
                    $6$.
            \end{enumerate}
        \end{multicols}
    \item % 2
        \begin{enumerate}[(1)]
            \item % 2.1
                $\displaystyle{\int_0^1\,\mathrm{d}y\int_{\sqrt{y}}^1f(x,y)\,\mathrm{d}x}$;
            \item % 2.2
                $\displaystyle{\int_0^1\,\mathrm{d}y\int_{e^y}^ef(x,y)\,\mathrm{d}x}$;
            \item % 2.3
                $\displaystyle{\int_0^1\,\mathrm{d}y\int_y^{\sqrt{y}}}f(x,y)\,\mathrm{d}x$;
            \item % 2.4
                $\displaystyle{\int_a^b\,\mathrm{d}y\int_y^bf(x,y)\,\mathrm{d}x}$;
            \item % 2.5
                $\displaystyle{\int_{-1}^0\,\mathrm{d}y\int_{-\sqrt{1-y^2}}^{\sqrt{1-y^2}}f(x,y)\,\mathrm{d}x + \int_0^1\,\mathrm{d}y\int_{-\sqrt{1-y}}^{\sqrt{1-y}}f(x,y)\,\mathrm{d}x}$;
            \item % 2.6
                $\displaystyle{\int_{-1}^0\,\mathrm{d}y\int_0^{y+1}f(x,y)\,\mathrm{d}x + \int_0^1\,\mathrm{d}y\int_0^{-y+1}f(x,y)\,\mathrm{d}x}$;
            \item % 2.7
                {\color{red}remained}$\displaystyle{\int_{-1}^1\,\mathrm{d}y\int_{0}^{\arccos{y}}f(x,y)\,\mathrm{d}x}$.
        \end{enumerate}
    \item % 3
        \begin{proof}
            改变累次积分的次序, 则有
            \[
                \int_0^af(x)\,\mathrm{d}x\int_0^xf(y)\,\mathrm{d}y = \int_0^af(y)\,\mathrm{d}y\int_y^af(x)\,\mathrm{d}x.   
            \]
            因此
            \begin{align*}
                \int_0^xf(y)\,\mathrm{d}y &= \int_y^af(x)\,\mathrm{d}x \\
                &= \int_y^0f(x)\,\mathrm{d}x + \int_0^af(x)\,\mathrm{d}x \\
                &= -\int_0^yf(x)\,\mathrm{d}x + \int_0^af(x)\,\mathrm{d}x.
            \end{align*}
            将积分变元换为 $t$, 同一个函数 $f$ 对 $0$ 到 $x$ 和 $0$ 到 $y$ 是一样的, 因此
            \[
                \int_0^xf(t)\,\mathrm{d}t = \frac{1}{2}\int_0^af(t)\,\mathrm{d}t.   
            \]
            这样便有
            \[
                \int_0^af(x)\,\mathrm{d}x\int_0^xf(y)\,\mathrm{d}y = \int_0^af(t)\,\mathrm{d}t\int_0^xf(t)\,\mathrm{d}t = \frac12\left(\int_0^af(t)\,\mathrm{d}t\right)^2. \qedhere   
            \]
        \end{proof}
    \item % 4
        \begin{proof}
            改变累次积分的次序即可, 那么
            \[
                \int_0^a\,\mathrm{d}x\int_0^xf(y)\,\mathrm{d}y = \int_0^af(y)\,\mathrm{d}y\int_y^a\,\mathrm{d}x = \int_0^a(a-y)f(y)\,\mathrm{d}y.   
            \]
            再将积分变元换为 $t$ 即可.
        \end{proof}
    \item % 5
    \item % 6
    \item % 7
        \begin{proof}
            根据第 5 题的结论, 则有
            \begin{align*}
                \int_0^a\,\mathrm{d}x\int_0^x\,\mathrm{d}y\int_0^yf(z)\,\mathrm{d}z &= \int_0^af(z)\,\mathrm{d}z\int_z^a\,\mathrm{d}y\int_y^a\,\mathrm{d}x \\
                &= \int_0^af(z)\,\mathrm{d}z\int_z^a(z-y)\,\mathrm{d}z \\
                &= \left.-\frac12\int_0^a(a-y)^2\right|_z^af(z)\,\mathrm{d}z \\
                &= \frac12\int_0^a(a-z)^2f(z)\,\mathrm{d}z. \qedhere  
            \end{align*}
        \end{proof}
    \item % 8
        根据推论 10.4.1, 则有
        \[
            \iint\limits_{x^2+y^2\leq r^2}f(x,y)\,\mathrm{d}x\mathrm{d}y = f(\boldsymbol{\xi})\pi r^2,   
        \]
        其中 $\boldsymbol{\xi} \in \{(x,y) : x^2+y^2 \leq r^2\}$. 因此
        \[
            \lim_{r\to0}\frac{1}{\pi r^2} \iint\limits_{x^2+y^2\leq r^2}f(x,y)\,\mathrm{d}x\mathrm{d}y = \lim_{r\to0}f(\boldsymbol{\xi}) = f(\boldsymbol{\xi}).   
        \]
\end{enumerate}
% \end{document}
